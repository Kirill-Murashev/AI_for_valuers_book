\documentclass[]{scrartcl}
\input{standard_preamble.tex}

\title{Очень краткое введение в~математический анализ~для~оценщиков}
\author{К.\,А.\,Мурашев}

\begin{document}

\maketitle

\begin{abstract}
	Какую~бы работу не~выполнял оценщик, во~всех случаях он~имеет дело с~информацией и~данными. Часто эти~данные представляют собой числа либо могут быть формализованы иным образом. В~любом случае требуется алгоритмическая обработка входных данных и~преобразование их~в~информацию, а~в~некоторых случаях "--- в~знания. Целью данного фрагмента является формирование общих представлений об~основных понятиях и~методах математического анализа, необходимых современному оценщику. Автор постарался прибегать к~минимальному числу формул и~сложных определений, хотя это~и~не~вполне получилось. Поскольку конечной целью всей работы является цифровизация оценочной деятельности, в~тексте приводятся короткие листинги на~языках R и~Python, позволяющие реализовать то, о~чём говорится в~тексте. 
\end{abstract}

\tableofcontents
\section{Некоторые особенности материала}
\subsection{Список обозначений}
Все~обозначения, используемые в~материале, соответствуют общепринятым в~математике. Далее приводится краткая шпаргалка~\cite{CSC:intro-in-matan}.
\begin{description}
	\item[$\mathbb{N}$] "--- множество \textbf{натуральных чисел}, т.\,е.~таких чисел, которые получаются при~счёте объктов:~$1, 2, 3, 4, 5\ldots$. Наименьшее натуральное число "--- $1$. Наибольшего натурального числа не~существует. \textbf{Натуральный~ряд} "--- это последовательность всех натуральных чисел. В~натуральном ряду каждое число больше предыдущего на~1. Натуральный ряд бесконечен, наибольшего натурального числа в~нём~не~существует.
	\item[$\mathbb{Z}$] "--- множество \textbf{целых чисел}, включающее в~себя \emph{натуральные числа}, все~числа противоположные им~по~знаку, а~также число ноль.
	\item[$\mathbb{Q}$] "--- множество \textbf{рациональных чисел}, т.\,е.~дробей вида $\frac{m}{n}$, где~ $m \in \mathbb{Z}$ и~$n \in \mathbb{N}$.
	[\item[$\mathbb{I}$] "--- множество \textbf{иррациональных чисел}, т.\,е. , бесконечных непериодических дробей. Примерами являются $\sqrt{2}$, число $\pi \approx 3.15159$, число $e \approx 2.718281828459$ и~т.\,д.
	\item[$\mathbb{R}$] "--- множество \textbf{вещественных~(действительных) чисел}, содержащее в~себе все~\emph{рациональные} и~\emph{иррациональные} числа.
	\item[$\in$] "--- оператор принадлежности. Запись $x \in \mathbb{Z}$ означает <<x~принадлежит к~множеству \emph{целых чисел}>> либо <<x~является \emph{целым числом}>>.
	\item[$x\in X:a$] "--- означает подмножество множества $X$, состоящее из элементов, удовлетворяющих условию $a$.
	\item[$A\bigcup$B] "--- объединение множеств $A$ и~$B$.
	\item[$A\bigcap$B] "--- пересечение множеств $A$ и~$B$.
	\item[$A\subset$B] "--- множество~$A$ является подмножеством множества~$B$.
	\item[$\bigcup \limits_{k=1}^{n}A_k$] "--- объединение всех множеств $A_1, A_2,\ldots, n$.
	\item[$\bigcap \limits_{k=1}^{n}A_k$] "--- пересечение всех множеств $A_1, A_2,\ldots, n$.
	
	\item[{$\left[ a,b \right]$}] "--- \textbf{отрезок} между числами $a$ и~$b$ т.\,е.~множество вещественных чисел, лежащих между числами a~и~b, включая сами числа a~и~b. На~математическом языке это~можно записать как~$[a, b] = {x \in \mathbb{R}: a \leq x \leq b }$. При~$a=b$ отрезок состоит из~одной точки и~называется \emph{вырожденным отрезком}.
	\item[$(a, b)$] "--- \textbf{интервал} между числами $a$ и~$b$ т.\,е.~множество вещественных чисел, лежащих строго между $a$~и~$b$, не~включая их~самих. На~математическом языке это~можно записать как~$(a, b) = {x \in \mathbb{R}: a < x < b }$.
	\item[{$\left[ a, b), (a, b\right] $}] "--- \textbf{полуинтервалы} между числами $a$ и~$b$: $[a,b) = \{x \in \mathbb{R}: a \leq x < b\}$, $(a,b] = \{x \in \mathbb{R}: a < x \leq b\}$.
	\item[$[a, +\infty)$] "--- луч: $[a, +\infty)] = \{x \in \mathbb{R}: a \leq x\}$.
	\item[($a, +\infty$)] "--- луч: $(a, +\infty)] = \{x \in \mathbb{R}: a < x\}$.
	\item[{$(-\infty, b]$}] "--- луч: $(- \infty, b] = \{x \in \mathbb{R}: x \leq b\}$.
	\item[$(-\infty, b)$] "--- луч: $(-\infty, b) = \{x \in \mathbb{R}: x < b\}$.
	\item[Промежуток] "--- \emph{отрезок}, \emph{интервал} либо \emph{полуинтервал}.Промежуток любого из четырех типов обозначается $\langle a, b \rangle$. В~рамках одного утверждения запись $\langle a, b \rangle$ всегда обозначает один и~тот же~подвид промежутка.
	\item[$\langle a, b \rangle$] "--- любой из~двух промежутков  $(a,b)$ и~$[a,b)$.
	\item[$\forall$] "--- квантор всеобщности, используется для~сокращённой записи вместо понятий <<каждый>>, <<любой>>, или~<<для~всякого>>, <<для любого>> и~т.\,п.
	\item[$\exists$] "--- квантор существования, используется для~сокращённой записи вместо слов <<найдётся>>, <<существует>> и~т.~п.
	\item[$\sum \limits_{k=n}^{n} a_k$] "--- сумма чисел $a_k$ по~$k$ от~$m$ до~$n$, т.\,е.~$a_m + a_{m+1}+a_{m+1}+\ldots+a_n$.
	\item[$f:X \textrightarrow Y$] "--- функция, заданная на~множестве $X$, множество значений которой лежит в~$Y$ (но~необязательно с~ним~совпадает).
	\item[label] description
\end{description}
\section{Последовательности}
\subsection{Понятие множества}
Под~\emph{множеством} понимают совокупность, класс или~собрание объектов безразлично какой природы. Согласно определению основоположника теории множеств \href{https://ru.wikipedia.org/wiki/Кантор,_Георг}{Г.\,Кантора}~\cite{Wiki:Kantor}, множество "--- это~собрание предметов одинаковых или~различных между собой, мыслимое как единое целое. Собрание предметов рассматривается как~один предмет. Не~следует понимать множество как~совокупность действительно существующих предметов, принадлежность предметов одному множества не~требует от~них~сосуществования во~времени и~пространстве. В~логике множество понимается как~абстрактный объект, в~котором каждый предмет рассматривается с~точки зрения признаков, по~которым данный предмет принадлежит данному множеству. В~множестве предметы становятся неразличимыми друг от~друга по~признакам и~их~только по~именам.

Объект, принадлежащий данному множеству, называется его~\textbf{элементом}. Множество обозначается заглавными латинскими буквами $А, В, С$. Элементы, входящие в~множество, обозначаются строчными латинскими буквами и~заключаются в~фигурные скобки: ${a,b,c}$.

Множество, содержащее конечное число элементов, называется \textbf{конечным}, а~бесконечное число элементов "--- \textbf{бесконечным}.

Два множества называются \textbf{равными}, если содержат одинаковые элементы $(А={2,4,8}=В={2,2,4,8})$.

Элементами множества могут быть другие множества $А={{2,3},{4,5}}$. При~этом $А={{2,3},{4,5}}\neq В= {2,3,4,5}$.

Часть множества называется подмножеством данного множества. Например, $A \subset B$.
EEnd.\cite{Studopedia:mnozhestvo}
\subsection{Понятие отображения}
Большую роль в~математике имеет установление связей между двумя множествами $X$ и~$Y$, связанное с~рассмотрением пар объектов, образованных из~элементов первого множества и~соответствующих им~элементов второго множества. Особое значение при~этом имеет \emph{отображение множеств}.

Пусть $X$ и~$Y$ "--- произвольные множества. Отображением множества $X$ на~множество $Y$ называется $\forall$ правило $f$, по~которому каждому элементу множества $X$ сопоставляется вполне определённый~(единственный) элемент множества~$Y$. Тот~факт, что~$f$ есть отображение $X$ в~$Y$, кратко записывают в~виде: $f:X->Y$.

\section{Функции и~непрерывность)}
\section{Производные}
\section{Интегралы}




\nocite{CSC:intro-in-matan}

\printbibliography[title=Источники информации]

\end{document}
