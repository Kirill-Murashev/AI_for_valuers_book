\documentclass[]{scrartcl}
% Лицензия
% Apache License Version 2.0, January 2004
% http://www.apache.org/licenses/
% Copyright [2020] [Kirill A. Murashev]
% Licensed under the Apache License, Version 2.0 (the "License"); you may not use this file except in compliance with the License. You may obtain a copy of the License at
% http://www.apache.org/licenses/LICENSE-2.0
% Unless required by applicable law or agreed to in writing, software
% distributed under the License is distributed on an "AS IS" BASIS,
% WITHOUT WARRANTIES OR CONDITIONS OF ANY KIND, either express or implied.
% See the License for the specific language governing permissions and limitations under the License.


%%% Работа с русским языком
\usepackage{cmap}					% поиск в PDF
\usepackage{mathtext} 				% русские буквы в формулах
\usepackage{fontspec}
\defaultfontfeatures{Renderer=Basic,Ligatures={TeX}}
\setmainfont{CMU Serif}
\setsansfont{CMU Sans Serif}
\setmonofont{CMU Typewriter Text}
\usepackage[english,russian]{babel}
%\usepackage[T1,T2A]{fontenc}			% кодировка
%\usepackage[lutf8]{luainputenc}			% кодировка исходного текста
%\usepackage[english,russian]{babel}	% локализация и переносы
\usepackage{indentfirst}            % красная строка
\usepackage{misccorr}               % доработки для babel
\frenchspacing                      % французский стиль пробелов

%\usepackage{beton} %изменение шрифта для тёмной цветовой схемы
%\usepackage{concrete}
%%% Дополнительная работа с математикой
\usepackage{amsmath,amsfonts,amssymb,amsthm,mathtools} % AMS
\usepackage{icomma} % "Умная" запятая: $0,2$ --- число, $0, 2$ --- перечисление

%% Номера формул
%\mathtoolsset{showonlyrefs=true} % Показывать номера только у тех формул, на которые есть \eqref{} в тексте.
%\usepackage{leqno} % Нумерация формул слева

%% Перенос знаков в формулах (по Львовскому)
\newcommand*{\hm}[1]{#1\nobreak\discretionary{}
	{\hbox{$\mathsurround=0pt #1$}}{}}

%%% Работа с картинками
\usepackage{graphicx}  % Для вставки рисунков
\graphicspath{{Images/}}  % папки с картинками
\setlength\fboxsep{3pt} % Отступ рамки \fbox{} от рисунка
\setlength\fboxrule{1pt} % Толщина линий рамки \fbox{}
\usepackage{wrapfig} % Обтекание рисунков текстом

%%% Работа с таблицами
\usepackage{array, tabularx, tabulary, booktabs, xtab} % Дополнительная работа с таблицами
\usepackage{longtable}  % Длинные таблицы
\usepackage{multirow} % Слияние строк в таблице

%%% Теоремы
\theoremstyle{plain} % Это стиль по умолчанию, его можно не переопределять.
\newtheorem{theorem}{Теорема}[section]
\newtheorem{proposition}[theorem]{Утверждение}

\theoremstyle{definition} % "Определение"
\newtheorem{corollary}{Следствие}[theorem]
\newtheorem{problem}{Задача}[section]

\theoremstyle{remark} % "Примечание"
\newtheorem*{nonum}{Решение}

%%% Программирование
\usepackage{etoolbox} % логические операторы

\usepackage{lastpage} % Узнать, сколько всего страниц в документе.

\usepackage{keyval}

\usepackage{totcount} % Узнать, сколько всего объектов в документе.

%\usepackage{xcolor-solarized}

%%% Страница
%\usepackage{extsizes} % Возможность сделать 14-й шрифт
%\usepackage{geometry} % Простой способ задавать поля
%	\geometry{top=25mm}
%	\geometry{bottom=35mm}
%	\geometry{left=35mm}
%	\geometry{right=20mm}
%

%\usepackage{fancyhdr} % Колонтитулы

%	\pagestyle{fancy}
%\renewcommand{\headrulewidth}{0pt}  % Толщина линейки, отчеркивающей верхний колонтитул
%\fancyhf{}
%\lhead{Часть \thepart}
%\chead{Глава \thechapter}
%\rhead{Раздел \thesection}
%\lfoot{version 0.251}
%\cfoot{\today} % По умолчанию здесь номер страницы
%\rfoot{\thepage/\ref{LastPage}}
%\pagestyle{fancy}

%\usepackage{setspace} % Интерлиньяж
%\onehalfspacing % Интерлиньяж 1.5
%\doublespacing % Интерлиньяж 2
%\singlespacing % Интерлиньяж 1

\usepackage{soul} % Модификаторы начертания

\usepackage[usenames,dvipsnames,svgnames,table,rgb]{xcolor} % Подключение пакета для задания цвета

%\definecolor{Backcolor}{HTML}{042029} % Задание цвета для фона
%\definecolor{Textcolor}{HTML}{819090} % Задание цвета для текста
%\pagecolor{Backcolor}                 % Подключение тёмной
%\color{Textcolor}                     % темы

\usepackage{csquotes} % Ещё инструменты для ссылок

\usepackage[backend=biber,bibencoding=utf8,sorting=ynt,maxcitenames=5,sortupper=true,date=iso]{biblatex} % подключение пакета для работы с автоматизированной библиографией

%\usepackage[style=authoryear,maxcitenames=2,backend=biber,sorting=nty]{biblatex}

%\renewcommand\bibname{Источники информации} % Переопределение названия для библиографии

\usepackage{multicol} % Несколько колонок

\usepackage{microtype}              %<-- added for better inter word spacing

\usepackage{tabularx}

\usepackage{tikz} % Работа с графикой
\usepackage{pgfplots}
\usepackage{pgfplotstable}

\usepackage{eqlist}

\usepackage{desclist} % Дополнительное окружение для списка Глоссария

\usepackage{lineno} % Нумерация строк

\setcounter{tocdepth}{8} % Глубина оглавления

% подавление висячих строк
\clubpenalty=400 % Разрешение = 300, абсолютный запрет = 10000
\widowpenalty=400 % Увеличиваем эти числа до тех пор, пока не начнёт увеличиваться количество страниц.

% Выбор между разрежением и переполнением
\tolerance=500 % max=10000, default=200

\looseness=-1 % иногда можно удлинять страницу на одну строку.

\hfuzz=2.5pt % иногда можно вылезти за край строки на 2.5 pt.

\usepackage{calc} % Вычисления

\usepackage{scrlayer-scrpage} % Стиль страницы

\usepackage{lineno} % нумерация строк

%\pagestyle{scrpage}

%\usepackage{concrete}

\usepackage{booktabs}

\usepackage[owncaptions]{vhistory} % Log of versions

\usepackage{progressbar} % Формирование линейки, показывающей прогресс в работе

\usepackage{epigraph} % работа с эпиграфами

\usepackage {listings}
\lstloadlanguages{[Latex]Tex, bash, R, Python, SQL}
\lstset{extendedchars=true , % включаем не латиницу
frame=tb, % рамка сверху и снизу
commentstyle=\itshape , % шрифт для комментариев
stringstyle =\ttfamily % шрифт для строк
%keywordstyle=\color{blue}
}

%\usepackage{titling} %дополнительная настройка титульного листа

\setcounter{secnumdepth}{8} % Установка глубины нумерации заголовков

% Работа с гиперрсылками, подключается последним
\usepackage{hyperref}       % Подключение пакета для работы с гиперссылками
\hypersetup{				% Гиперссылки
	unicode=true,           % русские буквы в раздела PDF
	pdftitle={Искусственный интеллект в~оценке стоимости},   % Заголовок
	pdfauthor={К.\,А.~Мурашев},      % Автор
	pdfsubject={Системы поддержки принятия решений, основанные на искусственном интеллекте},      % Тема
	pdfcreator={К.\,А.~Мурашев}, % Создатель
	pdfproducer={К.\,А.~Мурашев}, % Производитель
	pdfkeywords={Искусственный интеллект, машинное обучение, математические методы, оценочная деятельность, цифровая экономика, Data Science, анализ данных} % Ключевые слова
	colorlinks=true,       	% false: ссылки в рамках; true: цветные ссылки
	linkcolor=red,          % внутренние ссылки
	citecolor=green,        % на библиографию
	filecolor=magenta,      % на файлы
	urlcolor=blue           % на URL
}

\usepackage{pgfplots} 
\pgfplotsset{compat=1.15}
\usepackage{mathrsfs}
\usetikzlibrary{arrows}
%\usepackage{url}

%\usepackage{totpages}

%\usepackage[strings]{underscore}

%\author{К.\,А.~Мурашев\thanks {\href{kirill.murashev@tutanota.de}{kirill.murashev@tutanota.de}, \href{https://t.me/Maas\_88}{https://t.me/Maas\_88}, \href{https://www.facebook.com/murashev.kirill}{https://www.facebook.com/murashev.kirill}}}
%\title{\Large Современные системы поддержки принятия решений оценщиками, основанные на~применении методов машинного обучения: практическое руководство по~применению языка программирования R в~повседневной практике оценщика}
%\date{\today}

%\normalsize

% Макрос для рисунков, обтекаемых текстом
\newcommand*{\EpsWrapD}[7]{%
	\begin{wrapfigure}[#5]{#3}{#2 \textwidth} % #3=l,r,L,R
		\begin{center} \sffamily
			\includegraphics*[width= #2 \textwidth ]{#1} % 1-имя файла и метка заодно,
			% 2-ширина рисунка (доля от ширины страницы)
			\vspace{-#7mm} % #7: сократить расстояние между подписью снизу и рисунком
			\caption{\label{fig:#1}#4} % #4 - подпись под рисунком
			\vspace{-#6pt}
		\end{center}% #6: сократить расстояние между подписью снизу и текстом после таблицы 
	\end{wrapfigure}}
%
% макрос для создания таблицы, обтекаемой текстом
\newcommand*{\TableBE}[5]{
	\begin{table}[#1] %\captionabove
		\vspace*{-#5mm}
		\centering \sffamily \caption{\label{tab:#2}#3} \begin{tabular}{#4} \toprule }
		
		\newcommand*{\TableEN}[3]{
			\bottomrule \end{tabular}
		\vspace{-#2mm} \small \begin{flushleft} #1 \end{flushleft}
		\vspace{-#3mm}
\end{table}}


\addbibresource{/home/kaarlahti/TresoritDrive/Methodics/My/AI_for_valuers/Book/AI_for_valuers_book/Basic_principles.bib}
\addbibresource{/home/kaarlahti/TresoritDrive/Methodics/My/AI_for_valuers/Book/AI_for_valuers_book/LaTeX.bib}
\addbibresource{/home/kaarlahti/TresoritDrive/Methodics/My/AI_for_valuers/Book/AI_for_valuers_book/Mathstat.bib}
\addbibresource{/home/kaarlahti/TresoritDrive/Methodics/My/AI_for_valuers/Book/AI_for_valuers_book/Murashev.bib}
\addbibresource{/home/kaarlahti/TresoritDrive/Methodics/My/AI_for_valuers/Book/AI_for_valuers_book/Python.bib}
\addbibresource{/home/kaarlahti/TresoritDrive/Methodics/My/AI_for_valuers/Book/AI_for_valuers_book/R.bib}
\addbibresource{/home/kaarlahti/TresoritDrive/Methodics/My/AI_for_valuers/Book/AI_for_valuers_book/RussianLaws.bib}
\addbibresource{/home/kaarlahti/TresoritDrive/Methodics/My/AI_for_valuers/Book/AI_for_valuers_book/Sci&Tech.bib}
\addbibresource{/home/kaarlahti/TresoritDrive/Methodics/My/AI_for_valuers/Book/AI_for_valuers_book/Valuation.bib}
\addbibresource{/home/kaarlahti/TresoritDrive/Methodics/My/AI_for_valuers/Book/AI_for_valuers_book/ValuationStandards.bib}
\addbibresource{/home/kaarlahti/TresoritDrive/Methodics/My/AI_for_valuers/Book/AI_for_valuers_book/ZHZL.bib}

\pagestyle{headings} 
\markright{Искусственный интеллект в~оценке стоимости}
\usepackage{pgfplots}
\pgfplotsset{compat=1.15}
\usepackage{mathrsfs}
\usetikzlibrary{arrows}

%\usepackage{polyglossia}

%\usepackage{minted}

\newtheorem{Thexmpl}[theorem]{Пример}

\usepackage[inkscapearea=page]{svg}
\usepackage{adjustbox}


\title{Очень краткое введение в~математический анализ~для~оценщиков}
\author{К.\,А.\,Мурашев}

\begin{document}

\maketitle

\begin{abstract}
	Какую~бы работу не~выполнял оценщик, во~всех случаях он~имеет дело с~информацией и~данными. Часто эти~данные представляют собой числа либо могут быть формализованы иным образом. В~любом случае требуется алгоритмическая обработка входных данных и~преобразование их~в~информацию, а~в~некоторых случаях "--- в~знания. Целью данного фрагмента является формирование общих представлений об~основных понятиях и~методах математического анализа, необходимых современному оценщику. Автор постарался прибегать к~минимальному числу формул и~сложных определений, хотя это~и~не~вполне получилось. Поскольку конечной целью всей работы является цифровизация оценочной деятельности, в~тексте приводятся короткие листинги на~языках R и~Python, позволяющие реализовать то, о~чём говорится в~тексте. 
\end{abstract}

\tableofcontents
\section{Некоторые особенности материала}
\subsection{Список обозначений}
Все~обозначения, используемые в~материале, соответствуют общепринятым в~математике. Далее приводится краткая шпаргалка~\cite{CSC:intro-in-matan}.
\begin{description}
	\item[$\mathbb{N}$] "--- множество \textbf{натуральных чисел}, т.\,е.~таких чисел, которые получаются при~счёте объктов:~$1, 2, 3, 4, 5\ldots$. Наименьшее натуральное число "--- $1$. Наибольшего натурального числа не~существует. \textbf{Натуральный~ряд} "--- это последовательность всех натуральных чисел. В~натуральном ряду каждое число больше предыдущего на~1. Натуральный ряд бесконечен, наибольшего натурального числа в~нём~не~существует.
	\item[$\mathbb{Z}$] "--- множество \textbf{целых чисел}, включающее в~себя \emph{натуральные числа}, все~числа противоположные им~по~знаку, а~также число ноль.
	\item[$\mathbb{Q}$] "--- множество \textbf{рациональных чисел}, т.\,е.~дробей вида $\frac{m}{n}$, где~ $m \in \mathbb{Z}$ и~$n \in \mathbb{N}$.
	[\item[$\mathbb{I}$] "--- множество \textbf{иррациональных чисел}, т.\,е. , бесконечных непериодических дробей. Примерами являются $\sqrt{2}$, число $\pi \approx 3.15159$, число $e \approx 2.718281828459$ и~т.\,д.
	\item[$\mathbb{R}$] "--- множество \textbf{вещественных~(действительных) чисел}, содержащее в~себе все~\emph{рациональные} и~\emph{иррациональные} числа.
	\item[$\in$] "--- оператор принадлежности. Запись $x \in \mathbb{Z}$ означает <<x~принадлежит к~множеству \emph{целых чисел}>> либо <<x~является \emph{целым числом}>>.
	\item[$x\in X:a$] "--- означает подмножество множества $X$, состоящее из элементов, удовлетворяющих условию $a$.
	\item[$A\bigcup$B] "--- объединение множеств $A$ и~$B$.
	\item[$A\bigcap$B] "--- пересечение множеств $A$ и~$B$.
	\item[$A\subset$B] "--- множество~$A$ является подмножеством множества~$B$.
	\item[$\bigcup \limits_{k=1}^{n}A_k$] "--- объединение всех множеств $A_1, A_2,\ldots, n$.
	\item[$\bigcap \limits_{k=1}^{n}A_k$] "--- пересечение всех множеств $A_1, A_2,\ldots, n$.
	
	\item[{$\left[ a,b \right]$}] "--- \textbf{отрезок} между числами $a$ и~$b$ т.\,е.~множество вещественных чисел, лежащих между числами a~и~b, включая сами числа a~и~b. На~математическом языке это~можно записать как~$[a, b] = {x \in \mathbb{R}: a \leq x \leq b }$. При~$a=b$ отрезок состоит из~одной точки и~называется \emph{вырожденным отрезком}.
	\item[$(a, b)$] "--- \textbf{интервал} между числами $a$ и~$b$ т.\,е.~множество вещественных чисел, лежащих строго между $a$~и~$b$, не~включая их~самих. На~математическом языке это~можно записать как~$(a, b) = {x \in \mathbb{R}: a < x < b }$.
	\item[{$\left[ a, b), (a, b\right] $}] "--- \textbf{полуинтервалы} между числами $a$ и~$b$: $[a,b) = \{x \in \mathbb{R}: a \leq x < b\}$, $(a,b] = \{x \in \mathbb{R}: a < x \leq b\}$.
	\item[$[a, +\infty)$] "--- луч: $[a, +\infty)] = \{x \in \mathbb{R}: a \leq x\}$.
	\item[($a, +\infty$)] "--- луч: $(a, +\infty)] = \{x \in \mathbb{R}: a < x\}$.
	\item[{$(-\infty, b]$}] "--- луч: $(- \infty, b] = \{x \in \mathbb{R}: x \leq b\}$.
	\item[$(-\infty, b)$] "--- луч: $(-\infty, b) = \{x \in \mathbb{R}: x < b\}$.
	\item[Промежуток] "--- \emph{отрезок}, \emph{интервал} либо \emph{полуинтервал}.Промежуток любого из четырех типов обозначается $\langle a, b \rangle$. В~рамках одного утверждения запись $\langle a, b \rangle$ всегда обозначает один и~тот же~подвид промежутка.
	\item[$\langle a, b \rangle$] "--- любой из~двух промежутков  $(a,b)$ и~$[a,b)$.
	\item[$\forall$] "--- квантор всеобщности, используется для~сокращённой записи вместо понятий <<каждый>>, <<любой>>, или~<<для~всякого>>, <<для любого>> и~т.\,п.
	\item[$\exists$] "--- квантор существования, используется для~сокращённой записи вместо слов <<найдётся>>, <<существует>> и~т.~п.
	\item[$\sum \limits_{k=n}^{n} a_k$] "--- сумма чисел $a_k$ по~$k$ от~$m$ до~$n$, т.\,е.~$a_m + a_{m+1}+a_{m+1}+\ldots+a_n$.
	\item[$f:X \textrightarrow Y$] "--- функция, заданная на~множестве $X$, множество значений которой лежит в~$Y$ (но~необязательно с~ним~совпадает).
	\item[label] description
\end{description}
\section{Последовательности}
\subsection{Понятие множества}
Под~\emph{множеством} понимают совокупность, класс или~собрание объектов безразлично какой природы. Согласно определению основоположника теории множеств \href{https://ru.wikipedia.org/wiki/Кантор,_Георг}{Г.\,Кантора}~\cite{Wiki:Kantor}, множество "--- это~собрание предметов одинаковых или~различных между собой, мыслимое как единое целое. Собрание предметов рассматривается как~один предмет. Не~следует понимать множество как~совокупность действительно существующих предметов, принадлежность предметов одному множества не~требует от~них~сосуществования во~времени и~пространстве. В~логике множество понимается как~абстрактный объект, в~котором каждый предмет рассматривается с~точки зрения признаков, по~которым данный предмет принадлежит данному множеству. В~множестве предметы становятся неразличимыми друг от~друга по~признакам и~их~только по~именам.

Объект, принадлежащий данному множеству, называется его~\textbf{элементом}. Множество обозначается заглавными латинскими буквами $А, В, С$. Элементы, входящие в~множество, обозначаются строчными латинскими буквами и~заключаются в~фигурные скобки: ${a,b,c}$.

Множество, содержащее конечное число элементов, называется \textbf{конечным}, а~бесконечное число элементов "--- \textbf{бесконечным}.

Два множества называются \textbf{равными}, если содержат одинаковые элементы $(А={2,4,8}=В={2,2,4,8})$.

Элементами множества могут быть другие множества $А={{2,3},{4,5}}$. При~этом $А={{2,3},{4,5}}\neq В= {2,3,4,5}$.

Часть множества называется подмножеством данного множества. Например, $A \subset B$.
EEnd.\cite{Studopedia:mnozhestvo}
\subsection{Понятие отображения}
Большую роль в~математике имеет установление связей между двумя множествами $X$ и~$Y$, связанное с~рассмотрением пар объектов, образованных из~элементов первого множества и~соответствующих им~элементов второго множества. Особое значение при~этом имеет \emph{отображение множеств}.

Пусть $X$ и~$Y$ "--- произвольные множества. Отображением множества $X$ на~множество $Y$ называется $\forall$ правило $f$, по~которому каждому элементу множества $X$ сопоставляется вполне определённый~(единственный) элемент множества~$Y$. Тот~факт, что~$f$ есть отображение $X$ в~$Y$, кратко записывают в~виде: $f:X->Y$.

\section{Функции и~непрерывность)}
\section{Производные}
\section{Интегралы}




\nocite{CSC:intro-in-matan}

\printbibliography[title=Источники информации]

\end{document}
