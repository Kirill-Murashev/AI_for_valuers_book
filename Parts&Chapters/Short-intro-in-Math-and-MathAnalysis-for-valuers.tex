\documentclass[]{scrartcl}
\input{standard_preamble.tex}

\title{Введение в~математику и~математический анализ~для~оценщиков}
\author{К.\,А.\,Мурашев}

\begin{document}

\maketitle

\begin{abstract}
	Какую~бы работу не~выполнял оценщик, во~всех случаях он~имеет дело с~информацией и~данными. Часто эти~данные представляют собой числа либо могут быть формализованы иным образом. В~любом случае требуется алгоритмическая обработка входных данных и~преобразование их~в~информацию, а~в~некоторых случаях "--- в~знания. Целью данного фрагмента является формирование общих представлений об~основных понятиях и~методах математического анализа, необходимых современному оценщику. Материал построен таким образом, при~котором существует возможность ссылаться на~него при~решении практически всех математических задач, возникающих у~оценщиков, начиная со~школьной программы 5-класса, заканчивая математическим анализом, в~объёме преподаваемом на~нематематических специальностях вузов. Специфические вопросы, касающиеся частотного подхода в~математической статистике, байесовского подхода, а~также математических методов, применяемых в~машинном обучении, а~также иных специфических методов, выходящих за~рамки программы нематематических специальностей, рассмотрены в~отдельных материалах.  Автор постарался прибегать к~минимальному числу формул и~сложных определений, хотя это~и~не~вполне получилось. Поскольку конечной целью всей работы является цифровизация оценочной деятельности, в~тексте приводятся короткие листинги на~языках R и~Python, позволяющие реализовать то, о~чём говорится в~тексте. 
\end{abstract}

\tableofcontents
\section{Некоторые особенности материала}
\subsection{Список обозначений}\label{mathan-gloss-symbols}
Все~обозначения, используемые в~материале, соответствуют общепринятым в~математике. Далее приводится краткая шпаргалка~\cite{CSC:intro-in-matan}.
\begin{description}
	\item[$\mathbb{N}$] "--- множество \textbf{натуральных чисел}, т.\,е.~таких чисел, которые получаются при~счёте объектов:~$1, 2, 3, 4, 5\ldots$. Наименьшее натуральное число "--- $1$. Наибольшего натурального числа не~существует. \textbf{Натуральный~ряд} "--- это последовательность всех натуральных чисел. В~натуральном ряду каждое число больше предыдущего на~1. Натуральный ряд бесконечен, наибольшего натурального числа в~нём~не~существует.
	\item[$\mathbb{Z}$] "--- множество \textbf{целых чисел}, включающее в~себя \emph{натуральные числа}, все~числа противоположные им~по~знаку, а~также число ноль.
	\item[$\mathbb{Q}$] "--- множество \textbf{рациональных чисел}, т.\,е.~дробей вида $\frac{m}{n}$, где~ $m \in \mathbb{Z}$ и~$n \in \mathbb{N}$.
	[\item[$\mathbb{I}$] "--- множество \textbf{иррациональных чисел}, т.\,е. , бесконечных непериодических дробей. Примерами являются $\sqrt{2}$, число $\pi \approx 3.15159$, число $e \approx 2.718281828459$ и~т.\,д.
	\item[$\mathbb{R}$] "--- множество \textbf{вещественных~(действительных) чисел}, содержащее в~себе все~\emph{рациональные} и~\emph{иррациональные} числа.
	\item[$\in$] "--- оператор принадлежности. Запись $x \in \mathbb{Z}$ означает <<x~принадлежит к~множеству \emph{целых чисел}>> либо <<x~является \emph{целым числом}>>.
	\item[$x\in X:a$] "--- означает подмножество множества $X$, состоящее из элементов, удовлетворяющих условию $a$.
	\item[${A\bigcup B}$] "--- объединение множеств $A$ и~$B$.
	\item[${A\bigcap B}$] "--- пересечение множеств $A$ и~$B$.
	\item[${A\subset B}$] "--- множество~$A$ является подмножеством множества~$B$.
	\item[$A \backslash B$] "--- разность множеств $A$ и~$B$.
	\item[$A \triangle B$] "--- симметричная разность множеств $A$ и~$B$.
	\item[$A'$] "--- Дополнение к~множеству $A$.
	\item[$\bigcup \limits_{k=1}^{n}A_k$] "--- объединение всех множеств $A_1, A_2,\ldots, n$.
	\item[$\bigcap \limits_{k=1}^{n}A_k$] "--- пересечение всех множеств $A_1, A_2,\ldots, n$.
	\item[$\varnothing$] "--- пустое множество.
	\item[\AE{}] "--- пустое множество.
	\item[$M_A$] "--- множество всех подмножеств множества $A$.
	\item[{$\left[ a,b \right]$}] "--- \textbf{отрезок} между числами $a$ и~$b$ т.\,е.~множество вещественных чисел, лежащих между числами a~и~b, включая сами числа a~и~b. На~математическом языке это~можно записать как~$[a, b] = {x \in \mathbb{R}: a \leq x \leq b }$. При~$a=b$ отрезок состоит из~одной точки и~называется \emph{вырожденным отрезком}.
	\item[$(a, b)$] "--- \textbf{интервал} между числами $a$ и~$b$ т.\,е.~множество вещественных чисел, лежащих строго между $a$~и~$b$, не~включая их~самих. На~математическом языке это~можно записать как~$(a, b) = {x \in \mathbb{R}: a < x < b }$.
	\item[{$\left[ a, b), (a, b\right] $}] "--- \textbf{полуинтервалы} между числами $a$ и~$b$: $[a,b) = \{x \in \mathbb{R}: a \leq x < b\}$, $(a,b] = \{x \in \mathbb{R}: a < x \leq b\}$.
	\item[$[a, +\infty)$] "--- луч: $[a, +\infty)] = \{x \in \mathbb{R}: x \geq a\}$.
	\item[($a, +\infty$)] "--- открытый луч: $(a, +\infty)] = \{x \in \mathbb{R}: x > a \}$.
	\item[{$(-\infty, b]$}] "--- луч: $(- \infty, b] = \{x \in \mathbb{R}: x \leq b\}$.
	\item[$(-\infty, b)$] "--- открытый луч: $(-\infty, b) = \{x \in \mathbb{R}: x < b\}$.
	\item[Промежуток] "--- \emph{отрезок}, \emph{интервал} либо \emph{полуинтервал}.Промежуток любого из четырех типов обозначается $\langle a, b \rangle$. В~рамках одного утверждения запись $\langle a, b \rangle$ всегда обозначает один и~тот же~подвид промежутка.
	\item[$\langle a, b \rangle$] "--- любой из~двух промежутков  $(a,b)$ и~$[a,b)$.
	\item[$\forall$] "--- квантор всеобщности, используется для~сокращённой записи вместо понятий <<каждый>>, <<любой>>, или~<<для~всякого>>, <<для любого>> и~т.\,п.
	\item[$\exists$] "--- квантор существования, используется для~сокращённой записи вместо слов <<найдётся>>, <<существует>> и~т.~п.
	\item[$\sum \limits_{k=n}^{n} a_k$] "--- сумма чисел $a_k$ по~$k$ от~$m$ до~$n$, т.\,е.~$a_m + a_{m+1}+a_{m+1}+\ldots+a_n$.
	\item[$f:X \textrightarrow Y$] "--- функция, заданная на~множестве $X$, множество значений которой лежит в~$Y$ (но~необязательно с~ним~совпадает).	
	\item[:] "--- в~формулах означает выражение <<при~условии>>, например $x^3>0:x>0$.
	\item[$\equiv$] "--- означает тождественность.
	\item[$\Rightarrow$] "--- знак импликации, следования. Означает <<влечёт>>, <<отсюда следует>>, <<следовательно>>.
	\item[$\Leftrightarrow$] "--- знак равносильности, означает <<если и только если>> либо <<равносильно>>.
	\item[$\wedge$] "--- логическое <<и>>~(конъюнкция).
	\item[$\vee$] "--- логическое <<или>>~(дизъюнкция).
	\item[$\neg$] "--- логическое <<нет>>~(отрицание).
	\item[$\stackrel{def}{=}$] "--- определение.
	\item[$\rho$] "--- расстояние между двумя точками.
	\item[$\vdots$] "--- означает делимость.
	\item[$\sim, \backsim$] "--- знаки пропорциональности.
	\item[$\smile$] "--- дуга.
	
	
\end{description}

\section{Основы арифметики и~алгебры}
\subsection{Виды чисел}
\begin{description}
	\item[Натуральными числами] называются такие числа, которые используются для~подсчёта количества объектов. Например, количество входов торгово-развлекательного комплекса выражается натуральным числом. Множество натуральных чисел обозначается символом~$\mathbb{N}$~(понятие множества рассмотрено в~\ref{multiple:definition}). Примерами \emph{натуральных чисел} являются:~$1, 2, 3, 4, 5\ldots$. Наименьшее натуральное число "--- $1$. Наибольшего натурального числа не~существует. \textbf{Натуральный~ряд} "--- это последовательность всех \emph{натуральных чисел}. В~натуральном ряду каждое число больше предыдущего на~1. \emph{Натуральный ряд} бесконечен, наибольшего натурального числа в~нём~не~существует. $0$~не~является \emph{натуральным числом}.
	\item[Целыми числами] являются все~\emph{натуральные числа}, все~числа противоположные им~по~знаку, а~также число ноль. Множество целых чисел обозначается символом~$\mathbb{Z}$.
	\item[Рациональными числами] являются дроби вида $\frac{m}{n}$, где~ $m \in \mathbb{Z}$ и~$n \in \mathbb{N}$. Множество \emph{рациональных чисел} обозначается символом $\mathbb{Q}$.
	\item[Иррациональными числами] называют бесконечные непериодические дроби, например $\sqrt{2}$, число $\pi \approx 3.15159$, число~$e \approx 2.718281828459$ и~т.\,д. Множество иррациональных чисел обозначается символом $\mathbb{I}$.
	\item[Вещественными~(действительными) числами] называют множество чисел включающее в~себя множества \emph{рациональных} и~\emph{иррациональных чисел}. Множество вещественных чисел обозначается символом~$\mathbb{R}$.
	\item[Комплексными числами) числами] называют расширение множества вещественных чисел. Такие числа могут быть записаны в~виде $z=x+iy$, где~$i$ "--- мнимая единица, для~которой выполняется равенство $i^2=-1$. Множество \emph{комплексных чисел} обозначается символом~$\mathbb{C}$.
\end{description} 
Помимо вышеперечисленных видов чисел также существуют \textbf{кватернионы}~($\mathbb{I}$), \textbf{октонионы}~($\mathbb{O}$), \textbf{седенионы}~($\mathbb{S}$), \textbf{адели} и~\textbf{идели}. Однако их~рассмотрение в~данном материале является избыточным. 

Общая иерархия чисел может быть записана выражением
\begin{equation}\label{eq:numbers-hierarchy}
\mathbb{N} \subset \mathbb{Z} \subset \mathbb{Q} \subset \mathbb{R} \subset \mathbb{C} \subset \mathbb{H} \subset \mathbb{O} \subset \mathbb{S}.
\end{equation}
На~естественном языке это~звучит как <<все~\emph{натуральные числа} являются \emph{целыми числам}и, но~не~все \emph{целые} "--- \emph{натуральными}, все~\emph{целые числе} являются \emph{рациональными}, но~не~все~\emph{рациональные} "--- \emph{целыми} и~т.\,д.>>. На~математическом языке это~звучит как~<<\emph{множество натуральных чисел} является \emph{подмножеством целых чисел}, \emph{множество целых} "--- \emph{подмножеством рациональных} и~т.\,д.>>. Данная иерархия показана графически на~рисунке~\ref{fig:numbers-types}. Как~правило, в~практике оценки стоимости работа осуществляется с~\emph{вещественными числами} и~их~подмножествами.

\begin{figure}[ht]
	\centering % Центрируем картинку
	\includegraphics[width=0.5\textwidth]{numbers-types.pdf}
	\caption{Иерархия типов чисел \cite{Wiki:numbers-types}}\label{fig:numbers-types}
\end{figure}

\subsection{Элементарные формулы, уравнения и~пропорции}
\subsubsection{Пропорции}

Две~величины \emph{прямо пропорциональны} друг другу, если изменение значения одной из~них в~$m$~раз влечёт за~собой такое~же изменение другой.
\begin{equation}\label{simple-dir-prop1}
	\begin{aligned}
		\frac{a}{b}=\frac{c}{x} \\
		xa = bc \\
		x = \frac{bc}{a}
	\end{aligned}
\end{equation} 

Две~величины \emph{обратно пропорциональны} друг другу, если увеличение~(уменьшение) значения одной из~них в~$m$~раз влечёт за~собой уменьшение~(увеличение) значения другой также в~$m$~раз.
\begin{equation}\label{simple-inv-prop1}
\begin{aligned}
\frac{a}{b}=\frac{c}{x} \\
xb = ac \\
x = \frac{ac}{b}
\end{aligned}
\end{equation}

\begin{Thexmpl}\label{ex:dir-prop}
	Для~отопления здания строительным объёмом 2000~куб.\,м необходима отопительная система мощностью 68~кВт. Какова потребная мощность отопительной системы для~здания строительным объёмом 2000~куб.\,м?
	
	\begin{equation*}\label{ex:-inv-prop}
	\begin{aligned}
	\frac{68}{2000}=\frac{x}{2500} \\
	2000x = 68\times 2500 \\
	2000x = 170000 \\
	x = \frac{170000}{2000} \\
	x = 85
	\end{aligned}
	\end{equation*}
	
	Ответ: для здания строительным объёмом 2500~куб.\,м необходима система мощностью 85\,кВт.
\end{Thexmpl}

\begin{Thexmpl}\label{ex:inv-prop}
Резец токарного станка утрачивает свои свойства и~нуждается в~обслуживании после 30~дней эксплуатации при~ежедневном односменном использовании (1~смена "--- 8 часов). Через сколько дней потребуется обслуживание резца при~трёхсменной работе?

\begin{equation*}\label{}
\begin{aligned}
\frac{1}{3}=\frac{30}{x} \\
3x = 30 \\
x = \frac{30}{3} \\
x = 10
\end{aligned}
\end{equation*}

Ответ: при~трёхсменной работе обслуживание резца потребуется через 10~дней.
\end{Thexmpl}

\subsubsection{Трансформация бесконечной периодической десятичной дроби в~обыкновенную}

В~ряде случаев возникает потребность трансформации бесконечной периодической десятичной дроби в~обыкновенную. Для~выполнения этой операции следует использовать формулу
\begin{equation}\label{eq:periodic-to-frac-1}
a.b(c)=a\frac{<b><c>-<b>}{x[9]y[0]},
\end{equation}
где $a$ "--- целая часть десятичной дроби,

$b$ "--- не повторяющая часть десятичной дроби,

$c$ "--- периодическая часть десятичной дроби,

$x$ "--- количество цифр 9 в~знаменателе, зависит от~количества чисел в~периодической части $c$,

$x$ "--- количество цифр 0 в~знаменателе, зависит от~количества чисел в~не повторяющейся части $b$.

В~случае отсутствия не~повторяющейся части используется формула
\begin{equation}\label{eq:periodic-to-frac-2}
a.(c)=a\frac{c}{x[9]}
\end{equation}

\begin{Thexmpl}\label{ex:periodic-to-frac-2}
	Вычислим
	$\begin{aligned}
		2.12(3) = 2\frac{123-12}{900} = 2\frac{111}{900} = 2\frac{37}{300} \\
		2.(3) = 2\frac{3}{9} =2\frac{1}{3} 
	\end{aligned}$
	
\end{Thexmpl}

\subsubsection{Работа с~многоэтажными дробями}

С~учётом повсеместного распространения компьютерных вычислений, нет~никакой сложности вычисления многоэтажных дробей. Однако при~работе с~аналитическими методами часто возникает необходимость приведения выражения к~табличному~(стандартному кем-то~уже исследованному) виду. Например, такая потребность возникает при~вычислении производных, дифференциалов и~интегралов, многие из~которых имеют стандартные решения. Умение видеть в~существующем выражении другое, имеющее стандартное решение, и~преобразовать первое ко~второму позволяет экономить много времени. Таким образом, хотя типичный оценщик возможно никогда не~будет вычислять дифференциалы, оценщику, занимающемуся разработкой экспертных систем, подобное знание не~будет лишним.
В~общем виде работа с~многоэтажными дробями выглядит следующим образом
\begin{equation}\label{eq:multilevel-frac}
\frac{\frac{a}{b}}{\frac{c}{d}} = \frac{a}{b} \div \frac{c}{d} = \frac{a}{b} \times \frac{d}{c}
\end{equation}

\subsubsection{Свойства числовых неравенств}
\textbf{Первым свойством неравенств} является сохранение знака неравенства при~сложении его~обеих частей с~константой.
\begin{equation}\label{eq:enequalities-property1}
	\begin{aligned}
	a&>b \\
	a+k&>b+k:\ \forall k\\
	\end{aligned}
\end{equation}

\textbf{Вторым свойством неравенств} является сохранение знака неравенства при~умножении либо делении его~обеих частей на~положительную константу.
\begin{equation}\label{eq:enequalities-property2}
\begin{aligned}
a&>b \\
ak&>bk:\ k>0\\
\end{aligned}
\end{equation}

\textbf{Третьим свойством неравенств} является изменение знака неравенства на~противоположный при~умножении либо делении его~обеих частей на~отрицательную константу.
\begin{equation}\label{eq:enequalities-property3}
\begin{aligned}
a&>b \\
ak&<bk:\ k<0\\
\end{aligned}
\end{equation}

\textbf{Четвёртым свойством неравенств} является возможность почленного сложения неравенств, имеющих одинаковый знак. При~этом знак неравенств сохраняется.
\begin{equation}\label{eq:enequalities-property4}
\begin{aligned}
a>b \\
c>d \\
a+c>b+d\\
\end{aligned}
\end{equation}

\textbf{Пятым свойством неравенств} является возможность почленного вычитания неравенств, имеющих одинаковый знак. При~этом сохраняется знак первого неравенства.
\begin{equation}\label{eq:enequalities-property5}
\begin{aligned}
a>b \\
c<d \\
a-c>b-d\\
\end{aligned}
\end{equation}

\textbf{Шестым свойством неравенств} является возможность их~почленного умножения при~одинаковом знаке с~его~сохранением в~том~случае, когда все~члены неравенств являются положительными числами.
\begin{equation}\label{eq:enequalities-property6}
\begin{aligned}
a&>b \\
c&>d \\
ac&>bd:\ a,b,c,d>0
\end{aligned}
\end{equation}
\textbf{Седьмое свойство неравенств} заключается в~сохранении знака неравенства при~сложении его~членов с~одной и~той~же константой.
\begin{equation}\label{eq:enequalities-property7}
\begin{aligned}
a&>b \\
b&>c \Rightarrow \\
a&>c\\
\end{aligned}
\end{equation}

\textbf{Восьмое свойство неравенств} заключается в~сохранении знака неравенства при~возведении его~членов в~степень с~одинаковым показателем, при~условии положительного значения членов.
\begin{equation}\label{eq:enequalities-property8}
\begin{aligned}
a&>b \Rightarrow\\
a^n&>b^n:\ a,b>0\\
\end{aligned}
\end{equation}

\subsubsection{Системы уравнений и~неравенств с~двумя переменными}\label{two-unknown-system}

В~общем виде такие уравнения задаются выражением
\begin{equation}\label{eq:systems-linear-equations}
\begin{aligned}
\begin{cases}
a_{1}x+b_{1}y+c_{1}=0\\
a_{2}x+b_{2}y+c_{2}=0
\end{cases}
\end{aligned}
\end{equation}
Решением системы таких уравнений является нахождение \textit{x} и~\textit{y}, удовлетворяющих каждому из~условий. 
\begin{description}
	\item[Корень уравнения] "--- такое значение переменной, при~подстановке которого уравнение обращается в~верное числовое равенство. \textbf{Корень уравнения} с~одной переменной также называют решением уравнения.
\end{description}

Существует несколько методов решения уравнений с~двумя переменными. В~данном материале будут рассмотрены \emph{метод подстановки}, \emph{метод сложения}, \emph{метод замены} и~\emph{метод деления}. В~первом случае одна из~неизвестных выражается через другую.
\begin{Thexmpl}\label{ex:sys1}
	Дано:
	
	$\begin{aligned}
		\begin{cases}
			2x-y=2 \\
			4x+3y = 9\\
		\end{cases}\\
		\text{Выразим y через x из первого уравнения}\\
		y=2x-2 \Rightarrow 4x+3(2x-2) = 9 \\
		4x+6x-6 = 9 \\
		10x = 15 \\
		x = 1.5 \Rightarrow 3-y=2 \\
		y = 1 \\
		\text{Ответ: x = 1.5, y = 1} \\
		\end{aligned}$
	\end{Thexmpl}
Во~случае применения \emph{метода подстановки} на~первом этапе одно из~уравнений умножается на~константу так, чтобы при~почленном сложении на~втором этапе одно из~неизвестных уничтожилось. Умножение обоих уравнений на~константы также допускается. При~этом константой может быть любое положительное число.
\begin{Thexmpl}\label{ex:sys2}
	Дано:
	
	$\begin{aligned}
	\begin{cases}
	2x-y=2 \\
	4x+3y=9\\
	\end{cases}\\
	\begin{cases}
	2x-y=2 | \times3 \\
	4x+3y=9\\
	\end{cases}
	\begin{cases}
	6x-3y=6\\
	+\\
	4x+3y=9\\
	\end{cases}
	10x=15\\
	x=1.5\\
	3y=9-6\\
	y=1\\
	\text{Ответ: x = 1.5, y = 1} \\
	\end{aligned}$
\end{Thexmpl}

В~случае применения \emph{метода замены} какая-либо неудобная переменная заменяется на~другую.

\begin{Thexmpl}\label{ex:sys3}
	Дано:
	
	${\displaystyle
	\begin{aligned}
	\begin{cases}
	\dfrac{2}{x-3y}+\dfrac{3}{2x+y}=2\\
	\dfrac{8}{x-3y}-\dfrac{9}{2x+y}=1.\\
	\end{cases}\\
	\text{Сделаем замены:}
	\begin{cases}
	\dfrac{2}{x-3y}=a\\
	\dfrac{3}{2x+y}=b.
	\end{cases}	
	\text{Тогда:}
	\begin{cases}
	a+b=2|\times3\\
	4a-3b=1
	\end{cases}\\
	\begin{cases}
	3a+3b=3\\
	+\\
	4a-3b=1
	\end{cases}
	\Rightarrow
	\begin{cases}
	a=1\\
	b=1.
	\end{cases}
	\Rightarrow
	\begin{cases}
	\dfrac{2}{x-3y}=1\\
	\dfrac{3}{2x+y}=1
	\end{cases}
	\Rightarrow
	\begin{cases}
	x-3y=2\\
	2x+y=3
	\end{cases}
	\Rightarrow
	\begin{cases}
	x-3y=2\\
	+\\
	2x+y=3|\times 3
	\end{cases}
	\Rightarrow\\
	\Rightarrow
	\begin{cases}
	x=\dfrac{11}{7}\\
	y=-\dfrac{1}{7}
	\end{cases}
	\end{aligned}}$
\end{Thexmpl}
В~случае применения \emph{метода деления} одно из~уравнений делится на~другое в~результате чего возникает новое уравнение с~одним неизвестным.

\begin{Thexmpl}\label{ex:sys4}
	Дано:
	
	${\displaystyle
		\begin{aligned}
		\begin{cases}
		x^3+xy^2=5\\
		y^3+x^{2}y=10
		\end{cases}\\
		\begin{cases}
		x(x^2+y^2)=5\\
		y(x^2+y^2)=10
		\end{cases}
		\text{разделим второе уравнение на~первое}
		\dfrac{y}{x}=2
		\Rightarrow
		y=2x\\
		\text{подставим значение y в~первое уравнение}
		x^3+x4x^2=5
		\Rightarrow
		5x^3=5
		x=1\\
		\begin{cases}
		x=1\\
		y=\pm2
		\end{cases}  
		\end{aligned}}$
\end{Thexmpl}

\subsubsection{Уравнения с~двумя неизвестными}\label{Two-unknown-1}
\begin{description}
	\item[Уравнения с~двумя неизвестными] "--- уравнения вида
	\begin{equation}\label{eq:two-unknown-1}
	f(x,y)=0
	\end{equation}
\end{description}
\paragraph{Уравнения с~двумя неизвестными, имеющие бесконечное число решений}
Частный случай таких уравнений может быть записан следующим образом.
\begin{equation}\label{eq:two-unknown-2}
ax+by+c=0:\ a \vee b \neq 0
\end{equation}
Такие уравнения имеют бесконечное число решений, которые могут быть представлены графически.

\begin{Thexmpl}\label{ex:two-unknown-1}
	Дано:
	
	$\begin{aligned}
	4x-2y+2=0
	\end{aligned}$
	
	\text{Методом подстановки можно догадаться, что~решением является} x=1, y=3.
	
	\text{Однако решениями будут и~} x=0, y=1, x=-3, y=-2. \text{и~т.\,д.}
		
    \text{Графическое решение показано на~рисунке~\ref{fig:two-unknown}.} 
\end{Thexmpl}

\begin{figure}[ht]
	\centering % Центрируем картинку
	\includegraphics[width=0.5\textwidth]{two-unknown.pdf}
	\caption{Графическое решение уравнения с~двумя неизвестными и~окружность, задаваемая уравнением}\label{fig:two-unknown}
\end{figure}

\paragraph{Однородные уравнения}
\begin{description}
	\item[Уравнения с~двумя неизвестными] "--- уравнения вида
	\begin{equation}\label{eq:two-unknown-3}
	p(x,y)=0,
	\end{equation}
\end{description}
где ${\textstyle p(x,y)}$ "--- многочлен следующего вида:
\begin{equation}\label{eq:two-unknown-4}
p(x,y)=a_{n}x^{n}+a_{n-1}x^{n-1}y+\ldots+a_{1}xy^{n-1}+a_{0}y^{n}.
\end{equation}
Таким образом, показатель степени ${\textstyle x}$ повышается, ${\textstyle y}$ "--- понижается.

\begin{Thexmpl}\label{ex:two-unknown-2}
	${\textstyle x^3+4xy^2-5y^3=0}$
	
	Разделим всё~выражение на~${\textstyle y^3}$
	
	${\textstyle x^3+4xy^2-5y^3=0|:y^3\neq 0 \Leftrightarrow  (\dfrac{x}{y})^3+4(\dfrac{x}{y})-5=0}$
	
	Заменим ${\textstyle \dfrac{x}{y}}$ на~${\textstyle t}$.
	
	Тогда ${\textstyle t^3+4t-5=0 \Leftrightarrow t=1 \Rightarrow x=y}$. 	
\end{Thexmpl}



\paragraph{Практическое применение уравнений с~двумя неизвестными}\label{two-unknown-practice}
Графиком уравнения ${\textstyle p(x,y)=0}$ является множество точек на~координатной плоскости, удовлетворяющих данному уравнению.

Расстояние между двумя точками ${\textstyle A(x_{1},y_{1}),\ B(x_{0},y_{0})}$ на~координатной плоскости, то~есть длина отрезка ${\textstyle AB}$ вычисляется по~формуле:
\begin{equation}\label{eq:dist-between-points}
|AB|=\sqrt{(x_{1}-x_{0})^2+(y_{1}-y_{0})^2}.
\end{equation}
Из~данной формулы следует, например уравнение окружности с~центром ${\textstyle O(x_{0},y_{0})}$ и~радиусом~${\textstyle R}$.
\begin{equation}\label{eq:circle-equation}
(x_{1}-x_{0})^{2}+(y_{1}-y_{0})^2=R^2
\end{equation}
Примеры окружностей, заданных уравнением, показаны на~\ref{fig:two-unknown}.

\subsubsection{Операции со~степенями с~натуральными и~нулевым показателями}

Свойствами степени с~натуральным показателем являются:
\begin{equation}\label{eq:nat-degrees-prop-1}
	\begin{aligned}
	a^n \times a^m &= a^{n+m} \\
	\frac{a^n}{a^m} &=a^{n-m} \\
	a^{n^{m}} &= a^{nm} \\
	a^n \times b^n &= (ab)^n \\
	\frac{a^n}{b^n} &= (\frac{a}{b})^n \\
	a^0 &= 1: a \neq 0 \\
	0^0 &\not\exists.
	\end{aligned}
\end{equation}




\subsubsection{Понятие одночлена и~операции с~ним}
\begin{description}
	\item[Одночлен] "--- произведение переменных либо чисел, возведённое в~степень с~натуральным показателем.
\end{description}
Для~работы с~одночленами, как~правило, их~приводят в~стандартный вид, при~котором числовая часть выносится вперёд.
\begin{Thexmpl}\label{ex:monomial-1}
	Дано:
	
	$4x^{3}y^{3}z^{2} \times -2x^{2}y^{2}z^{-1}$
	
	Необходимо привести данный одночлен к~стандартному виду.
	
	Ответ: $-8x^{5}y^{5}z$
\end{Thexmpl}

\begin{description}
	\item[Подобные одночлены] "--- одночлены, состоящие из~одних и~тех~же переменных, возведённых в~степень с~одним и~тем~же показателем.
\end{description}

\subsubsection{Понятие многочлена и~операции с~ним}
\begin{description}
	\item[Многочлен] "--- сумма одночленов.
\end{description}
Для~осуществления операций c~многочленами, как~правило, необходимо привести каждый из~входящих в~него одночленов в~стандартный вид, а~затем привести подобные слагаемые.

Существует ряд стандартных формул для~упрощённого умножения многочленов.

Квадрат суммы двух выражений:
\begin{equation}\label{eq:simple-multipl-of-polynomials-1}
(a+b)^2=a^2+2ab+b^2.
\end{equation}
Квадрат разности двух выражений:
\begin{equation}\label{eq:simple-multipl-of-polynomials-2}
(a-b)^2=a^2-2ab+b^2.
\end{equation}
Разность квадратов двух выражений:
\begin{equation}\label{eq:simple-multipl-of-polynomials-3}
a^2-b^2 = (a-b)(a+b).
\end{equation}
Куб суммы двух выражений:
\begin{equation}\label{eq:simple-multipl-of-polynomials-4}
(a+b)^3 = a^3 + 3a^{2}b+3ab^{2}+b^3.
\end{equation}
Куб разности двух выражений:
\begin{equation}\label{eq:simple-multipl-of-polynomials-5}
(a-b)^3 = a^3 - 3a^{2}b+3ab^{2}-b^3.
\end{equation}
Сумма кубов двух выражений:
\begin{equation}\label{eq:simple-multipl-of-polynomials-6}
a^3+b^3=(a+b)(a^2-ab+b^2),
\end{equation}
где $(a^2-ab+b^2)$ "--- неполный квадрат разности выражений.
Разность кубов двух выражений:
\begin{equation}\label{eq:simple-multipl-of-polynomials-7}
a^3-b^3=(a-b)(a^2+ab+b^2),
\end{equation}
где $(a^2-ab+b^2)$ "--- неполный квадрат суммы выражений.

Применение стандартных формул упрощает работу с~аналитическими выражениями.

\begin{Thexmpl}\label{ex:full-square}
	Земельный участок какой максимальной площади можно огородить, имея ограждение общей длиной 240\,м? Какова длина сторон такого участка, если участок имеет прямоугольную форму? 
	
	Примем длину одной стороны за $x$, тогда другая сторона равна $120-x$.
	
	$S=(120-x)(x)=120x-x^2=-(x^2-30x)$
	
	Выделим полный квадрат выражения.
	
	$-(\underbrace{x^2-2x60+60^2}_{\text{полный квадрат}}-60^2)=3600-(x-60)^2$
	
	Проанализируем полученное выражение. Логично, что~площадь будет максимальной~(3600~кв.\,м) в~том случае, если выражение в~скобке будет равно нулю, что~достигается при~$x=60$. Таким образом, максимально возможная площадь составляет 3600~кв.\,м при~всех сторонах равных 60~м, т.\,е.~тогда, когда участок будет иметь форму квадрата, что~соответствует априорным знаниям.
\end{Thexmpl}

\begin{description}
	\item[Разложение многочлена на~множители] "--- представление многочлена в~виде произведения других многочленов.
\end{description}

\subsubsection{Тождество}
\begin{description}
	\item[Тождество] "--- равенство, являющееся верным при~всех допустимых значения переменных.
\end{description}

Примерами тождеств являются формулы сокращённого умножения~(\ref{eq:simple-multipl-of-polynomials-1}--\ref{eq:simple-multipl-of-polynomials-7}).

\subsubsection{Алгебраические дроби}
\begin{description}
	\item[Алгебраической дробью] называется выражение вида
	\begin{equation}\label{eq:alg-frac}
	\frac{P}{Q}:\ Q \neq 0.
	\end{equation} 
\end{description}
Примерами алгебраических дробей являются, например выражения $\frac{x^2+y}{x}, \frac{x+2}{x-2}, \frac{5}{11}$ и~т.\,д.

\textbf{Основным свойством} \emph{алгебраических дробей} является то, что~при~умножении их~числителя и~знаменателя на~один и~тот~же многочлен, значение алгебраической дроби не~меняется при~условии ненулевого значения такого многочлена.

Из~этого свойства следует возможность сокращения числителя и~знаменателя на~общий множитель.

Для~сложения и~вычитания алгебраических дробей их~следует привести к~общему знаменателю по~обычным правилам.

\begin{Thexmpl}\label{ex:alg-frac-1}
	
	$\begin{aligned}
		\frac{x}{x+y}+\frac{y}{x-y}=\frac{x(x-y)}{(x+y)(x-y)}+\frac{y(x+y)}{(x+y)(x-y)}=\frac{x(x-y)+y(x+y)}{(x+y)(x-y)}=\\
		=\frac{x^{2}-xy+xy+y^{2}}{x^{2}-y^{2}}=\frac{x^{2}+y^{2}}{x^{2}-y^{2}}
	\end{aligned}$
\end{Thexmpl}
Умножение и~деление алгебраических дробей осуществляется согласно общим правилам осуществления таких операций с~дробями.
\begin{equation}\label{eq:alg-frac-multiple}
\frac{a}{b}\times\frac{c}{d}=\frac{ac}{bd}
\end{equation}
\begin{equation}\label{eq:alg-frac-division}
\frac{a}{b}\div\frac{c}{d}=\frac{ad}{bc}
\end{equation}
Возведение алгебраической дроби в~степень осуществляется отдельно для~числителя и~знаменателя.
\begin{equation}\label{eq:alg-frac-power}
(\frac{a}{b})^n=\frac{a^n}{b^n}
\end{equation}

\subsubsection{Рациональные уравнения}
\begin{description}
	\item[Рациональные уравнения] "--- это~выражения вида
	\begin{equation}\label{eq:rational-equations}
	p(x)=0
	\end{equation}
\end{description}

\subsubsection{Степень с~целым отрицательным показателем}
Вычисление степени с~целым отрицательным показателем осуществляется по~формуле
\begin{equation}\label{eq:neg-power}
a^{-n}=\frac{1}{a^{n}}
\end{equation}

\subsubsection{Функция $y=\sqrt{x}$, её~свойства и~график}
По~свойству квадрата областью определения данной функции являются все~неотрицательные числа, т.\,е. Областью её~значений также являются все~неотрицательные числа.
\begin{equation}\label{eq:sqrt-function-domain}
y=\sqrt{x}
D(y)=[0,+\infty)
E(y)=[0,+\infty)
\end{equation}
Данная функция является возрастающей, т.\,е.~большему значению аргумента соответствует большее значение функции.
$\begin{aligned}
x_2>x_1\\
y_2>y_1\\
\end{aligned}$
График функции $\sqrt{x}$ показан на~рисунке~\ref{fig:sqrt-func}.
\begin{figure}[ht]
	\centering % Центрируем картинку
	\includegraphics[width=0.8\textwidth]{sqrt-func.pdf}
	\caption{Функция $\sqrt{x}$}\label{fig:sqrt-func}
\end{figure}

\textbf{Первое свойство} квадратного корня:
\begin{equation}\label{sqrt-prop-1}
\sqrt{ab}=\sqrt{a} \times \sqrt{b}:\ a,b \geq 0.
\end{equation}
Из~этого также следует что
\begin{equation}\label{sqrt-prop-2}
\sqrt{\frac{a}{b}}=\frac{\sqrt{a}}{\sqrt{b}}:\ a,b \geq 0.
\end{equation}

\subsubsection{Модуль действительного числа}
Общее выражения для~модуля числа \textit{x}: 
\begin{equation}\label{eq:modul}
|x|=\begin{aligned} \begin{cases}
x:\ x &\geq 0\\
-x:\ x &\leq 0.\\
\end{cases}
\end{aligned}
\end{equation}
Свойствами модуля являются:
\begin{equation}\label{eq:module-properties}
\begin{aligned}
|a| &\geq 0\\
|ab| &= |a|\times|b|\\
|\frac{a}{b}| &= \frac{|a|}{|b|}\\
|a|&=|-a|\\
|a|^2&=a^2\\
|a+b| &\leq |a|+|b|\\
|a| &\geq a\\
\end{aligned}
\end{equation}

Пример графика функции, содержащей модуль, приведён на~рисунке~\ref{fig:modul-graph}.

\begin{figure}[ht]
	\centering % Центрируем картинку
	\includegraphics[width=0.8\textwidth]{modul-graph.pdf}
	\caption{График функции, содержащей модуль, и~аналогичной функции без модуля.}\label{fig:modul-graph}
\end{figure}

\subsubsection{Квадратные уравнения}
\paragraph{Основные понятия}

Квадратное уравнение "--- это уравнение следующего вида:
\begin{equation}\label{eq:suare-eq}
ax^2+bx+c=0:\ a \neq 0,
\end{equation}
где x "--- неизвестное,

\textit{a} "--- старший коэффициент,

\textit {b} "--- средний коэффициент,

\textit{c} "--- свободный член.

Уравнение, в~которой $a=1$, называется \emph{приведённым}. Уравнения, в~которых \textit{b} либо \textit{c} равны нуля называются \emph{неполными квадратными уравнениями}.
Решением квадратного уравнения является нахождение такого значения~(значений) \textit{x}, при~котором~(которых) выполняется исходное равенство. Такие значения \textit{x} называются \emph{корнями} квадратного уравнения.

\paragraph{Формула корней квадратного уравнения}

Для~решения квадратного уравнения чаще всего используют \emph{формулу дискриминанта}:
\begin{equation}\label{Discriminant-1}
D=b^2-4ac
\end{equation}
В~зависимости от~значения дискриминанта возможны следующие варианты:
\begin{equation}\label{Discriminant-2}
\begin{aligned}
D&>0 \Rightarrow x=\frac{-b \pm \sqrt{D}}{2a}\  \text{"--- 2 вещественных корня}\\
D&=0 \Rightarrow x=-\frac{b}{2a}\  \text{"--- 1 вещественный корень}\\
D&<0 \Rightarrow\  \text{"--- нет вещественных корней}.\\
\end{aligned}
\end{equation}
В~последнем случае можно говорить о~том, что~существуют два комплексных корня. Также выражение можно переписать, выразив корень из~отрицательного числа в~виде произведения корня с мнимой единицей.
\begin{equation}\label{Discriminant-3}
x=\frac{-b \pm i\sqrt{|D|}}{2a}
\end{equation}

Однако в~контексте оценки стоимости можно говорить о~том, что~уравнения с~$D<0$ не~имеют корней.

\paragraph{Теорема Вийета}
В~случае \emph{приведённого квадратного} уравнения, т.\,е.~такого, в~котором \emph{старший коэффициен}т равен единице, решение может быть осуществлено по~упрощённой формуле без~вычисления дискриминанта. 
\begin{theorem}
	Сумма корней приведённого квадратного уравнения равна среднему коэффициенту, взятому с~противоположным знаком, а~их~произведение "--- свободному члену.
\end{theorem}
\begin{equation}\label{Wijet}
\begin{aligned}
x^2+px+q&=0\\
x_1+x2&=-p\\
x_1 \times x_2 &= q\\
\end{aligned}
\end{equation}

\paragraph{Разложение квадратного трехчлена на~линейные множители}
Разложение квадратного трехчлена на~линейные множители осуществляется по~формуле:
\begin{equation}\label{eq:square-polynom-decomposition}
ax^2+bx+c=a(x-x_1)(x-x_2)
\end{equation}

\subsubsection{Делимость чисел}
Под~делимостью в~данной секции подразумевается делимость без~остатка. Рассмотрим два натуральных числа \textit{a} и~\textit{b}. В~случае существования такого натурального числа \textit{q}, умножение на~которого \textit{b} даёт \textit{a}, можно говорить о~делимости \textit{a} на~\textit{b}.
\begin{equation}\label{eq:delimostq}
a \vdots b: a=bq,\quad a,\ b,\ q \in \mathbb{N}
\end{equation}
Свойства делимости:
\begin{equation}\label{eq:delimostq-prop}
\begin{aligned}
a \vdots b,\ b \vdots c &\Rightarrow a \vdots c\\
a \vdots b,\ c \vdots b &\Rightarrow a+c \vdots b\\
a \vdots b,\ c \neg \vdots b &\Rightarrow a+c \neg \vdots b\\
a \vdots b &\Leftrightarrow ac \vdots\ bc\\
a \vdots b &\Rightarrow ac \vdots b\\
\text{среди n последовательных чисел одно и~только одно делится на~n}\\
\end{aligned}
\end{equation}
\emph{Простыми числами} называются числа, имеющие только два делителя "--- единицу и~самих себя. Числа имеющие более двух делителей называются \emph{составными}. Единица не~является ни~простым, ни~составным числом.

\begin{description}
	\item[Наименьшим общим кратным n~чисел~(НОК)] является наименьшее число, которое делится без~остатка на~любое из~них.
\end{description}

\begin{description}
	\item[Наибольшим общим делителем n~чисел~(НОД)] является наибольшее число, на~которое любое из~них делится без~остатка.
\end{description}
\begin{equation}\label{eq:Nod-Nok}
HOK(a,b) \times HOD (a,b) = a \times b
\end{equation} 

\subsubsection{Основная теорема арифметики натуральных чисел}
\begin{theorem}
	Всякое число, большее  1,  может быть разложено в~произведение простых чисел, и~это~разложение единственно с~точностью до~порядка множителей.
\end{theorem}
Иными словами любое натуральное число кроме 1 либо является простым, либо может быть разложено на~простые множители единственным способом.

\subsubsection{Уравнения высших степеней}
Уравнениями высших степеней являются уравнения вида
\begin{equation}\label{eq:equations-high-power}
P(x)=0,
\end{equation}
где~\textit{P} многочлен в~степени больше 2.

Существует два метода решения таких уравнений:
\begin{itemize}
	\item метод разложения на~множители, при~котором уравнение сводится к~квадратному;
	\item метод замены, при~котором на~первом этапе члены уравнения заменяются на~многочлены второй степени, после чего осуществляется решение нового уравнения с~ними, на~втором этапе полученные значения подстановка значений в~новую систему уравнений.
\end{itemize}
\subsubsection{Уравнения c~модулями}
Существует два метода решения таких уравнений:
\begin{itemize}
	\item метод последовательного раскрытия модуля со~знаками плюс и~минус;
	\item метод вынесения части, не~содержащей модуль, в~другую часть уравнения.
\end{itemize}
\subsubsection{Иррациональные уравнения}
Иррациональными уравнениями называются уравнения, содержащие иррациональные значения корня в~знаменателе. Решение таких уравнений сводится к~домножению членов на~множители так, чтобы иррациональная часть переместилась в~числитель, а~знаменатели сократились. Дальнейшее решение уравнения осуществляется по~общим правилам.
\subsubsection{Задачи с~параметрами}
Задачей с~параметром является уравнение, в~котором часть коэффициентов заменены буквенным выражением.
В~общем виде такие задачи можно выразить как:
\begin{equation}\label{eq:tasks-with-pararmetres}
f(x,a) = 0.
\end{equation}
Особенностью данных задач является необходимость решения для~каждого значения параметра.
\begin{Thexmpl}\label{ex:tasks-with-parametres-1}
	$\begin{aligned}
		ax+4&=12\\
		x&=\frac{8}{a}:\ a \neq 0
		x&\in \AE{}:\ a=0
	\end{aligned}$
\end{Thexmpl} 

\begin{Thexmpl}\label{ex:tasks-with-parametres-2}
	$\begin{aligned}
		(a^2+8a-5)x&=a-2\\
		(a+5)(a-2)x&=a-2\\
		x &\in \mathbb{R}:\ a=2\\
		x &\in \AE:\ a=-5\\
		x&=\frac{a-2}{(a+5)(a-2)}:\ a\neq 2,\ a\neq-5\\
	\end{aligned}$
\end{Thexmpl}
\subsubsection{Линейные и~квадратные неравенства}
С~точки зрения механики вычислений решение таких неравенств принципиально не~отличается от~решения уравнений. Особенность неравенств является то, что~при~их~умножении на~положительное число знак неравенства не~меняется, на~отрицательное "--- меняется на~противоположный.
\subsubsection{Стандартный вид числа}
\begin{description}
	\item[Стандартный вид числа] "--- запись числа в~виде
	\begin{equation}\label{eq:stand-numer}
	a=a_{0}10^{n},
	\end{equation}
где $0 \geq a < 10$, n-порядок числа.
\end{description}
В~информатике вместо $10^n$ как~правило используют $E$.
\begin{Thexmpl}\label{ex:stand}
	$\begin{aligned}
	1703 &= 1.703 \times 10^3 &= 1.703E3\\
	398.098 &= 3.98098 \times 10^2 &=3.98098E2\\
	0.572 &=5.72 \times 10^-1 &=5.72E-1\\
	\end{aligned}$
\end{Thexmpl}

\subsubsection{Рациональные неравенства}
\begin{description}
	\item[Рациональные неравенства] "--- неравенства вида:
	\begin{equation}\label{eq:rational-inequal}
	f(x)
	\begin{cases}
	>\\
	<\\
	\geq\\
	\leq\\
	\end{cases}
	0.
	\end{equation}
\end{description}
Два~неравенства называются \emph{равносильными}
\begin{equation}\label{eq:rational-inequal-equiv}
f(x)>0 \Leftrightarrow g(x)>0,
\end{equation}
если они~имеют одинаковое решение, либо оба~не~имеют решения вовсе. Равносильное неравенство может быть получено путём умножения неравенства на~выражение. В~случае положительного значения результата выражения, знак неравенства сохраняется, при~отрицательном "--- меняется на~противоположный.

Общая схема решения таких неравенств заключается к~поиску их~корней путём приравнивания к~нулю каждого члена и~проверки соблюдения знака при~подстановке значений больше и~меньше каждого корня. При~этом важным свойством является чередование знаков вокруг корней, получаемых из~членов, стоящих в~любой нечётной степени, включая первую, и~их~повторение вокруг корней, образуемых членами, стоящими в~любой чётной степени.

\subsubsection{Системы и~совокупности неравенств}
\textbf{Системы неравенств} в~общем виде выглядят следующим образом:
\begin{equation}\label{eq:inequal-system}
	\begin{cases}
	f(x)>0\\
	g(x)>0\\
	\ldots\\
	z(x)>0,
	\end{cases}
\end{equation}
при~этом знаки неравенств могут быть любыми: ${\textstyle >,<,\geq,\leq}$ равно как~и~число неравенств. Для~того, чтобы решить \emph{систему неравенств}, необходимо найти все~значения~${\textstyle x}$, удовлетворяющие каждому из~них. Иначе говоря, решением системы неравенств является \emph{множество}, представляющее собой \emph{пересечение множеств}, получаемых при~решении каждого неравенства в~отдельности. Если хотя~бы одно неравенство в~системе не~имеет решения, тогда его~не~имеет вся~система. Если одно из~неравенств выполняется при~любом значении ${\textstyle x}$, решение оставшейся части неравенств также 
является решением всей системы.

\textbf{Совокупности неравенств} в~общем виде выглядят следующим образом:
\begin{equation}\label{eq:inequal-complex}
\begin{sqcases}
f(x)>0\\
g(x)>0\\
\ldots\\
z(x)>0,
\end{sqcases}
\end{equation}
при~этом аналогично с~\emph{системами неравенств} знаки неравенств, образующих совокупность, могут быть любыми: ${\textstyle >,<,\geq,\leq}$ равно как~и~число неравенств. Для~того, чтобы решить \emph{совокупность неравенств}, необходимо найти все~значения~${\textstyle x}$, удовлетворяющие хотя~бы одному из~них. Иначе говоря, решением системы неравенств является \emph{множество}, представляющее собой \emph{объединение множеств}, получаемых при~решении каждого неравенства в~отдельности.

\subsubsection{Неравенства с~модулями}
\textbf{Первым} видом неравенства с~модулем является неравенство вида:
\begin{equation}\label{eq:inequalities-with-modules-1}
|f(x)|<c:\ c>0.
\end{equation}
Тогда:
\begin{equation}\label{eq:inequalities-with-modules-2}
|f(x)|<c\Leftrightarrow
\begin{cases}
f(x)>-c\\
f(x)<c\\
\end{cases}
\Leftrightarrow
\begin{cases}
-c<f(x)<c.
\end{cases}
\end{equation}
\begin{Thexmpl}\label{ex:inequalities-with-modules-1}
	$\begin{aligned}
	|3x+2|<4 \Leftrightarrow -4<3x+2<4 = -6<3x<2 = -2<x<\dfrac{2}{3}
	\end{aligned}$
\end{Thexmpl}
\textbf{Вторым} видом неравенства с~модулем является неравенство вида:
\begin{equation}\label{eq:inequalities-with-modules-3}
|f(x)|>c:\ c>0.
\end{equation}
Тогда:
\begin{equation}\label{eq:inequalities-with-modules-4}
|f(x)|<c\Leftrightarrow
\begin{sqcases}
f(x)>c\\
f(x)<-c\\
\end{sqcases}
\end{equation}
\begin{Thexmpl}\label{ex:inequalities-with-modules-2}
	$\begin{aligned}
	|2x+5|>10 \Leftrightarrow
	\begin{sqcases}
	2x+5>10\\
	2x+5<-10\\
	\end{sqcases}
	\Rightarrow
	\begin{sqcases}
	2x>5\\
	2x<-15\\
	\end{sqcases}
	\Rightarrow
	\begin{sqcases}
	x>2.5\\
	x<-7.5\\
	\end{sqcases}
	\Rightarrow\\ \Rightarrow x \in (-\infty;-7.5)\cup(2.5;\infty)
	\end{aligned}$
\end{Thexmpl}

\textbf{Третьим} видом неравенства с~модулем является неравенство вида:
\begin{equation}\label{eq:inequalities-with-modules-5}
|f(x)|<g(x).
\end{equation}
Поскольку ${\textstyle g(x)}$ больше \emph{модуля} некоторого выражения, очевидно, что~${\textstyle g(x)>0}$. Тогда
\begin{equation}\label{eq:inequalities-with-modules-6}
\begin{split}
|f(x)|<g(x)\Leftrightarrow
\begin{cases}
g(x>0)\\
-g(x)<f(x)<g(x)
\end{cases} 
\Leftrightarrow
\begin{cases}
g(x>0)\\
f^2(x)<g^2(x)
\end{cases}
\Leftrightarrow
\begin{cases}
g(x>0)\\
f^2(x)-g^2(x)<0
\end{cases}
\Leftrightarrow\\
\Leftrightarrow
\begin{cases}
g(x>0)\\
(f(x)+g(x))(f(x)-g(x))<0
\end{cases}
\end{split}
\end{equation}

\textbf{Четвёртым} видом неравенства с~модулем является неравенство вида:
\begin{equation}\label{eq:inequalities-with-modules-7}
|f(x)|>g(x).
\end{equation}
В~этом случае:
\begin{equation}\label{eq:inequalities-with-modules-8}
|f(x)|>g(x) \Leftrightarrow
\begin{sqcases}
	\begin{cases}
	g(x)<0\\
	x\in D(f)
	\end{cases}
	\\
	\begin{cases}
	g(x)\geq 0\\
	f^2(x)>g^2(x)
	\end{cases}
\end{sqcases}
\vee
\begin{sqcases}
\begin{cases}
g(x)<0\\
x\in D(f)
\end{cases}
\\
\begin{cases}
g(x)\geq 0\\
\begin{sqcases}
f(x)>g(x)\\
f(x)<-g(x)
\end{sqcases}
\end{cases}
\end{sqcases}
\end{equation}

\subsubsection{Иррациональные неравенства}
\textbf{Первый} тип иррациональных неравенств
	\begin{equation}\label{eq:irrational-inequal-1}
	\sqrt{f(x)}<g(x)
	\end{equation}
Тогда
\begin{equation}\label{eq:irrational-inequal-2}
\sqrt{f(x)}<g(x) \Leftrightarrow
\begin{cases}
g(x)>0\\
f(x)\geq 0\\
f^2(x)<g^2(x).
\end{cases}
\end{equation}

\begin{Thexmpl}\label{ex:irrational-inequal-1}
	
	$\begin{aligned}
	\sqrt{x^2-x-12}<x
	\Leftrightarrow
	\begin{cases}
	x>0\\
	x^2-x-12\geq 0\\
	x^2-x-12<x^2
	\end{cases}
	\Leftrightarrow
	\begin{cases}
	x>0\\
	(x+3)(x-4)\geq 0\\
	x>-12
	\end{cases}
	\Leftrightarrow\\
	\Leftrightarrow
	x\geq 4
	\end{aligned}$
\end{Thexmpl}

\textbf{Второй} тип иррациональных неравенств
\begin{equation}\label{eq:irrational-inequal-3}
\sqrt{f(x)}>g(x)
\end{equation}
Тогда
\begin{equation}\label{eq:irrational-inequal-4}
\sqrt{f(x)}>g(x) \Leftrightarrow
\begin{sqcases}
\begin{cases}
g(x)<0\\
f(x)\geq 0
\end{cases}
\\
\begin{cases}
g(x)\geq 0\\
f(x)>g^2(x).
\end{cases}
\end{sqcases}
\end{equation}
\begin{Thexmpl}\label{ex:irrational-inequal-2}
	$\begin{aligned}
	\begin{sqcases}
	\begin{cases}
	x<0\\
	x^2-x-12\geq 0
	\end{cases}
	\\
	\begin{cases}
	x\geq 0\\
	x^2-x-12>x^2
	\end{cases}
	\end{sqcases}
	\Leftrightarrow
	\begin{sqcases}
	\begin{cases}
	x<0\\
	(x+3)(x-4)\geq 0
	\end{cases}
	\\
	\begin{cases}
	x\geq 0\\
	x<-12
	\end{cases}
	\end{sqcases}
	\Leftrightarrow x\in (-\infty;-3].
	\end{aligned}$
\end{Thexmpl}

\subsubsection{Неравенства с~двумя переменными}
\begin{description}
	\item[Неравенства с~двумя переменными] "--- это~неравенства вида
	\begin{equation}\label{eq:inequal-two-unknown-1}
	p(x,y)
	\begin{sqcases}
	>\\
	\geq\\
	\leq\\
	<
	\end{sqcases}
	0
	\end{equation}
\end{description}
Как~правило, для~решения таких неравенств используется графический метод. Рассмотрим несколько примеров.
\begin{Thexmpl}
	${\textstyle (x+7)^{2}+(y+5)^{2}\leq 4}$
	
	В~\ref{two-unknown-practice} было показано, что~выражения такого вида задают окружность. Решением данного неравенства является множество точек, представляющих собой часть плоскости внутри окружности, включая саму окружность. В~случае строгости неравенства сама окружности не~входила~бы в~эту~часть плоскости. В~случае противоположного знака решением~бы было множество точек за~пределами окружности, включая либо не~включая её~саму, в~зависимости от~строгости неравенства. Графическое решение данного неравенства приведено на~рисунке~\ref{fig:ineq-two-unknown}.
\end{Thexmpl}

\begin{Thexmpl}
	Разберём другое неравенство. ${\textstyle xy<1}$. Рассмотрим три~случая:
	
	${\displaystyle 
	\begin{aligned}
	1.\ x=0 \Rightarrow y \in \mathbb{R}\\
	2.\ x>0 \Rightarrow y<\dfrac{1}{x}\\
	3.\ x>0 \Rightarrow y>\dfrac{1}{x}\\ 
	\end{aligned}}
	$
	
	Графическим решением, будет часть плоскости между соответствующими гиперболами, не~включая их~самих в~силу строгости неравенства, см.~рисунок~\ref{fig:ineq-two-unknown}. 
\end{Thexmpl}

\begin{figure}[ht]
	\centering % Центрируем картинку
	\includegraphics[width=0.8\textwidth]{ineq-two-unknown.pdf}
	\caption{Графическое решение неравенств с~двумя переменными}\label{fig:ineq-two-unknown}
\end{figure} 

\subsubsection{Однородные и~симметрические системы уравнений}
\begin{description}
	\item[Однородная система уравнений] "---  Система уравнений вида
	\begin{equation}\label{eq:homogeneous-equations}
	\begin{cases}
	p(x,y)=a\\
	q(x,y)=b,
	\end{cases}
	\end{equation}
	если ${\textstyle p,\ q}$ "--- \emph{однородные многочлены}, ${\textstyle a,\ b}$ "--- некоторые числа.
\end{description}

\begin{description}
	\item[Однородный многочлен] "---  многочлен, все~одночлены которого имеют одинаковую сумму степеней. Любая алгебраическая форма является однородным многочленом. Квадратичная форма задается однородным многочленом второй степени, бинарная форма "--- однородным многочленом любой степени от~двух переменных.
\end{description}
Примеры:

$\begin{aligned}
x^2+y^2&\text{\ "--- однородный многочлен}\\
x^3+2x{y}^2&\text{\ "--- однородный многочлен}\\
x^4+qzyx&\text{\ "--- однородный многочлен}\\
x+yz&\text{\ "--- неоднородный многочлен}
\end{aligned}
$

Решение таких уравнений осуществляется путём преобразования правой части к~нулю, после чего возможно применение обычных приёмов, описанных в~\ref{two-unknown-system}.

\begin{description}
	\item[Симметрическая система уравнений] "---  система уравнений вида
	\begin{equation}\label{eq:symmetrical-equations}
	p(x,y)=p(y,x),\ \text{т.\,е.}
	\begin{cases}
	p(x,y)=0\\
	q(x,y)=0\\	
	\end{cases}
	\begin{cases}
	u=x+y\\
	v=xy	
	\end{cases}
	\end{equation}
\end{description}

\begin{Thexmpl}
	Дано:
	
	$\begin{aligned}
	\begin{cases}
	x+y&=5\\
	x^2+y^2&=13
	\end{cases}
	\text{Введём}\ u=x+y,\ v=xy.\ 
	\text{Преобразуем}\ x^2+y^2=(x+y)^2-2xy\\
	\text{Тогда}
	\begin{cases}
	u=5\\
	u^2-2v=13
	\end{cases}
	\Rightarrow
	\begin{cases}
	u=5\\
	v=6
	\end{cases}
	\Rightarrow
	\begin{cases}
	x+y=5\\
	xy=6
	\end{cases}
	\Rightarrow
	\begin{sqcases}
	\begin{cases}
	x=2\\
	y=3
	\end{cases}\\
	\begin{cases}
	x=3\\
	2y=6
	\end{cases}
	\end{sqcases}
	\end{aligned}$ 
\end{Thexmpl}

\subsubsection{Иррациональные системы уравнений, системы уравнений с~модулями}
Для~решения подобных систем возможно использование всех ранее рассмотренных методов, однако существует ряд нюансов: в~случае \emph{иррациональных уравнений} какие-либо их~члены находятся под~знаком корня, что~означает необходимость принятия во~внимание \emph{области допустимых значений} (выражения, находящиеся под~корнем чётной степени не~могут иметь отрицательное значение), в~случае уравнений с~модулями, содержащиеся в~них~выражения могут раскрываться с~любым знаком.

\subsection{Основы геометрии}
\subsubsection{Основные геометрические объекты}\label{main-figures}
Базовые геометрические объекты не~имеют строгих определений и~определяются через свои свойства, изложенные в~аксиомах. Следующие ниже описания не~являются строгими и~служат для~понимания основных свойств таких объектов.
\begin{description}
	\item[Точка] "--- неделимый элемент соответствующего математического пространства.
\end{description}
\begin{description}
	\item[Прямая] "--- длина без~ширины, равно лежащая на~всех своих точках.
\end{description}
\begin{description}
	\item[Плоскость] "--- поверхность, содержащая полностью каждую прямую, соединяющую любые её~точки.
\end{description}
Следующие объекты уже~имеют формализованные описания.
\begin{description}
	\item[Отрезок прямой] "--- часть прямой, ограниченная двумя точками. Точнее: это множество, состоящее из~двух различных точек данной прямой (которые называются \emph{концами отрезка}) и~всех точек, лежащих между ними (которые называются его~внутренними точками).
\end{description}
Отрезок, концами которого являются точки  ${\displaystyle A} {\displaystyle B}$, обозначается символом ${\displaystyle AB}$. Расстояние между концами отрезка называют его~длиной и~обозначают ${\displaystyle AB}$ или~${\displaystyle |AB|}$.
Отрезки ${\displaystyle AB}$ и~${\displaystyle CD}$ пропорциональны отрезкам ${\displaystyle A_{1}B{1}}$ и~${\displaystyle C_{1}D_{1}}$, если выполняется равенство:
\begin{equation}\label{eq:prog-segments-1}
\frac{AB}{A_{1}B_{1}} = \frac{CD}{C_{1}D_{1}}
\end{equation}
Таким образом,
\begin{equation}\label{eq:prog-segments-2}
AB,\ CD \propto A_{1}B{1},\ C_{1}D_{1}:\ \frac{AB}{A_{1}B_{1}} = \frac{CD}{C_{1}D_{1}} 
\end{equation}



\subsubsection{Треугольники и~их~свойства}\label{triangles}
\paragraph{Признаки равенства треугольников}
Первым признаком равенства треугольников является признак по~двум сторонам и~углу между ними.
\begin{theorem}
	Если две стороны и~угол между ними одного треугольника соответственно равны двум сторонам и~углу между ними другого треугольника, то~эти~треугольники равны.
\end{theorem}
Вторым признаком равенства треугольников является признак по~двум углам и~стороне между ними.
\begin{theorem}
	Если сторона и~два~прилежащих угла одного треугольника соответственно равны стороне и~двум прилежащим углам другого треугольника, то~эти~треугольники равны.
\end{theorem}
Вторым признаком равенства треугольников является признак по~трём сторонам.
\begin{theorem}
	Если все~стороны треугольника соответственно равны сторонам другого треугольника, то~эти~треугольники равны.
\end{theorem}

\paragraph{Замечательные прямые и~точки треугольника}
В~данном материале будут рассмотрены простейшие замечательные прямые:
\begin{itemize}
	\item медиана;
	\item биссектриса;
	\item высота;
	\item прямая Эйлера,
\end{itemize}
а~также простейшие замечательные точки:
\begin{itemize}
	\item центроид;
	\item инцентр;
	\item ортоцентр.
\end{itemize}

\begin{description}
	\item[Медиана треугольника] "--- отрезок, соединяющий вершину треугольника с~серединой противоположной стороны. Иногда \emph{медианой} называют также прямую, содержащую этот отрезок. Точка пересечения медианы со~стороной треугольника называется \textbf{основанием медианы}.
\end{description}
Все~три медианы треугольника пересекаются в~одной точке, которая называется \textbf{центроидом} либо центром тяжести треугольника, и~делятся этой точкой на~две части в~отношении 2:1, считая от~вершины.

На~рисунке~\ref{fig:triangle-lines-points} медианы выделены зелёным цветом, а~точка~J является центроидом.


\begin{description}
	\item[Биссектриса треугольника] "--- отрезок биссектрисы угла, проведённый от~вершины угла до~её~пересечения с~противолежащей стороной. Точка пересечения биссектрисы угла треугольника с~его стороной, не~являющейся стороной этого угла, называется \textbf{основанием биссектрисы}.
\end{description}

Любая точка на~биссектрисе равноудалена от~сторон данного угла, т.\,е.~длины перпендикуляров, опущенных из~любой точки биссектрисы на~стороны угла, одинаковы. Также верно и~обратное: если точка равноудалена от~сторон угла, она~лежит на~его~биссектрисе.

Все~три биссектрисы внутренних углов треугольника пересекаются в~одной точке, называемом \textbf{инцентр} и~являющейся центром вписанной в~этот треугольник окружности. Биссектриса делит сторону, на~которую она~опускается, на~отрезки пропорциональные прилежащим к~ним сторонам.

На~рисунке~\ref{fig:triangle-lines-points} \emph{биссектрисы} выделены синим цветом, а~точка~D является \emph{инцентром}.

\begin{description}
	\item[Высота треугольника] "--- перпендикуляр, опущенный из~вершины треугольника на~противоположную сторону (точнее, на~прямую, содержащую противоположную сторону).
\end{description}
В зависимости от типа треугольника высота может содержаться внутри треугольника (для остроугольного треугольника), совпадать с его стороной (являться катетом прямоугольного треугольника) или проходить вне треугольника у тупоугольного треугольника. Все 3 высоты треугольника пересекаются в~1 точке, называемой \textbf{ортоцентром}.

На~рисунке~\ref{fig:triangle-lines-points} \emph{высоты}~(и~их~продолжения) выделены оранжевым цветом, а~точка~Q является \emph{ортоцентром}.

\begin{description}
	\item[Пряма́я Э́йлера] "--- прямая, проходящая через центр описанной окружности и ортоцентр треугольника.
\end{description}

На~рисунке~\ref{fig:triangle-lines-points} \emph{прямая Эйлера}  выделена фиолетовым цветом.

\begin{figure}[ht]
	\centering % Центрируем картинку
	\includegraphics[width=0.8\textwidth]{triangle-lines-points+.pdf}
	\caption{Основные замечательные прямые и~точки треугольника}\label{fig:triangle-lines-points}
\end{figure} 

\paragraph{Серединный перпендикуляр}
\begin{description}
	\item[Серединный перпендикуляр] "---  прямая, перпендикулярная данному отрезку и~проходящая через его~середину.
\end{description}
Свойства:
\begin{enumerate}
	\item Серединные перпендикуляры к~сторонам треугольника (или~другого многоугольника, для которого существует описанная окружность) пересекаются в~одной точке — центре описанной окружности. У остроугольного треугольника эта точка лежит внутри, у тупоугольного — вне треугольника, у прямоугольного — на середине гипотенузы.
	\item Любая точка серединного перпендикуляра к~отрезку равноудалена от~концов этого отрезка, т.\,е., от~вершин треугольника. Верно и~обратное утверждение: каждая точка, равноудалённая от~концов отрезка (вершин треугольника), лежит на~серединном перпендикуляре к~нему.
	\item В~равнобедренном треугольнике высота, биссектриса и~медиана, проведённые из~вершины угла с~равными сторонами, совпадают и~являются серединным перпендикуляром, проведённым к~основанию треугольника, а~два других серединных перпендикуляра равны между собой.	
\end{enumerate}
На~рисунке~\ref{fig:triangle-lines-points} \emph{серединные перпендикуляры} выделены малиновым цветом и~пересекаются в~точке~R.
\paragraph{Свойства треугольника}
\begin{enumerate}
	\item Сумма углов любого треугольника равна 180\,\textdegree.
	\item В~прямоугольном треугольнике катет, лежащий против угла в~30\,\textdegree, равен половине гипотенузы.
\end{enumerate}

\paragraph{Площадь треугольника}
Площадь треугольника:
\begin{equation}\label{eq:triangle-square-1}
S=\frac{1}{2}ah,
\end{equation} 
где~\textit{a} "--- длина основания, \textit{h} "--- длина высоты, опущенной на~основание.
Площадь прямоугольного треугольника также равна
\begin{equation}\label{eq:triangle-square-2}
S=\frac{ab}{2},
\end{equation}
где~\textit{a, b} "--- длины катетов.
Формула~\ref{eq:triangle-square-1} имеет следствие: \emph{площади треугольников, имеющих равные высоты, относятся друг к~другу также~как относятся соответствующие основания этих треугольников}. Из~этого следствия вытекает ещё~одно: площади треугольников относятся друг к~другу также как~произведение их~сторон, имеющих равный общий соответственный угол.

\paragraph{Теорема Пифагора}
\begin{theorem}
	В~прямоугольном треугольнике сумма квадратов катетов равна квадрату гипотенузы. Гипотенузой является сторона, лежащая против прямого угла.
\end{theorem}
 В~настоящее время известно свыше двухсот  доказательств данной теоремы, являющейся, пожалуй, самой известной теоремой в~математике. 
 
\paragraph{Теорема обратная теореме Пифагора}
\begin{theorem}
	Если в~треугольнике квадрат одной из~его сторон равен сумме квадратов других сторон, такой треугольник является прямоугольным. Если квадрат одной из~сторон больше суммы квадратов других сторон, угол, лежащий против первой стороны, является тупым. Если квадрат одной из~сторон меньше суммы квадратов других сторон, угол, лежащий против первой стороны, является острым.
\end{theorem}

\paragraph{Формула Герона}
Данная формула позволяет находить площадь треугольника по~длине его~сторон и~имеет вид:
\begin{equation}\label{eq:formula-geron}
S=\sqrt{p(p-a)(p-b)(p-c)},
\end{equation}
где, a, b, c "--- стороны треугольника, p "--- его~полупериметр, определяемый по~формуле:
\begin{equation}\label{eq:half-perimeter}
p=\frac{a+b+c}{2}
\end{equation}

\paragraph{Подобные треугольники}
\subparagraph{Определение и~основные свойства}
Строгое определение:
\begin{description}
	\item[Фигура F называется подобной фигуре F'] если существует преобразование подобия, при~котором ${\textstyle F \mapsto F'}$. Подобие фигур является отношением эквивалентности.
\end{description}
Иными словами каждой точке фигуры $\textstyle F$ соответствует какая-то единственная точка фигуры $\textstyle F'$. При~этом выполняется соотношение
\begin{equation}\label{eq:similar-figures-1}
\frac{MN}{M_{1}N_{1}}=k:\ \forall M,\ N,
\end{equation}
где $\textstyle k$ "--- коэффициент пропорциональности.
Нестрогое определение:
\begin{description}
	\item[Подобными фигурами] являются фигуры, имеющие одинаковую форму. 
\end{description}

\begin{enumerate}
	\item Подобие есть взаимно однозначное отображение евклидова пространства на~себя.
	\item Подобие является аффинным преобразованием плоскости.
	\item Подобие сохраняет порядок точек на~прямой, т.\,е.~ если точка ${\textstyle B}$ лежит между точками ${\textstyle A,\ C}$, и ${\textstyle B',\ A',\ C'}$ "--- соответствующие их~образы при~некотором подобии, то~${\textstyle B'}$ также лежит между точками ${\textstyle A'}$ и~${\textstyle C'}$.
	\item Точки, не~лежащие на~прямой, при~любом подобии переходят в~точки, не~лежащие на~одной прямой.
	\item Подобие преобразует прямую в~прямую, отрезок в~отрезок, луч в~луч, угол в~угол, окружность в~окружность, n-угольник в~n-угольник.
	\item Подобие сохраняет величины углов между кривыми.
	\item Подобие с~коэффициентом ${\textstyle k\not =1}$, преобразующее каждую прямую в~параллельную ей~прямую, является гомотетией с~коэффициентом ${\textstyle k}$ либо ${\textstyle -k}$.
	\item Каждое подобие можно рассматривать как~композицию движения ${\textstyle D}$ и~некоторой гомотетии ${\textstyle \Gamma}$ с~положительным коэффициентом.
	\item Подобие называется собственным (несобственным), если движение ${\textstyle D}$ является собственным (несобственным). Собственное подобие сохраняет ориентацию фигур, а~несобственное изменяет ориентацию на~противоположную.
	\item Площади подобных фигур пропорциональны квадратам их~сходственных линий (например, сторон). Так, площади кругов пропорциональны отношению квадратов их~радиусов.
\end{enumerate}
 
\begin{description}
	\item[Подобными треугольниками] являются треугольники, имеющие одинаковые углы. 
\end{description}  
Стороны, лежащие против равных углов в~двух треугольниках, называются \textbf{сходственными}. Математическая запись условий подобия треугольников выглядит следующим образом:
\begin{equation}\label{key}
\triangle ABC \sim \triangle A_{1}B_{1}C_{1}: \angle A = \angle A_{1},\ \angle B = \angle B_{1},\ \angle A = \angle A_{1} \wedge \ \frac{AB}{A_{1}B_{1}}=\frac{AC}{A_{1}C_{1}}=\frac{BC}{B_{1}C_{1}}=k,
\end{equation}
где~\textit{k} "--- коэффициент подобия.

Отношение периметров подобных треугольников равно коэффициенту подобия.

Отношение площадей подобных треугольников равно квадрату коэффициента подобия.

\subparagraph{Признаки подобия треугольников}\label{similarity-of-triangles-signs}

\textbf{Первым} признаком подобия треугольников является равенство двух углов.

\textbf{Вторым} признаком подобия треугольников является равенство соответствующих углов и~пропорциональность образующих их~сторон.

\textbf{Третьим} признаком подобия треугольников является пропорциональность всех ~сторон.

\subparagraph{Подобие произвольных фигур}

\paragraph{Средняя линия треугольника}
\begin{description}
	\item[Средняя линия треугольника] "--- отрезок, соединяющий середины двух его~сторон.
\end{description}
Треугольник имеет три~средние линии. На~рисунке~\ref{fig:triangle-lines-points} \emph{средние линии} выделены коричневым цветом.

Свойства средней линии:
\begin{enumerate}
	\item средняя линия параллельна третьей стороне треугольника и~равна половине её~длины;
	\item периметр треугольника, образуемого средними линиями, равен половине периметра основного треугольника.
\end{enumerate}

\paragraph{Пропорциональные отрезки в~прямоугольном треугольнике}
Рассмотрим прямоугольный треугольник ${\displaystyle ABC}$ (рисунок~\ref{fig:rect-triangle-1}) c~прямым углом ${\displaystyle \beta}$ у~вершины ${\displaystyle C}$. Опустим высоту ${\displaystyle h}$ из~угла ${\displaystyle \beta}$ на~сторону ${\displaystyle c}$ в~точку ${\displaystyle D}$, получив, таким образом, два~треугольника ${\displaystyle ACD,\ BDC}$, являющихся прямоугольными.

Докажем подобие треугольников ${\displaystyle ACD,\ BDC}$ друг другу, а~также треугольнику ${\displaystyle ABC}$. Докажем что~$\textstyle \triangle ACD \sim \triangle ABC$. Оба~этих треугольника прямоугольные и~имеют одинаковый угол $\textstyle \alpha = 49.1\,\textdegree$. Равенство двух углов означает, что~треугольники подобны по~первому признаку подобия треугольников (см.~\ref{similarity-of-triangles-signs}). Аналогичным образом доказывается, что~$\textstyle \triangle BDC \sim \triangle ABC$. Из~этого следует, что~$\textstyle \triangle ACD \sim \triangle BDC$.

В~${\text \triangle ABC}$ стороны ${\textstyle a,\ b}$ "--- катеты, сторона ${\textstyle c}$ "--- гипотенуза. В~таком случае, отрезок ${\textstyle AD}$ "--- проекция катета ${\textstyle b}$ на~гипотенузу ${\textstyle c}$, отрезок ${\textstyle BD}$ "--- проекция катета ${\textstyle a}$ на~гипотенузу ${\textstyle c}$. Обозначим эти~отрезки, являющиеся проекциями, как~${\textstyle b_h}$ и~${\textstyle a_h}$ соответственно.

Введём понятие \textbf{среднего геометрического} ${\textstyle x}$ для~переменных ${\textstyle i,\ j}$
\begin{equation}\label{eq:average-geom}
x=\sqrt{ij}
\end{equation}
Докажем, что~высота ${\textstyle h}$ является средним геометрическим проекций катетов ${\textstyle a,\ b}$. Т.\,е.~что~${\textstyle h=\sqrt{a_h \times b_h}}$. Из~$\textstyle \triangle ACD \sim \triangle BDC$ следует, что~их~стороны сходственно пропорциональны. Тогда 
$\frac{h}{b_h}=\frac{a_h}{h} \Rightarrow h^2=a_{h}b_{h} \Rightarrow h=\sqrt{a_h \times b_h}$. Таким образом, было доказано, что~высота прямоугольного треугольника, опущенная из~его~прямого угла, равна среднему геометрическому между проекциями катетов на~гипотенузу. Из~этого также следует, что~${\textstyle a=\sqrt{a_{h}c},\ b=\sqrt{b_{h}c}}$, а~также ${\textstyle h=\frac{ab}{c}}$.


\begin{figure}[ht]
	\centering % Центрируем картинку
	\includegraphics[width=0.8\textwidth]{rect-triangle.pdf}
	\caption{Прямоугольный треугольник}\label{fig:rect-triangle-1}
\end{figure}

\paragraph{Тригонометрические функции острого угла прямоугольного треугольника}
\subparagraph{Определения функций}
\begin{description}
	\item[Синусом] острого угла прямоугольного треугольника называется отношение противолежащего катета к~гипотенузе.
\end{description}
Так, синусом угла ${\textstyle \alpha}$ на~\ref{fig:rect-triangle-1}, будет являться отношение ${\textstyle \sin \alpha = \dfrac{a}{c} \approx 0.756}$.
\begin{description}
	\item[Косинусом] острого угла прямоугольного треугольника называется отношение прилежащего катета к~гипотенузе.
\end{description}
Так, косинусом угла ${\textstyle \alpha}$ на~\ref{fig:rect-triangle-1}, будет являться отношение ${\textstyle \cos \alpha = \dfrac{b}{c} \approx 0.655}$.
\begin{description}
	\item[Тангенсом] острого угла прямоугольного треугольника называется отношение противолежащего катета к~прилежащему.
\end{description}
Так, тангенсом угла ${\textstyle \alpha}$ на~\ref{fig:rect-triangle-1}, будет являться отношение ${\textstyle \tg \alpha = \dfrac{a}{b} \approx 1.155}$.
\begin{description}
	\item[Котангенсом] острого угла прямоугольного треугольника называется отношение прилежащего катета к~противолежащему.
\end{description}
Так, тангенсом угла ${\textstyle \alpha}$ на~\ref{fig:rect-triangle-1}, будет являться отношение ${\textstyle \ctg \alpha = \dfrac{b}{a} \approx 0.866}$.
\begin{description}
	\item[Секансом] острого угла прямоугольного треугольника называется отношение гипотенузы к прилежащему катету.
\end{description}
Так, секансом угла ${\textstyle \alpha}$ на~\ref{fig:rect-triangle-1}, будет являться отношение ${\textstyle \sec \alpha = \dfrac{c}{b} \approx 1.527}$.
\begin{description}
	\item[Косекансом] острого угла прямоугольного треугольника называется отношение гипотенузы к противолежащему катету.
\end{description}
Так, секансом угла ${\textstyle \alpha}$ на~\ref{fig:rect-triangle-1}, будет являться отношение ${\textstyle \cosec \alpha = \dfrac{c}{a} \approx 1.323}$.

Любую тригонометрическую функцию можно выразить через любую другую тригонометрическую функцию с~тем~же аргументом (с~точностью до~знака из-за неоднозначности раскрытия квадратного корня). Нижеприведённые формулы верны для~$\textstyle 0 < x < \dfrac{\pi}{2}$. В~таблице~\ref{tab:trig-func-rel} приводятся соотношения тригонометрических функций.

\begin{table}[ht]
	\caption{Соотношения тригонометрических функций}  \label{tab:trig-func-rel}
	\centering% центрируем таблицу
	\small
	\begin{tabularx}{\textwidth}{>{$}c<{$}>{$}c<{$}>{$}c<{$}>{$}c<{$}>{$}c<{$}>{$}c<{$}>{$}c<{$}} 
		\hline
		&\sin    &\cos                &\tg    &\ctg &\sec  &\cosec   \\
		\hline
		\sin x=&\sin x  &\sqrt{1-\cos^{2} x}&\dfrac{\tg x}{\sqrt{1+\tg^{2} x}}&\dfrac{1}{\sqrt{\ctg^{2} x +1}}&\dfrac{\sec^{2} x-1}{\sec x}&\dfrac{1}{\cosec x}\\
		\hline
		\cos x=&\sqrt{1-\sin^{2} x}&\cos x&\dfrac{1}{\sqrt{1+\tg^{2}x}}&\dfrac{\ctg x}{\sqrt{\ctg^2 x +1}}&\dfrac{1}{\sec x}&\dfrac{\sqrt{\cosec^2 x - 1}}{\cosec x}\\
		\hline
		\tg x=&\dfrac{\sin x}{\sqrt{1-\sin^2 x}}&\dfrac{\sqrt{1-\cos^2 x}}{\cos x}&\tg x&\dfrac{1}{\ctg x}&\sqrt{\sec^2 x -1}&\dfrac{1}{\sqrt{\cosec^2 x -1}}\\
		\hline
		\ctg x=&\dfrac{\sqrt{1-\sin^2 x}}{\sin x}&\dfrac{\cos x}{\sqrt{1-\cos^2 x}}&\dfrac{1}{\tg x}&\ctg x&\dfrac{1}{\sqrt{\sec^2 x -1}}&\sqrt{\cosec^2 x -1}\\
		\hline
		\sec x=&\dfrac{1}{\sqrt{1-\sin^2 x}}&\dfrac{1}{\cos x}&\sqrt{1+\tg^2 x}&\dfrac{\sqrt{\ctg^2 x +1}}{\ctg x}&\sec x &\dfrac{\cosec x}{\sqrt{\cosec^2 x-1}}\\
		\hline	
		\cosec x=&\dfrac{1}{\sin x}&\dfrac{1}{\sqrt{1-\cos^2 x}}&\dfrac{\sqrt{1+\tg^2 x}}{\tg x}&\sqrt{\ctg^2 x+1}&\dfrac{\sec x}{\sqrt{\sec^2 x -1}}&\cosec x\\
		\hline
	\end{tabularx}
\normalsize
\end{table}

Помимо шести основных вышеуказанных функций существует ряд относительно редко используемых в~настоящее время, например \emph{синус-верзус}, \emph{косинус-верзус}, \emph{гаверсинус}, \emph{гаверкосинус}, \emph{эксеканс}, \emph{экскосеканс}. Рассмотрение данных функций в~контексте математической подготовки оценщика является избыточным.

\subparagraph{Тригонометрические тождества}
\begin{equation}\label{eq:trigon-identity-1}
\sin^2 \alpha + \cos^2 \alpha =1:\ \forall \alpha
\end{equation}
Данное выражение называется основным тригонометрическим тождеством.
Другие тождества приведены ниже.
\begin{equation}\label{eq:trigon-identity-2}
\tg^2 \alpha +1 = \frac{1}{\cos^2 \alpha} = \sec^2 \alpha:\ \alpha \neq \frac{\pi}{2} + \pi n:\ n \in \mathbb{Z}
\end{equation}

\begin{equation}\label{eq:trigon-identity-3}
\ctg^2 \alpha + 1 = \frac{1}{\sin^2 \alpha} = \cosec^2 \alpha:\ \alpha \neq \pi n,\ n \in \mathbb{Z}
\end{equation}

\begin{equation}\label{eq:trigon-identity-4}
\tg \alpha \ctg \alpha = 1:\ \alpha \neq \frac{\pi n}{2},\ n \in \mathbb{Z}
\end{equation}

\subparagraph{Свойства и~значения тригонометрических функций}
Синус и~косинус вещественного аргумента представляют собой периодические, непрерывные и~бесконечно дифференцируемые вещественнозначные функции. Остальные четыре функции на вещественной оси также вещественнозначны, периодичны и~бесконечно дифференцируемы, за~исключением счётного числа разрывов второго рода: у~тангенса и~секанса в~точках  $\textstyle \pm \pi n+ \dfrac{\pi}{2}$, а~у~котангенса и~косеканса "--- в~точках $\textstyle \pm \pi n$. Последнее обстоятельство связано с~тем, что~вычисление значений этих четырёх функций приводит к~операции деления на~ноль, что~означает, что~в~этих точках их~значение не~определено, а~в~окрестности этих точек "--- стремится к~бесконечности. На~рисунке~\ref{fig:sin-cos-etc} показаны значения шести основных тригонометрических функций в~зависимости от~значения угла в~градусах.

\begin{figure}[ht]
	\centering % Центрируем картинку
	\includegraphics[width=0.8\textwidth]{sin-cos-etc.pdf}
	\caption{Значения шести основных тригонометрических функций в~зависимости от~значения угла в~градусах}\label{fig:sin-cos-etc}
\end{figure}

\subsubsection{Многоугольники}

\paragraph{Общие сведения}

Рассмотрим рисунок~\ref{fig:segments-1}. На~нём изображена фигура, состоящая из~последовательных отрезков~(\foreignlanguage{english}{segment}) \textit{f, g, h, i, j}. Соседние отрезки, например \textit{f} и~\textit{g}, \textit{g} и~\textit{h}, \textit{i} и~\textit{g}, называются \emph{смежными}. Если отрезки \textit{f, g, h, i, j} не~лежат на~одной прямой, то~образуемая ими~фигура называется \emph{ломаной}, сами отрезки являются её~\emph{звеньями}, а~точки \textit{A, B, C, D, E, F} "--- её~\emph{вершинами}. Длиной \emph{ломаной} является сумма длин образующих её~отрезков.

В~случае, когда крайние точки \emph{ломаной} совпадают, такую \emph{ломаную} называют \emph{замкнутой}. В~случае, когда несмежные отрезки замкнутой ломаной не~имеют общих точек, образуемая ими~фигура называется многоугольником~(\foreignlanguage{english}{polygon}), см.~рисунок~\ref{fig:polygon-1}. Многоугольник, имеющий \textit{n}~вершин, называется n-угольником. Примером многоугольника, является, в~частности, треугольник, рассмотренный ранее в~\ref{triangles}. Число сторон многоугольника равно числу его~вершин. Две вершины многоугольника, лежащие на~одной стороне называются  соседними. Таким образом, соседними являются вершины \textit{A} и~\textit{B}, \textit{B} и~\textit{C}, \textit{F} и~\textit{A}. Отрезок, соединяющий две вершины многоугольника, не~являющиеся соседними, называется \emph{диагональю}. На~рисунке~\ref{fig:polygon-1} диагональю является отрезок \textit{l}. Любой многоугольник делит плоскость на~внешнюю и~внутреннюю части. Максимально возможное число диагоналей \emph{выпуклого многоугольника} определяется по~формуле
\begin{equation}\label{eq:n-polygon-vertex}
\frac{n(n-3)}{2},
\end{equation}
где \textit{n} "--- число вершин многоугольника.

\begin{description}
	\item[Выпуклым многоугольником] называется многоугольник, лежащий в~одной полуплоскости относительно прямой, проходящей через любую его~сторону.
\end{description}
Пример выпуклого многоугольника показан на~рисунке~\ref{fig:polygon-1}, на~котором его~стороны изображены чёрным цветом, диагональ "--- синим, прямые, проходящие через его~стороны, "--- красным.
Пример невыпуклого многоугольника показан на~рисунке~\ref{fig:polygon-2}, на~котором его~внутренняя сторона закрашена красно-коричневым цветом.

В~данном материале чаще всего речь будет идти о~выпуклых многоугольниках.
 
\begin{figure}[ht]
\centering % Центрируем картинку
\includegraphics[width=0.8\textwidth]{segments-1.pdf}
\caption{Последовательные отрезки}\label{fig:segments-1}
\end{figure}

\begin{figure}[ht]
\centering % Центрируем картинку
\includegraphics[width=0.8\textwidth]{polygon-1.pdf}
\caption{Выпуклый многоугольник}\label{fig:polygon-1}
\end{figure}  

\begin{figure}[ht]
\centering % Центрируем картинку
\includegraphics[width=0.8\textwidth]{polygon-2.pdf}
\caption{Невыпуклый многоугольник}\label{fig:polygon-2}
\end{figure} 

\begin{description}
	\item[Правильным многоугольником] называется многоугольник, все~стороны и~углы которого равны.
\end{description} 

Выпуклые многоугольники обладают рядом свойств:
\begin{enumerate}
	\item опускание всех диагоналей из~любой вершины приводит к~образованию \textit{n-2} треугольников;
	\item вследствие этого и~в~силу правила равенства суммы углов треугольника 180\,\textdegree можно сделать вывод о~том, что~сумма углов многоугольника равна $180\,\textdegree \times (n-2)$;
	\item сумма внешних~(смежных) углов многоугольника равна 360\,\textdegree.
\end{enumerate}  
На~рисунке~\ref{fig:polygon-3} показаны внешние углы правильного многоугольника.
\begin{figure}[ht]
	\centering % Центрируем картинку
	\includegraphics[width=0.8\textwidth]{polygon-3.pdf}
	\caption{Смежные углы многоугольника}\label{fig:polygon-3}
\end{figure}

\paragraph{Параллелограмм}
\begin{description}
	\item[Параллелограмм] "--- выпуклый четырёхугольник, стороны которого попарно параллельны.
\end{description} 
Основные свойства параллелограмма:
\begin{enumerate}
	\item противоположные стороны равны между собой;
	\item противоположные углы равны между собой;
	\item точка пересечения диагоналей делит их~пополам;
	\end{enumerate}
Все основные свойства параллелограмма показаны на~рисунке~\ref{fig:parallelogram-1}.

\begin{figure}[ht]
	\centering % Центрируем картинку
	\includegraphics[width=0.8\textwidth]{parallelogram-1.pdf}
	\caption{Параллелограмм и~его~основные свойства}\label{fig:parallelogram-1}
\end{figure}

\paragraph{Трапеция}
\begin{description}
	\item[Трапеция] "--- выпуклый четырёхугольник, две стороны которого параллельны.
\end{description}
Параллельные стороны трапеции называются её~\emph{основаниями}, две другие "--- \emph{боковыми сторонами}. Трапеция, боковые стороны которой равны между собой, называется \emph{равнобедренной}. Углы при~основании равнобедренной трапеции равны между собой. Диагонали равнобедренной трапеции равны между собой. 

Трапеция, один из~её~углов равен 90\,\textdegree, называется \emph{прямоугольной}. В~прямоугольной трапеции два угла равны 90\,\textdegree.

Равнобедренная трапеция и~её~основные свойства показана на~рисунке~\ref{fig:trapeze}.

\begin{figure}[ht]
	\centering % Центрируем картинку
	\includegraphics[width=0.8\textwidth]{trapeze.pdf}
	\caption{Равнобедренная трапеция и~её~основные свойства}\label{fig:trapeze}
\end{figure}

\subparagraph{Средняя линия трапеция}
\begin{description}
	\item[Средняя линия трапеции] "--- отрезок, соединяющий середины боковых сторон.
\end{description}
Средняя линия трапеции параллельна её~основаниям и~равна половине их~суммы. На~рисунке~\ref{fig:trapeze} средней линией является отрезок~${\textstyle e}$. 

\paragraph{Прямоугольник, ромб и~квадрат}
\begin{description}
	\item[Прямоугольник] "--- параллелограмм, все~углы которого являются прямыми.
\end{description}
Из~определения следует, что~прямоугольник обладает всеми свойствами параллелограмма. Диагонали прямоугольника равны между собой и~делятся пополам в~точке пересечения.

\begin{description}
	\item[Ромб] "--- параллелограмм, все~стороны которого равны.
\end{description}
Из~определения следует, что~ромб обладает всеми свойствами параллелограмма. Диагонали ромба пересекаются под~прямым углом.

\begin{description}
	\item[Квадрат] "--- прямоугольник, все~стороны которого равны.
\end{description}
Из~определения следует, что~ромб обладает всеми свойствами прямоугольника и, как~следствие "--- параллелограмма. Также квадрат обладает всеми свойствами ромба. Отличие квадрата от~ромба в~том, что~во-первых углы ромба могут отличаться от~90\,\textdegree, во-вторых, диагонали ромба могут не~быть равны между собой. Иными словами, любой квадрат является ромбом, но~не~любой ромб является квадратом.

Прямоугольник, ромб и~квадрат, а~также их~основные свойства показаны на~рисунке~\ref{fig:rectangle-rhombus-square}.

\begin{figure}[ht]
	\centering % Центрируем картинку
	\includegraphics[width=0.8\textwidth]{rectangle-rhombus-square.pdf}
	\caption{Прямоугольник, ромб и~квадрат и~их~основные свойства}\label{fig:rectangle-rhombus-square}
\end{figure}

\paragraph{Площадь многоугольника}
\begin{description}
	\item[Площадь многоугольника] "--- величина части плоскости внутри многоугольника.
\end{description}
Свойства площади многоугольника.
\begin{enumerate}
	\item Если фигуры равны, то~и~их~площади равны.
	\item Если фигура разбита на~несколько частей, то~сумма фигуры равна сумме частей.
\end{enumerate}
Площадь квадрата:
\begin{equation}\label{eq:square-square}
S=a^2,
\end{equation}
где \textit{a} "--- длина стороны.

Площадь прямоугольника:
\begin{equation}\label{eq:rectangle-square}
S=ab,
\end{equation}
где \textit{a, b} "--- длина не~равных между собой сторон.

Площадь параллелограмма:
\begin{equation}\label{eq:parallelogram-square}
S=ah,
\end{equation}
где \textit{a} "--- длина стороны, \textit{h} "--- длина высоты, опущенной на~эту сторону. На~рисунке~\ref{fig:parallelogram-1} такой стороной является, например \textit{d}, а~высотой "--- отрезок \textit{h}. Также возможно использование, например \textit{c} и~\textit{t}.

Площадь трапеции:
\begin{equation}\label{eq:parallelogram-square}
S=\frac{ab}{2}h,
\end{equation}
где, \textit{a, b} "--- основания трапеции, \textit{h} "--- её~высота.


\subsubsection{Осевая симметрия}
\begin{description}
	\item[Осевая симметрия] "--- симметрия относительно прямой.
\end{description}

\begin{description}
	\item[Центральная симметрия] "--- симметрия относительно точки.
\end{description} 

\subsubsection{Взаимное расположение прямой и~окружности}
Рассмотрим рисунок~\ref{fig:line-circle}, на~котором изображена окружность ${\textstyle c}$ с~радиусом ${\textstyle r=1}$. На~одном плоскости с~этой окружностью мы~видим четыре~прямые ${\textstyle f}$ (голубая), ${\textstyle j}$ (зелёная), ${\textstyle e}$ (красная), ${\textstyle g}$ (фиолетовая). Минимальное расстояние ${\textstyle h}$, получаемое путём опускания перпендикуляра из~центра окружности на~прямую, между центром окружности и~прямой может быть меньше радиуса, больше его либо равно ему. 
\begin{itemize}
	\item В~случае, если расстояние ${\textstyle h}$ равно радиусу, такая прямая называется \textbf{касательной} и~имеет одну общую точку с~окружностью. На~рисунке~\ref{fig:line-circle} такой прямой является прямая ${\textstyle g}$.
	\item В~случае, если расстояние ${\textstyle h}$ меньше радиуса, такая прямая называется \textbf{секущей} и~имеет две общие точки с~окружностью. На~рисунке~\ref{fig:line-circle} такими прямыми являются прямые ${\textstyle f,\ e}$. В~случае прохождения прямой через центр окружности является частным случаем равенства ${\textstyle h < r}$.
	\item В~случае, если если расстояние ${\textstyle h}$ больше радиуса, такая прямая не~имеет общих точек с~окружностью. На~рисунке~\ref{fig:line-circle} такой прямой является прямая ${\textstyle j}$.
\end{itemize}

\begin{figure}[ht]
	\centering % Центрируем картинку
	\includegraphics[width=0.8\textwidth]{line-circle.pdf}
	\caption{Взаимное расположение прямой и~окружности}\label{fig:line-circle}
\end{figure}

\begin{table}[ht]
	\caption{Варианты взаимного расположения прямой и~окружности}  \label{tab:line-circ}
	\centering% центрируем таблицу
	\begin{tabularx}{\textwidth}{ccc} 
		\hline
	Условие&Число общих точек&Название прямой\\ \hline
	${\textstyle h<r}$&2&секущая\\ \hline
	${\textstyle h=r}$&1&касательная\\ \hline
	${\textstyle h>r}$&0&"---\\ \hline
	\end{tabularx}
	\end{table}

\subsubsection{Градусная мера дуги окружности, теорема о~вписанном угле}
Рассмотрим окружность ${\textstyle c}$, изображённую на~рисунке~\ref{fig:central-inscribed-angle}.
\begin{description}
	\item[Дуга] "--- часть окружности, ограниченная двумя точками
\end{description}
Дугой является, например ${\textstyle \smile CFC'}$.
\begin{description}
	\item[Полуокружность] "--- дуга, образуемая точками, отрезок, соединяющий которые, проходит через центр окружности.
\end{description}
\begin{description}
	\item[Центральный угол] "--- угол, вершина которого находится в~середине окружности.
\end{description}
Таким углом является угол ${\textstyle COC'}$, обозначенный как~${\textstyle \angle \alpha}$. В~случае, если дуга меньше полуокружности, то~её~градусная мера равна градусной мере центрального угла. Таким образом, градусная мера ${\textstyle \smile CFC' = \angle \alpha = 100\,\textdegree}$. В~случае, если дуга больше полуокружности, её~градусная мера определяется по~формуле
\begin{equation}\label{eq:arc-1}
\smile ABC = 360 - \alpha.
\end{equation}
Таким образом, градусная мера ${\textstyle \smile CGC' = 360\,\textdegree - \angle \alpha = 260\,\textdegree}$.
\begin{description}
	\item[Вписанный угол] "--- угол, вершина которого лежит на~окружности.
\end{description}
Вписанным углом является, например угол ${\textstyle \smile CBC'}$, обозначенный как~${\textstyle \angle \beta}$.
\begin{theorem}
	Градусная мера вписанного угла равна половине градусной меры центрального угла, опирающегося на~те~же точки на~окружности.
\end{theorem}
При~этом, вершина вписанного угла может быть любой. Как~видно на~рисунке~\ref{fig:central-inscribed-angle} ${\textstyle \angle \beta = \angle \gamma = \angle \gamma = \angle \varepsilon = \dfrac{1}{2}\angle \alpha = 50\,\textdegree}$. Вписанный угол, опирающийся на~полуокружность "--- прямой. Например ${\textstyle \angle \zeta = 90\,\textdegree}$.

\begin{figure}[ht]
	\centering % Центрируем картинку
	\includegraphics[width=0.8\textwidth]{central-inscribed-angle.pdf}
	\caption{Центральный и~вписанный углы}\label{fig:central-inscribed-angle}
\end{figure}

\subsubsection{Вписанная и~описанная окружность}
\begin{description}
	\item[Вписанная в~многоугольник окружность] "--- окружность, касающаяся всех его~сторон.
\end{description}
\begin{description}
	\item[Описанная вокруг многоугольника окружность] "--- окружность, касающаяся всех его~вершин.
\end{description}
Для~построения вписанной в~треугольник окружности необходимо использовать в~качестве её~центра \emph{инцентр} "--- точку пересечения биссектрис углов при~его~вершинах. При~этом сам~треугольник будет являться описанным для~данной окружности. Для~построения описанной вокруг треугольника окружности необходимо использовать в~качестве её~центра  точку пересечения серединных перпендикуляров. При~этом сам~треугольник будет являться вписанным для~данной окружности. Вписанная и~описанная окружности треугольника показаны на~рисунке~\ref{fig:triangle-circles}.

Окружность может быть вписана в~любой треугольник равно как~и~описана около него. Окружность может быть вписана в~четырёхугольник, если равны суммы длин его~противоположных сторон. Окружность может быть описана около четырёхугольника тогда и~только тогда, когда сумма его~противоположных углов равна 180\,$\textdegree$.

\begin{figure}[ht]
	\centering % Центрируем картинку
\includegraphics[width=0.8\textwidth]{triangle-circles.pdf}
\caption{Вписанная и~описанная окружности треугольника}\label{fig:triangle-circles}
\end{figure}

\subsubsection{Векторы}
\paragraph{Понятие вектора}
\begin{description}
	\item[Вектор] "--- направленный отрезок прямой, то~есть отрезок, для~которого указано, какая из~его~граничных точек является началом, а~какая "--- концом.
\end{description}
Вектор с~началом в~точке~${\textstyle A}$ и~концом в~точке~${\textstyle B}$ принято обозначать как~${\textstyle {\overrightarrow {AB}}}$. Векторы также могут обозначаться малыми латинскими буквами со~стрелкой над ними, например~${\textstyle {\vec {a}}}$.
\begin{description}
	\item[Нулевой вектор] "--- вектор, начало которого совпадает с~концом.
\end{description}
Точка является \emph{нулевым вектором}. Нулевой вектор обозначают как~${\textstyle {\vec {0}}}$ либо ${\textstyle {\overrightarrow {AA}}}$.
\begin{description}
	\item[Длина вектора] "--- длина соответствующего отрезка.
\end{description}
\begin{equation}\label{eq:vector-lenght}
|{\overrightarrow {AB}}|=|AB|
\end{equation}
\paragraph{Коллинеарность и~равенство векторов}
Два~ненулевых вектора называются \textbf{коллинеарными}, если они~лежат на~одной прямой либо на~параллельных прямых. Такие векторы обозначаются следующим образом: ${\textstyle \vec{a} \parallel \vec{b}}$. Два коллинеарных вектора могут быть \textbf{сонаправленными} ${\textstyle \vec{a} \uparrow \uparrow   \vec{b}}$ либо \textbf{противоположно направленными} ${\textstyle \vec{a} \textuparrow \textdownarrow \vec{c}}$. Два~вектора являются \emph{сонаправленными}, если они~коллинеарны и~лежат по~одну сторону от~прямой, проходящей через их начало. Два~вектора являются \emph{противоположно направленными}, если они~коллинеарны и~лежат по~разные стороны от~прямой, проходящей через их начало.
Два~вектора равны, если они~сонаправлены и~имеют одинаковую длину.
\begin{equation}\label{eq:vec-equality}
\vec{a}=\vec{b} \Leftrightarrow
	\begin{cases}
	\vec{a}\uparrow \uparrow \vec{b}\\
	|\vec{a}|=|\vec{b}|
	\end{cases}
\end{equation}
\paragraph{Сложение векторов}

\subparagraph{Сложение по~правилу треугольника}

Рассмотрим ${\textstyle \vec{a}\ \vec{b}}$ на~рисунке~\ref{fig:vec-sum-1}. Для~их сложения возьмём произвольную точку~${\textstyle A}$ и~отложим из~неё~${\textstyle \vec{u} = \vec{a}}$. Затем из~конца ${\textstyle \vec{u}}$ в~точке~${\textstyle B}$ отложим ${\textstyle \vec{v} = \vec{b}}$ в~точку~${\textstyle C}$. Затем проведём ${\textstyle \vec{w}}$ из~${\textstyle A}$ в~${\textstyle C}$. Вектор ${\textstyle \vec{w}}$ и~будет являться результатом сложения ${\textstyle \vec{a}\ \vec{b}}$.

\subparagraph{Сложение по~правилу трёх точек}

Если отрезок ${\textstyle {\overrightarrow {AB}}}$ вектор ${\textstyle {\vec {a}}}$, а~отрезок ${\textstyle {\overrightarrow {BC}}}$ изображает вектор ${\displaystyle {\vec {b}}}$, то~$ {\textstyle {\overrightarrow {AC}}}$ изображает вектор ${\textstyle {\vec {a}}+{\vec {b}}}$.

\subparagraph{Сложение по~правилу параллелограмма}

Для~сложения двух векторов ${\textstyle {\vec {a}}}$ и~${\textstyle {\vec {b}}}$ по~\emph{правилу параллелограмма} оба~эти~векторы переносятся параллельно самим себе так, чтобы их~начала совпадали. Тогда вектор суммы задаётся диагональю построенного на~них параллелограмма, исходящей из~их~общего начала. Эта диагональ совпадает с~третьей стороной треугольника при~использовании \emph{правила треугольника}.

\subparagraph{Сложение по~правилу многоугольника (правилу ломаной)}
Начало второго вектора совмещается с~концом первого, начало третьего "--- с~концом второго и~т.\,д., сумма~же ${\textstyle n}$ векторов есть вектор, с~началом, совпадающим с~началом первого, и~концом, совпадающим с~концом ${\textstyle n}$-го (то~есть изображается направленным отрезком, замыкающим ломаную).
 
\subparagraph{Законы сложения векторов}

Коммутативный закон:
\begin{equation}\label{eq:vec-sum-rule1}
\vec{a} + \vec{b} = \vec{b} + \vec{a}.
\end{equation} 

Сочетательный закон:
\begin{equation}\label{eq:vec-sum-rule2}
(\vec{a} + \vec{b}) + \vec{c} = \vec{a} + (\vec{b})+\vec{c}).
\end{equation} 

\begin{figure}[ht]
	\centering % Центрируем картинку
	\includegraphics[width=0.8\textwidth]{vectors-001.pdf}
	\caption{Сложение и~вычитание векторов и~умножение их~на~число}\label{fig:vec-sum-1}
\end{figure} 

\paragraph{Вычитание векторов} 
Разностью ${\textstyle \vec{c}}$ и~${\textstyle \vec{d}}$ является такой ${\textstyle \vec{g}}$, который при~его~сложении с~${\textstyle \vec{d}}$ даёт ${\textstyle \vec{c}}$. См.~рисунок~\ref{fig:vec-sum-1}. Алгоритм вычитания: отложить оба вектора из~одной точки, затем отложить вектор из~конца вычитаемого вектора к~концу вектора, из~которого вычитается. Полученный вектор и~будет результатом.

Другим способом вычитания векторов является сложение вектора, из~которого вычитается, с~вектором, противоположным вычитаемому.
\begin{equation}\label{eq:vec-subtraction-1}
\vec{c}-\vec{d} = \vec{c} + (-\vec{d})
\end{equation}
\begin{equation}\label{eq:vec-subtraction-2}
\vec{d} = -\vec{e}:
\begin{cases}
|\vec{d}|=|\vec{e}|\\
\vec{d}\uparrow \downarrow \vec{e}
\end{cases}
\end{equation}

\paragraph{Произведение вектора на~число}
\begin{equation}\label{eq:vec-mult-number}
\begin{aligned}
k\times \vec{i}=\vec{j} \Rightarrow\\
1)\ |\vec{j}|=|k|\times|\vec{i}|\\
2)\ 
\begin{cases}
k>0 \Rightarrow \vec{j}\uparrow \uparrow \vec{i}\\
k<0 \Rightarrow \vec{j}\uparrow \downarrow \vec{i}\\
k=0 \Rightarrow \vec{j}=0
\end{cases}
\end{aligned}
\end{equation}
Свойства умножения вектора на~число:
\begin{equation}\label{eq:vec-mult-number-prop}
\begin{aligned}
kl(\vec{a})&=k(l\vec{a})\\
k(\vec{a}+\vec{b})&=k\vec{a}+k\vec{b}\\
(k+l)\times \vec{a}&=k\vec{a}\times k\vec{b}
\end{aligned}
\end{equation}

\section{Функции}
\subsection{Понятие функции}
\textbf{Функция} является одним из~самых важных понятий в~математике.
\begin{description}
	\item[Функция] "--- инструкция (набор инструкций) в~соответствии с~которой каждому элементу первого множества соответствует один и~только один элемент второго множества.
\end{description}
В~общем виде функцию можно записать следующим образом.
\begin{equation}\label{function}
y=f(x),
\end{equation}
где y "--- зависимая переменная (значение функции),

x "--- независимая переменная (аргумент функции),

f "--- выражение.

Одним из~ключевых понятий являются \textbf{область определения функции} и~\textbf{область значения функци}и.
\begin{description}
	\item[Область определения функции] "--- все~возможные значения независимой переменной, при~которых существуют значения зависимой переменной.
\end{description}
Область определения функции записывается следующим образом.
\begin{equation}\label{eq:function-domain}
D(f)
\end{equation}
\begin{description}
	\item[Область значения функции] "--- все~возможные значения зависимой переменной.
\end{description}
Область значения функции записывается следующим образом.
\begin{equation}\label{eq:function-exists}
E(y)
\end{equation}

Функция является \emph{возрастающей} на~отрезке \textit{[a, b]}, если при~$x_1 < x_2 f(x_1) < f(x_2)$. Функция является \emph{убывающей} на~отрезке \textit{[a, b]}, если при~$x_1 < x_2 f(x_1) > f(x_2)$. Возрастающая либо убывающая функция называется \emph{монотонной функцией}.

\subsection{Определение числовой функции}
Рассмотрим некоторое числовое множество ${\textstyle X}$. Пусть для~элементов этого множества задано некоторое правило ${\textstyle f}$, согласно которому каждому элементу данного множества ставится в~соответствие некоторое число ${\textstyle X}$. Т.\,е.
\begin{equation}\label{eq:function-def-1}
X\ \text{"--- числовое множество},\ f\ \text{"--- правило},\ x \in X \rightarrow y.
\end{equation}
Тогда можно сказать, что~на~множестве ${\textstyle X}$ задана числовая функция ${\textstyle y=f(x)}$, где~${\textstyle x}$ "--- независимая переменная~(аргумент), ${\textstyle f}$ "--- правило, ${\textstyle y}$ "--- значение функции.

\subsection{Способы задания функции}
\textbf{Аналитический способ} состоит в задании функции одной или~несколькими формулами (например ${\textstyle y=f(x)}$) и~был рассмотрен ранее.
\textbf{Рекурсивный способ} состоит в~задании функции через саму себя, при~этом значения функции определяются через другие её~же значения. Такой способ задания функции используется в~задании множеств и~рядов.
\textbf{Графический способ} заключается в~проведении линии~(графика), у~которой абсциссы изображают значения аргумента, а~ординаты "--- соответствующие значения функции.
\textbf{Словесный способ} состоит в~задании функции естественным языком, т.\,е.~словами. При~этом необходимо задать входные и~выходные значения, а~также соответствие между ними.
\textbf{Табличный способ} заключается в~задании таблицы отдельных значений аргумента и~соответствующих им~значений функции. Такой способ задания функции применяется только в~том~случае, когда \emph{область определения функции} является дискретным конечным множеством.

\subsection{Свойства функций, исследование функций}
Исследование любой функции в~базовом варианте включает в~себя процессы установления таких значений как:
\begin{enumerate}
	\item \textbf{область определения функции} ${\textstyle D(f)}$, проще говоря, все~возможные значений ${\textstyle x}$, т.\,е.~для~того, чтобы ответить на~вопрос об~области определения функции необходимо ответить на~вопросы:
	\begin{itemize}
		\item  <<откуда и~докуда существует график функции по~оси~абсцисс?>>;
		\item <<какие ${\textstyle x}$ можно подставить в~данную формулу так, чтобы существовал её~смысл?>>, например очевидно, что~в~случае сущестования дроби в~выражении, её~знаменатель не~может равняться нулю, а~в~случае существования выражения под~корнем чётной степени, оно~не~может принимать отрицательные значения,
	\end{itemize}
	таким образом сочетание графического и~аналитического подходов позволяет задать область определения функции на~основе достаточно тривиальной логики;
	\item \textbf{область значения функции} ${\textstyle E(y)}$, проще говоря, все~возможные значений ${\textstyle y}$, т.\,е.~для~того, чтобы ответить на~вопрос об~области значения функции необходимо ответить на~вопрос: <<откуда и~докуда существует график функции по~оси~ординат?>>;
	\item \textbf{нули функции} "--- точки пересечения графика функции с~осью абсцисс;
	\item \textbf{области возрастания и~убывания функции} "--- координаты по~оси абсцисс, при~которых функция возрастает либо убывает;
	\item \textbf{промежутки знака постоянства} "--- диапазоны координат по~оси абсцисс, при~которых значение~${\textstyle y}$ постоянно больше либо меньше ${\textstyle 0}$ по~оси абсцисс, проще говоря "--- координаты по~оси абсцисс областей, в~которых график функции (т.\,е.~значения~${\textstyle y}$) находится выше этой оси;
	\item \textbf{чётность либо нечётность функции}: в~том случае, когда график функции симметричен относительно оси~ординат "--- функция \emph{чётная}, относительно начала координат "--- \emph{нечётная}, не~симметричен относительно ни~того, ни~другого "--- \emph{ни~чётная, ни~нечётная}, последние функции также называют \emph{не~обладающими свойством чётности}, аналитическая техника определения наличия свойства чётности и~её~значений также достаточно проста: необходимо заменить все~${\textstyle x}$ на~${\textstyle -x}$: если вид, а~следовательно и~значение, функции при~этом не~изменяется "--- функция чётная, изменяются на~противоположные по~знаку "--- нечётная, изменяются иным образом "--- ни~чётная, ни~нечётная, т.,е.~не~имеет свойства чётности, иными словами при~наличии свойства чётности при~смене знака перед~${\textstyle x}$ сама функция сохраняется с~тем~же либо противоположным знаком "--- она~имеет свойство чётности, в~случае изменения самой функции "--- не~имеет его;
	\item минимальное и~максимальное значение функции "--- минимальное и~максимальное возможные значения координат графика функции по~оси ординат.  
\end{enumerate}
Далее будут даны более строгие определения данных понятий.

\subsubsection{Ограниченные и~неограниченные функции}
Обозначим ${\textstyle X}$ некоторое множество чисел, входящих в~область определения  ${\textstyle D(f)}$ функции ${\textstyle y=f(x)}$.
\begin{description}
	\item[Функцию ${\textstyle y=f(x)}$ называют ограниченной сверху на~множестве ${\textstyle X}$] если существует такое число ${\textstyle \alpha}$, что~для~любого~${\textstyle x}$   из~множества~${\textstyle X}$ выполняется неравенство
	\begin{equation}\label{eq:function-1}
	f(x)\leq \alpha.
	\end{equation}
\end{description}
\begin{description}
	\item[Функцию ${\textstyle y=f(x)}$ называют ограниченной снизу на~множестве ${\textstyle X}$] если существует такое число ${\textstyle \beta}$, что~для~любого~${\textstyle x}$   из~множества~${\textstyle X}$ выполняется неравенство
	\begin{equation}\label{eq:function-2}
	f(x)\geq \beta.
	\end{equation}
\end{description}
\begin{description}
	\item[Функцию ${\textstyle y=f(x)}$ называют ограниченной на~множестве ${\textstyle X}$] если существуют такие числа ${\textstyle \alpha,\ \beta}$, что~для~любого~${\textstyle x}$  из~множества~${\textstyle X}$ выполняется неравенство
	\begin{equation}\label{eq:function-3}
	\beta \leq f(x)\leq \alpha.
	\end{equation}
\end{description}
\begin{description}
	\item[Функцию ${\textstyle y=f(x)}$ называют неограниченной сверху на~множестве ${\textstyle X}$] если для~любого числа ${\textstyle \alpha}$ существует такой~${\textstyle x}$  из~множества~${\textstyle X}$, для~которого выполняется неравенство
	\begin{equation}\label{eq:function-4}
	f(x) > \alpha.
	\end{equation}
\end{description}
\begin{description}
	\item[Функцию ${\textstyle y=f(x)}$ называют неограниченной сверху на~множестве ${\textstyle X}$] если для~любого числа ${\textstyle \beta}$ существует такой~${\textstyle x}$  из~множества~${\textstyle X}$, для~которого выполняется неравенство
	\begin{equation}\label{eq:function-5}
	f(x) < \beta.
	\end{equation}
\end{description}
\begin{description}
	\item[Функцию ${\textstyle y=f(x)}$ называют неограниченной на~множестве ${\textstyle X}$] если эта~функция или~не~ограничена сверху, или~не~ограничена снизу, или~не ограничена и~сверху, и~снизу.
\end{description}
\begin{description}
	\item[Наименьшее значение функции ${\textstyle y=f(x)}$] представляется собой такое значение ${\textstyle \beta}$, для~которого выполняется неравенство ${\textstyle f(x_0)=\beta \wedge f(x)\geq \beta}$, т.\,е.
	\begin{equation}\label{eq:function-10}
	\beta = y_{min}:
	\begin{cases}
	f(x_0)=\beta\\
	f(x)\geq \beta.
	\end{cases}
	\end{equation}
\end{description}
\begin{description}
	\item[Наибольшее значение функции ${\textstyle y=f(x)}$] представляется собой такое значение ${\textstyle \alpha}$, для~которого выполняется неравенство ${\textstyle f(x_0)=\alpha \wedge f(x)\leq \alpha}$, т.\,е.
	\begin{equation}\label{eq:function-11}
	\alpha = y_{max}:
	\begin{cases}
	f(x_0)=\alpha\\
	f(x)\leq \alpha.
	\end{cases}
	\end{equation}
\end{description}

\begin{description}
	\item[Минимум функции ${\textstyle y=f(x)}$] представляется собой такую точку~${\textstyle x_{min}}$, в~некоторой окрестности которой выполняется неравенство ${\textstyle f(x)>f(x_{min})}$, т.\,е.
	\begin{equation}\label{eq:function-12}
	x_{min}:f(x)>f(x_{min})
	\end{equation}
\end{description}

\begin{description}
	\item[Максимум функции ${\textstyle y=f(x)}$] представляется собой такую точку~${\textstyle x_{max}}$, в~некоторой окрестности которой выполняется неравенство ${\textstyle f(x)<f(x_{max})}$, т.\,е.
	\begin{equation}\label{eq:function-13}
	x_{max}:f(x)<f(x_{max})
	\end{equation}
\end{description}

\subsubsection{Монотонные и~строго монотонные функции}
\begin{description}
	\item[Функцию ${\textstyle y=f(x)}$ называют возрастающей на~множестве~${\textstyle y=f(x)}$] если для~любых чисел ${\textstyle x_1 \in X,\ x_2 \in X}$,удовлетворяющих неравенству ${\textstyle x_1 < x_2}$, выполняется неравенство ${\textstyle f(x_1) \leq f(x_2)}$, т.\,е.~меньшему значению аргумента соответствует такое~же либо меньшее значение функции.
	\begin{equation}\label{eq:function-6}
	y=f(x)\nearrow :x_1 < x_2 \Rightarrow f(x_1) \leq f(x_2)
	\end{equation}
\end{description}
Возрастающие функции также называют \textbf{неубывающими функциями}.
\begin{description}
	\item[Функцию ${\textstyle y=f(x)}$ называют убывающей на~множестве~${\textstyle y=f(x)}$] если для~любых чисел ${\textstyle x_1 \in X,\ x_2 \in X}$,удовлетворяющих неравенству ${\textstyle x_1 < x_2}$, выполняется неравенство ${\textstyle f(x_1) \geq f(x_2)}$, т.\,е.~меньшему значению аргумента соответствует такое~же либо большее значение функции.
	\begin{equation}\label{eq:function-7}
	y=f(x)\searrow :x_1 < x_2 \Rightarrow f(x_1) \geq f(x_2)
	\end{equation}
\end{description}
Убывающие функции также называют \textbf{невозрастающими функциями}.
\begin{description}
	\item[Функцию ${\textstyle y=f(x)}$ называют строго возрастающей на~множестве~${\textstyle y=f(x)}$] если для~любых чисел ${\textstyle x_1 \in X,\ x_2 \in X}$,удовлетворяющих неравенству ${\textstyle x_1 < x_2}$, выполняется неравенство ${\textstyle f(x_1) < f(x_2)}$, т.\,е.~меньшему значению аргумента соответствует меньшее значение функции.
	\begin{equation}\label{eq:function-8}
	y=f(x)\uparrow :x_1 < x_2 \Rightarrow f(x_1) < f(x_2)
	\end{equation}
\end{description}
\begin{description}
	\item[Функцию ${\textstyle y=f(x)}$ называют строго убывающей на~множестве~${\textstyle y=f(x)}$] если для~любых чисел ${\textstyle x_1 \in X,\ x_2 \in X}$,удовлетворяющих неравенству ${\textstyle x_1 < x_2}$, выполняется неравенство ${\textstyle f(x_1) > f(x_2)}$, т.\,е.~меньшему значению аргумента соответствует большее значение функции.
	\begin{equation}\label{eq:function-9}
	y=f(x)\downarrow :x_1 < x_2 \Rightarrow f(x_1) > f(x_2)
	\end{equation}
\end{description}
\begin{description}
	\item[Монотонная функция]"--- возрастающая либо убывающая функция.
\end{description}
\begin{description}
	\item[Стого монотонная функция]"--- строго возрастающая либо строго убывающая функция.
\end{description}
На~рисунке~\ref{fig:even-odd-functions} показаны примеры различных типов функций.
\begin{Thexmpl}
	$\begin{aligned}
	y&=x^2\downarrow:x\in (-\infty;o],\ y=x^2\uparrow:x\in [0;\infty)\\
	y&=-x^2\uparrow:x\in (-\infty;o],\ y=-x^2\downarrow:x\in [0;\infty)\\
	y&=x\uparrow:x\in (-\infty;\infty)\\
	y&=\arctg \uparrow:x\in (-\infty;\infty)\\
	\end{aligned}$
\end{Thexmpl}

\subsubsection{Чётные и~нечётные функции}
\begin{description}
	\item[Чётная функция] "--- функция вида ${\textstyle y=f(x)}$, определённая на~множестве~${\textstyle X}$, при~которой для~любых чисел ${\textstyle x}$ и~${\textstyle -x}$, принадлежащих множеству~${\textstyle X}$, выполняется неравенство ${\textstyle f(-x)=f(x)}$. На~математическом языке данная запись выглядит следующим образом:
	\begin{equation}\label{eq:even-functions}
	f(-x)=f(x):\ x\in X,\ D(f)\ \text{симметричное множество} \Rightarrow \text{чётная функция.}
	\end{equation}
\end{description}
\begin{description}
	\item[Нечётная функция] "--- функция вида ${\textstyle y=f(x)}$, определённая на~множестве~${\textstyle X}$, при~которой для~любых чисел ${\textstyle x}$ и~${\textstyle -x}$, принадлежащих множеству~${\textstyle X}$, выполняется неравенство ${\textstyle f(-x)=-f(x)}$. На~математическом языке данная запись выглядит следующим образом:
	\begin{equation}\label{eq:odd-functions}
	f(-x)=-f(x):\ x\in X,\ D(f)\ \text{симметричное множество} \Rightarrow \text{нечётная функция.}
	\end{equation}
\end{description}
Для~того, чтобы говорить о~чётности либо нечётности функции ${\textstyle y=f(x)}$ необходимо, чтобы она~была определена как~в~точке~${\textstyle x}$ так~и~в~точке~${\textstyle -x}$, т.\,е.~область определения функции (${\textstyle D(f)}$) должна являться \emph{симметричным множеством}.
\begin{description}
	\item[Ни~чётной, ни~нечётная функция] "--- функция вида ${\textstyle y=f(x)}$, определённая на~множестве~${\textstyle X}$, при~которой для~любых чисел ${\textstyle x}$ и~${\textstyle -x}$, принадлежащих множеству~${\textstyle X}$, не~выполняется ни~одно их~двух вышеприведённых неравенств. На~математическом языке данная запись выглядит следующим образом:
	\begin{equation}\label{eq:not-even-not-odd-functions}
	f(-x)\neq f(x)\wedge f(-x)\neq-f(x):\ x\in X \Rightarrow \text{ни~чётная, ни~нечётная функция.}
	\end{equation}
\end{description}
На~рисунке~\ref{fig:even-odd-functions} показаны примеры различных типов функций.
\begin{Thexmpl}
	$\begin{aligned}
	y&=x^2\ \text{"--- чётная функция}\\
	y&=-x^2\ \text{"--- чётная функция}\\
	y&=x\ \text{"--- нечётная функция}\\
	y&=\arctg(x)\ \text{"--- нечётная функция}\\
	y&=a^x:\ a \in (0;\infty)| a \neq 1\  \text{"--- ни~чётная, ни~нечётная функция}\\
	y&=\log_a{x}:\ a \in (0;\infty)| a \neq 1\  \text{"--- ни~чётная, ни~нечётная функция}\\
	\end{aligned}$
\end{Thexmpl}
\begin{theorem}
	Любую функцию ${\textstyle y=f(x)}$, определённую на~симметричном относительно точки~${\textstyle x=0}$ множестве~${\textstyle X}$,  можно представить в~виде суммы чётной и~нечётной функций.
\end{theorem}

\subsubsection{Периодические и~непериодические функции, период функции}
\begin{description}
	\item[Период функции~${\textstyle y=f(x)}$] "--- такое число ${\textstyle T}$, не~равное нулю, если для~любого числа~${\textstyle x \in D(f)}$ числа ${\textstyle x+T,\ x-T}$ также принадлежат области определения функции~(${\textstyle D(f)}$) и~справедливы равенства~(${\textstyle f(x-T)=f(x),\ f(x-T)=f(x)}$). На~математическом языке данная запись выглядит следующим образом:
	\begin{equation}\label{eq:period-of-function}
	f(x+T)=f(x) \wedge f(x-T)=f(x): x,\ x+T,\ x-T \in D(f),\ T\neq 0 \Rightarrow T\text{"--- период функции.}
	\end{equation}
\end{description}
\begin{description}
\item[Периодическая функция] "--- функция, имеющая \emph{период}.
\end{description}
\begin{description}
	\item[Непериодическая функция] "--- функция, не~имеющая \emph{периода}.
\end{description}
Если число~${\textstyle T}$ является периодом некоторой функции, то~и~число ${\textstyle kT}$, где~${\textstyle k}$ "--- любое целое число, отличное от~нуля, также является периодом этой функции.

Функции ${\textstyle y=\sin x,\ y=\cos x}$ являются периодическими функциями с~периодом~${\textstyle 2\pi]}$, функции~${\textstyle y=\tg x,\ y=\ctg x}$   являются периодическими функциями с~периодом~${\textstyle \pi]}$. Показательные, логарифмические и~степенные функции являются \emph{непериодическими функциями}.

\subsubsection{График функции. Свойства графиков чётных, нечётных и~периодических функций}
\begin{description}
	\item[График функции~${\textstyle y=f(x)}$] "--- множество всех точек, координаты которых имеют вид
	\begin{equation}\label{eq:func-graph-coord}
	(x;f(x)):x\in D(f).
	\end{equation}.
\end{description}
График чётной функции симметричен относительно оси ординат, график нечётной функции симметричен относительно начала координат. График периодической функции не~изменяется при~сдвиге вдоль оси абсцисс на~период вправо или~влево. Примеры графиков функций приведены на~рисунке~\ref{fig:even-odd-functions}.

\begin{figure}[ht]
	\centering % Центрируем картинку
	\includegraphics[width=\textwidth]{even-odd-functions.pdf}
	\caption{Примеры типов функций}\label{fig:even-odd-functions}
\end{figure}

\subsection{Частные примеры функций}
\subsubsection{Линейная функция и~её~график, взаимное расположение графиков линейных функций}

Общая формула линейной функции следует из~\ref{eq:two-unknown} и~представляет собой выражение
\begin{equation}\label{eq:linear-func-1}
ax+by+c=0: b\neq 0.
\end{equation}
Отсюда следует
\begin{equation}\label{eq:linear-func-2}
	\begin{aligned}
		by &= -ax-c\\
		y &= -\frac{a}{b}x - \frac{c}{b}\\
		k &= -\frac{a}{b}, m &= -\frac{c}{b} \Rightarrow \\
		y &= kx+m. 
	\end{aligned}
\end{equation}
Последнее выражение $y=kx+m$ называется \emph{линейной функцией}, в~которой $x$ "--- независимая переменная~(аргумент), $y$ "--- зависимая переменная (зачение функции). Графиком линейной функции является прямая.

Для~определения взаимного расположения графиков двух линейных функций
\begin{equation*}\label{eq:linear-func-3}
\begin{aligned}
y&=k_{1}x+m_1\\
y&=k_{2}x+m_2\\
\end{aligned}
\end{equation*}
следует использовать правила: в~случае, когда $k_1=k_2, m_1 \neq m_2$ "--- графики функций параллельны; в~случае, когда $k_1=k_2, m_1=m_2$ "--- графики функций совпадают; в~случае, когда $k_1 \neq k_2, m_1 \neq m_2$ "--- графики функций имеют пересечение, являющееся единственным. 

Для~поиска точки пересечения графиков можно использовать следующую простую логику: если выполняется условие пересечения графиков функций, следовательно существует такая единственная точка, в~которой $y_1=y_2$, следовательно $=k_{1}x+m_1 = k_{2}x+m_2$. Далее путём решения простого линейного уравнения можно найти $x$.

\begin{Thexmpl}\label{ex:two-linear-1}
	Дано:
	
	$\begin{aligned}
	y_{1}&=8x - 3\\
	y_{2}&=3x + 2\\
	\end{aligned}$
	
	Найти точку пересечения этих функций. Поскольку коэффициенты перед $x_1, x_2$ разные, следовательно графики функций имеют пересечение, а~значит существует такая единственная точка~в~которой выполняется условие~$y_1=y_2$. Соответственно в~этой точке $8x - 3 = 3x + 2$. Тогда $5x=5$. Из~этого следует, что~$x=1$. Подставив значение $x$ в~любую из~функций получим $y=5$.
	
	Ответ: графики функций пересекаются в~точке~(1, 5).
\end{Thexmpl}

\subsubsection{Функция вида ${\textstyle y=x^m:m \in \mathbb{Z}}$, её~свойства и~график}
Рассмотрим несколько вариантов такой функции. Для~начала рассмотрим степени с~положительными показателями.

Функция ${\textstyle y=x^{2n}: n \in \mathbb{N}}$ имеет следующие свойства:
\begin{enumerate}
	\item ${\textstyle D(f)=\mathbb{R}}$;
	\item данная функция является \emph{чётной};
	\item функция является \emph{ограниченной снизу} и~\emph{не~ограниченной сверху};
	\item ${\textstyle y_{min}=0}$;
	\item ${\textstyle y_{max}\not \exists}$;
	\item ${\textstyle y\downarrow:x\in (-\infty;0],\ y\uparrow:x\in [0;+\infty)}$;
	\item функция непрерывна:
	\item ${\textstyle E(y)=[0;+\infty)}$.
\end{enumerate}
Данная функция имеет вид параболы и~показана на~рисунке~\ref{fig:y=x^m} красным цветом.

Функция ${\textstyle y=x^{n+1}: n \in \mathbb{N}}$ имеет следующие свойства:
\begin{enumerate}
	\item ${\textstyle D(f)=\mathbb{R}}$;
	\item данная функция является \emph{нечётной};
	\item функция является \emph{не~ограниченной снизу} и~\emph{не~ограниченной сверху};
	\item ${\textstyle y_{min}\not \exists}$;
	\item ${\textstyle y_{max}\not \exists}$;
	\item ${\textstyle y\uparrow:x\in (-\infty;+\infty)}$;
	\item функция непрерывна:
	\item ${\textstyle E(y)=(-\infty;+\infty)}$.
\end{enumerate}
Данная показана на~рисунке~\ref{fig:y=x^m} синим цветом.

Рассмотрим степени с~отрицательными показателями.

Функция ${\textstyle y=x^{-2n} \equiv y=x^{\frac{1}{2n}} : n \in \mathbb{N}}$ имеет следующие свойства:
\begin{enumerate}
	\item ${\textstyle D(f)=\{x \in \mathbb{R} | x \neq 0\}}$;
	\item данная функция является \emph{чётной};
	\item функция является \emph{ограниченной снизу} и~\emph{не~ограниченной сверху};
	\item ${\textstyle y_{min}\not \exists}$;
	\item ${\textstyle y_{max}\not \exists}$;
	\item ${\textstyle y\downarrow:x\in (-\infty;0),\ y\uparrow:x\in (0;+\infty)}$;
	\item функция непрерывна: ${\textstyle x \neq 0}$;
	\item ${\textstyle E(y)=(0;+\infty)}$.
\end{enumerate}
Данная функция имеет вид схожий с~гиперболой, но~не~является ей~и~показана на~рисунке~\ref{fig:y=x^m} оранжевым цветом.

Функция ${\textstyle y=x^{-2n-1} \equiv y=x^{\frac{1}{2n-1}} : n \in \mathbb{N}}$ имеет следующие свойства:
\begin{enumerate}
	\item ${\textstyle D(f)=\{x \in \mathbb{R} | x \neq 0\}}$;
	\item данная функция является \emph{нечётной};
	\item функция является \emph{не~ограниченной снизу} и~\emph{не~ограниченной сверху};
	\item ${\textstyle y_{min}\not \exists}$;
	\item ${\textstyle y_{max}\not \exists}$;
	\item ${\textstyle y\downarrow:x\in (-\infty;+\infty)}$;
	\item функция непрерывна: ${\textstyle x \neq 0}$;
	\item ${\textstyle E(y)=(-\infty;0),(0;+\infty)}$.
\end{enumerate}
Данная функция имеет вид гиперболы и~показана на~рисунке~\ref{fig:y=x^m} сиреневым цветом.

\begin{figure}[ht]
	\centering % Центрируем картинку
	\includegraphics[width=\textwidth]{y=x-m.pdf}
	\caption{Примеры типов функций}\label{fig:y=x^m}
\end{figure}

\subsubsection{Функция вида ${\textstyle y=x^{\frac{1}{3}}}$, её~свойства и~график}
Функция ${\textstyle y=x^{\frac{1}{3}}}$ имеет следующие свойства:
\begin{enumerate}
	\item ${\textstyle D(f)=x \in \mathbb{R}}$;
	\item данная функция является \emph{нечётной};
	\item функция является \emph{не~ограниченной снизу} и~\emph{не~ограниченной сверху};
	\item ${\textstyle y_{min}\not \exists}$;
	\item ${\textstyle y_{max}\not \exists}$;
	\item ${\textstyle y\uparrow:x\in (-\infty;+\infty)}$;
	\item функция непрерывна;
	\item ${\textstyle E(y)=(-\infty;+\infty)}$.
\end{enumerate}
Данная функция показана на~рисунке~\ref{fig:y=x^m} зелёным цветом.

\subsection{Исследование функций}
\subsection{Пределы функций}
\subsection{Асимптоты графиков функций}

\section{Последовательности}
\subsection{Определение, способы задания и~свойства}
\subsubsection{Определение}
\begin{description}
	\item[Числовая последовательность] "--- последовательность элементов множества~${\textstyle X}$ вида ${\textstyle (x_n)_{n=1}^{\infty}}$ при~условии, что~${\textstyle X \in \mathbb{R} \vee X \in \mathbb{C}}$.
	\item[Подпоследовательность последовательности ${\textstyle (x_n)}$] "--- последовательность ${\textstyle (x_{n_{k}})}$, где~${\textstyle (n_{k})}$ "--- возрастающая последовательность элементов множества натуральных чисел. Иными словами, подпоследовательность получается из~последовательности удалением конечного или~счётного числа элементов.
\end{description}
Числовые последовательности являются одним из основных объектов рассмотрения в математическом анализе.

Рассмотрим некоторую функцию ${\textstyle y=f(x)}$. Для~задания самой функции необходимо задать:
	\begin{itemize}
		\item правило ${\textstyle f}$, в~соответствии с~которым каждому~${\textstyle x}$ ставится в~соответствие единственный~${\textstyle y}$;
		\item область определения функции: ${\textstyle x \in D(y)}$.
	\end{itemize}
Рассмотрим другую функцию ${\textstyle y=f(т):\ n \in \mathbb{N}}$. Такая функция задаёт числовую последовательность. Рассмотрим пример конкретной функции данного вида:
\begin{equation*}\label{eq:numerical-sequence-example}
y_n=\frac{n}{n+1}\Rightarrow y(1)=\frac{1}{2}, y(2)=\frac{2}{3}\ldots \text{общепринятой формой записи является:} y_1=\frac{1}{2}, y_2=\frac{2}{3}\ldots
\end{equation*}

\subsubsection{Способы задания последовательностей}
Существует несколько способов задания последовательности:
\begin{itemize}
	\item аналитический;
	\item рекуррентный;
	\item словесный.
	\end{itemize}
\textbf{Аналитический способ} задания последовательности реализуется в~виде формулы, содержащей порядковый номер члена последовательности, например ${\textstyle y_n=2n-1}$ "--- данная формула задаёт последовательность нечётных чисел.

\textbf{Задание последовательности с~помощью рекуррентного соотношения} реализуется путём задания каждого члена последовательности через предыдущий с~указание начального значения, например~${\textstyle y_n=2(y_{n-1}): y_1=1}$. C~помощью данного способа можно задать, например \emph{последовательность Фибоначчи} путём использования формулы
\begin{equation}\label{eq:fibonacci}
y_n=y_{n-1}+y_{n-2}: y_1=1, y_2=1,
\end{equation}
а~также факториал
\begin{equation}\label{eq:factorial}
y_n=n \times y_{n-1}=n!.
\end{equation}

\subsubsection{Свойства}
Универсальные свойства числовых последовательностей:

Свойствами числовых последовательностей являются:
\begin{enumerate}
	\item ограниченность;
	\item монотонность
\end{enumerate}
\textbf{Ограниченность числовой последовательности}. Если существует такое число ${\textstyle M}$, которое больше чем~все~члены числовой последовательности либо равно наибольшему из~них, то~такая последовательность является \textbf{ограниченной сверху}. Если существует такое число ${\textstyle m}$, которое меньше чем~все~члены числовой последовательности либо равно наименьшему из~них, то~такая последовательность является \textbf{ограниченной снизу}.
\begin{equation}\label{eq:numerical-sequence-limits}
m\leq y_n \leq M \Rightarrow \text{последовательность ограничена снизу и~сверху}
\end{equation}
\textbf{Монотонность числовой последовательности}. 

\subsection{Понятие множества}\label{multiple:definition}
Под~\emph{множеством} понимают совокупность, класс или~собрание объектов безразлично какой природы. Согласно определению основоположника теории множеств \href{https://ru.wikipedia.org/wiki/Кантор,_Георг}{Г.\,Кантора}~\cite{Wiki:Kantor}, множество "--- это~собрание предметов одинаковых или~различных между собой, мыслимое как единое целое. Собрание предметов рассматривается как~один предмет. Не~следует понимать множество как~совокупность действительно существующих предметов, принадлежность предметов одному множества не~требует от~них~сосуществования во~времени и~пространстве. В~логике множество понимается как~абстрактный объект, в~котором каждый предмет рассматривается с~точки зрения признаков, по~которым данный предмет принадлежит данному множеству. В~множестве предметы становятся неразличимыми друг от~друга по~признакам и~их~только по~именам.

Объект, принадлежащий данному множеству, называется его~\textbf{элементом}. Множество обозначается заглавными латинскими буквами ${\textstyle{A},\ B,\ C\ldots}$. Элементы, входящие в~множество, обозначаются строчными латинскими буквами и~заключаются в~фигурные скобки: ${\textstyle a,\ b,\ c \in A}$. Обратная запись: ${\textstyle A=\{a,\ b,\ c\}}$.

Множество, содержащее конечное число элементов, называется \textbf{конечным}, а~бесконечное число элементов "--- \textbf{бесконечным}. Примером \emph{бесконечного множества} является множество натуральных чисел ${\textstyle \mathbb{N}}$. Примером \emph{конечного множества} является, например множество многоквартирных жилых домов на~территории Санкт-Петербурга.

Множество ${\textstyle A}$ является подмножеством множества ${\textstyle B}$, если все~элементы множества ${\textstyle A}$ также являются элементами множества ${\textstyle B}$. На~математическом языке данная запись выглядит следующим образом: ${\textstyle A\subset B}$.

\begin{description}
	\item[Пересечение множеств ${\textstyle A,\ B}$] "--- такое множество, которое содержит все~элементы, входящие и~в~${\textstyle A}$, и~в~${\textstyle B}$, т.\,е.
	\begin{equation}\label{eq:sets-intercept}
	A\cap B = \{x\arrowvert x \in A \wedge x \in B\}
	\end{equation}
\end{description}
Например, ${\textstyle \mathbb{N} \cap \mathbb{Z} = \mathbb{N}}$.
\begin{description}
	\item[Объединение множеств ${\textstyle A,\ B}$] "--- такое множество, которое содержит все~элементы, входящие хотя~бы в~одно из~множеств: ${\textstyle A}$, или~${\textstyle B}$, т.\,е.
	\begin{equation}\label{eq:sets-join}
	A\cup B = \{x|x \in A \vee x \in B\}
	\end{equation}
\end{description}
Например, ${\textstyle \mathbb{N} \cup \mathbb{Z} = \mathbb{Z}}$.

Два множества называются \textbf{равными}, если содержат одинаковые элементы ${(\text{A}={2,4,8}=\text{B}={2,2,4,8})}$.

Элементами множества могут быть другие множества ${\text{A}={{2,3},{4,5}}}$. При~этом ${\text{A}={{2,3},{4,5}}\neq \text{B}= {2,3,4,5}}$.

\emph{Множество}, не~содержащее ни~одного элемента, называется \textbf{пустым множеством}.

\emph{Пустое множество} и~само множество~$А$ называются \textbf{несобственными} подмножествами множества~$А$, все~остальные подмножества "--- \textbf{собственными}.

\emph{Множество} называется \textbf{заданным}, если перечислены все~входящие в~него элементы либо определены признаки, по~которым данный объект можно отнести к~данному множеству:
\begin{description}
	\item[$A=\{x, P(x)\}$] "--- $x$ "--- элементы множества, $P(x)$ "--- свойства элементов данного множества.
	\item[$B=\{x, x=2n, n \in \mathbb{N}\}$] "--- множество чётных чисел.
\end{description}

Если \emph{множество} задано своим свойством, то~нельзя заранее сказать, будут~ли в~нём элементы.

Если множество $A$ содержит $n$ элементов, количество его~подмножеств составляет
\begin{equation}\label{n-submultitudes}
|M_{a}| = 2^n,
\end{equation}
где $n$ "--- число элементов множества.

\begin{Thexmpl}
	Дано:
	
	$\text{A}={\{a, b, c, d, e, f ,g\}}$
	
	$\text{B}={\{f, g, v, w, x, y, z\}}$
	
	$\text{C}={\{a, b\}}$
	
	Тогда:
	
	$C \subset A$
	
	$A \bigcup B = {\{a, b, c, d, e, f, g, v, w, x, y, z\}}$
	
	$A \bigcap B = {\{f, g\}}$
	
	$A \backslash B = {\{a, b, c, d, e\}}$
	
	$B \backslash A = {\{v, w, x, y, z\}}$
	
	$A \triangle B = {\{a, b, c, d, e, v, w, x, y, z\}}$
\end{Thexmpl}


\begin{Thexmpl}
	Дано: $\text{A}={{a, b, c}},\ n = 3$
	
	Вычислить число подмножеств $A$.
	
	$2^3 = 8$
	
	$M_A = \{\text{\AE{}}, a, b, c, \{ab\}, \{ac\}, \{bc\}, \{a,b,c\}\}$
	
	$M_A = 8$
\end{Thexmpl}

\begin{theorem}
	Пустое множество является подмножеством любого множества.
\end{theorem}
End.\cite{Studopedia:mnozhestvo}
\subsection{Понятие отображения множеств}
Большую роль в~математике имеет установление связей между двумя множествами $X$ и~$Y$, связанное с~рассмотрением пар объектов, образованных из~элементов первого множества и~соответствующих им~элементов второго множества. Особое значение при~этом имеет \emph{отображение множеств}.

Пусть $X$ и~$Y$ "--- произвольные множества. Отображением множества $X$ на~множество $Y$ называется $\forall$ правило $f$, по~которому каждому элементу множества $X$ сопоставляется вполне определённый~(единственный) элемент множества~$Y$. Тот~факт, что~$f$ есть отображение $X$ в~$Y$, кратко записывают в~виде: $f:X->Y$.

Таким образом, для~того чтобы задать отображение~$f$ множества~$Х$ в~множество $Y$, надо каждому элементу $x\in X$ поставить в~соответствие один и~только один элемент~$y \in Y$. Если при~этом элементу~$х \in Х$ сопоставлен элемент~$y \in Y$, то~$y$ называют \textbf{образом элемента}~$х$, а~$х$ "--- \textbf{прообразом элемента}~у при~отображении~f, что~записывается в~виде $f(x)=y$.

Из определения отображения~$f$ следует, что~у~каждого элемента~$x$ из~$Х$ есть только один \emph{образ} в~$Y$, однако для~элемента $y$ из~$Y$может быть несколько \emph{прообразов}. Множество всех прообразов элемента~$y$ из~$Y$ называется его~\emph{полным прообразом} и~обозначается через $f^{-1}(y)$. Таким образом, $f^{-1}(y)={x \in X | f(x) \in y}$.

Если множества $Х$ и~$Y$ числовые, то~$f$ называется \textbf{функцией}.

На~первый взгляд может показаться, что~всё~вышеизложенное не~имеет отношения к~оценочной деятельности и~не~имеет практического применения в~ней. Однако данное мнение является заблуждением. Оценщики очень часто сталкиваются с~понятием \emph{функции}. Например, замена исходных значений признака на~его квадрат либо логарифм являются типичными примерами отображения множеств. Так, например в~\cite{Laskin:lognorm} утверждается, что~использование логарифмов значений цен позволяет избежать систематического завышения результатов оценки. В~таблицах~\ref{tab:function-square}, \ref{tab:function-log} показаны примеры отображения при~которых $f$ представляет собой операцию возведения числа в~квадрат и~операцию логарифмирования соответственно.

\begin{table}[ht]
	\caption{Отображение множества при $f=^2$} \label{tab:function-square}
	\centering% центрируем таблицу
	\begin{tabular}{ccc} 
		\hline
		x  & f & y 
		\\ \hline \hline
		-5 & $^2$ & 25 \\ 
		-2 & $^2$ & 4 \\ 
		-1 & $^2$ & 1 \\ 
		0 & $^2$ & 0 \\ 
		1 & $^2$ & 1 \\ 
		2 & $^2$ & 4 \\ 
		5 & $^2$ & 25 \\
		\hline	
	\end{tabular}
\end{table}

\begin{table}[ht]
	\caption{Отображение множества при~$f=log$} \label{tab:function-log}
	\centering% центрируем таблицу
	\begin{tabular}{ccc} 
		\hline
		x  & f & y 
		\\ \hline \hline
		1 & log & 0.000 \\ 
		2 & log & 0.693 \\ 
		3 & log & 1.099 \\ 
		5 & log & 1.609 \\ 
		8 & log & 2.079 \\ 
		13 & log & 2.565 \\ 
		21 & log & 3.045 \\ 
		\hline	
	\end{tabular}
\end{table}

\subsection{Примеры последовательностей}
Последовательностью называется отображение множества натуральных числе во~множество вещественных чисел, т.\,е.~$\mathbb{N} -> \mathbb{R}$. Наиболее простым и~очевидным способом задания последовательности явным образом путём перечисления её~членов, например $x_1, x_2, x_3, x_4,\ldots, x_n$. Можно также использовать задание последовательности с~помощью формул либо словесных описаний. Например, последовательность квадратов натуральных чисел можно задать с~помощью формулы
\begin{equation}\label{eq:conseq-squares}
x_n=x^2.
\end{equation}
Последовательность десятичных знаков числа $\pi$ может быть задана формулой
\begin{equation}\label{eq:conseq-pi}
x_n=\frac{[10^{n-1}\pi]}{10^{n-1}}
\end{equation}
В~ряде случаев задание последовательности может быть выполнено графически. Например для~задания последовательности $1, 0, -1, 0, 1, 0, -1, 0, 1,\ldots$ можно использовать функцию
\begin{equation}\label{eq:sinus}
x_n=\sin \frac{\pi n}{2}.
\end{equation}
Графически такое отображение показано на~рисунке~\ref{fig:sinus}, на~котором заглавными латинскими буквами показаны элементы последовательности.
\begin{figure}[ht]
	\centering % Центрируем картинку
	\includegraphics[width=\textwidth]{sinus.pdf}
	\caption{Графическое отображение последовательности $1, 0, -1, 0, 1, 0, -1, 0, 1,\ldots$ )}\label{fig:sinus}
\end{figure}
\subsection{Пределы последовательностей}
Рассмотрим для~примера уже~знакомую ранее последовательность $1, 0, -1, 0, 1, 0, -1, 0, 1,\ldots$, а~затем другую: $1, 1.5, 1,41666, 1.41421566862\ldots, 1.4142135623\ldots$, задаваемую рекуррентно с~помощью формулы
\begin{equation}\label{eq:recurr}
y_{n+1}=\frac{1}{2}(y_n+\frac{2}{y_n}), y_1=1.
\end{equation}
Как~видно, данные последовательности имеют принципиальное отличие: члены первой последовательности чередуются, второй "--- приближаются к~некоторому числу~(квадратному корню из~числа 2).
Предел последовательности имеет форму записи
\begin{equation}\label{eq:limit1}
\lim_{n\to\infty}x_n=l.
\end{equation}
Данную запись можно описать как:
\begin{itemize}
	\item $l$ есть предел последовательности $x_n$ либо
	\item последовательность $x_n$ сходится к~$n$, либо
	\item последовательность $x_n$ стремится к~$n$.
\end{itemize}
Из~этого следует, что~для~любого интервала, содержащего точку $l$, вне~его~находится лишь конечное число последовательности. При~этом неважно, является данный интервал произвольным либо симметричным относительно этой точки, поскольку любой интервал может быть уменьшен либо увеличен для~симметричного. Таким образом во~всех случаях можно вести речь о~симметричных интервалах. Из~этого следует:
\begin{itemize}
	\item при~любом $\epsilon > 0$ вне~интервала $(l-\epsilon, l+\epsilon)$ находится лишь конечное число членов последовательности;
	\item для~любого $\epsilon > 0$ найдётся такой номер $N$, что~$|x_n-l| < \epsilon$, при~всех $n \geq N$;
	\item с~помощью кванторов, описанных в~\ref{mathan-gloss-symbols}, два~вышеуказанных утверждения можно записать кратко: $\forall \epsilon > 0 \quad \exists \quad N \qquad \forall n \geq N \qquad |x_n-l|<\epsilon$.
\end{itemize}
Рассмотрим пример. Возьмём последовательность
\begin{equation}\label{eq:limits2}
\lim_{n\to\infty}\frac{n^2}{n^2+1}=1
\end{equation}
и~покажем, что~она стремится к~$1$. Для~этого оценим модуль разности и~найти такое $n$, при~котором он~будет меньше~1.
\begin{equation}\label{eq:limits3}
|\frac{n^2}{n^2+1}-1|=\frac{1}{n^2+1}<\frac{1}{n^2}<\epsilon \quad \text{при}~n \geq [\epsilon^{(-\frac{1}{2})}+1].
\end{equation}


\section{Логарифмы}
\section{Функции и~непрерывность}
\section{Производные}
\section{Интегралы}




\nocite{CSC:intro-in-matan}

\printbibliography[title=Источники информации]

\end{document}
