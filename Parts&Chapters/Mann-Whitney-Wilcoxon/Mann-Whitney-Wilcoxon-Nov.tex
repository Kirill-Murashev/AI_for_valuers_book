\documentclass[]{scrartcl}
\input{standard_preamble.tex}
%opening
\title{Практические аспекты применения критерия Манна-Уитни-Уилкоксона в~оценочной деятельности}
\author{К.\,А.\,Мурашев}

\begin{document}

\maketitle

\begin{abstract}
В~своей практике оценщики часто сталкиваются с~необходимостью учёта различий количественных характеристик объектов. В~частности, одной из~стандартных задач является установление признаков, влияющих на~стоимость~(т.\,н.~ценообразующих факторов) и~их~отделение от~признаков, влияние которых на~стоимость отсутствует либо не~может быть установлено. В~практике оценки широкое распространение получил субъективный отбор признаков, учитываемых при~определении стоимости. При~этом конкретные количественные показатели влияния этих признаков на~стоимость зачастую берутся из~т.\,н.~<<справочников>>. Не~отказывая такому подходу в~быстроте и~невысокой стоимости его~реализации, нельзя не~признать, что~только данные, непосредственно наблюдаемые на~открытом рынке, являются надёжной основой суждения о~стоимости. Приоритет таких данных над~прочими, в~частности, полученными путём опроса экспертов, закреплён, в~том~числе в~\href{https://www.rics.org/uk/upholding-professional-standards/sector-standards/valuation/red-book/red-book-global/}{Стандартах оценки~RICS}~\cite{RVGS-2022}, \href{https://www.rics.org/uk/upholding-professional-standards/sector-standards/valuation/red-book/international-valuation-standards/}{Международных стандартах оценки~2022}~\cite{IVS-2022}, а~также \href{https://normativ.kontur.ru/document?moduleId=1&documentId=326168#l0}{МСФО~13~<<Оценка справеливой стоимости}~\cite{MSFO-13}. Поэтому можно говорить о~том, что~математические методы анализа данных, полученных на~открытом рынке, являются наиболее надёжным средством интерпретации рыночной информации, применяемой при~исследованиях рынка и~предсказании стоимости конкретных объектов. В~данном материале будут рассмотрены основные теоретические вопросы, касающиеся теста Манна-Уитни-Уилкоксона~(далее U-тест), а~также проведён пошаговый разбор применения данного теста к~конкретным данным. Материал содержит строки кода, необходимые для~проведения U-теста с~использованием языков программирования Python и~R, а~также приложение в~виде электронной таблицы, содержащей формулы для~проведения данного теста и~полностью готовой для~её~применения на~любых иных данных.

Данный материал и~все~приложения к~нему распространяются на~условиях лицензии \href{https://creativecommons.org/licenses/by-sa/4.0/}{cc-by-sa-4.0}~\cite{cc-by-sa-4.0}.
\end{abstract}

\section{Технические данные}
Данный материал, а~также приложения к~нему доступны по~постоянной ссылке.

\section{Предмет исследования}
В~случае работы с~рыночными данными перед оценщиком часто встаёт задача проверки гипотезы о~существенности влияния того или~иного качественного признака, измеренного в~количественной или~порядковой шкале, на~стоимость. Аналогичная задача возникает у~аналитиков рынка недвижимости, специалистов компаний-застройщиков, риелторов. При~этом зачастую отсутствует возможность сбора больших массивов данных, позволяющих применить широкий спектр методов машинного обучения. В~ряде случаев оценщики осознанно сужают область сбора данных до~узкого сегмента рынка, в~результате чего в~их~распоряжении оказываются лишь сверхмалые выборки объёмом менее тридцати наблюдений. При~этом, ценовые данные чаще всего имеют распределение отличное от~нормального. В~данном случае рациональным решением является применение U-теста. Сформулируем задачу:
\begin{itemize}
	\item предположим, что~у~нас~существуют две~выборки удельных цен коммерческих помещений, часть из~которых имеет отдельный вход, часть "--- нет;
	\item необходимо установить: оказывает~ли наличие отдельного входа существенное влияние на~удельную стоимость недвижимости данного типа.
\end{itemize}
Данные, используемые в~настоящей работе, являются вымышленными и~служат для~учебных целей. 
 
\section{Основные сведения о~тесте}

\section{Практическая реализация}

\section{Выводы}

\printbibliography[title=Источники информации]

\end{document}
