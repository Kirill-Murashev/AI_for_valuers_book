\documentclass[]{scrartcl}
% Лицензия
% Apache License Version 2.0, January 2004
% http://www.apache.org/licenses/
% Copyright [2020] [Kirill A. Murashev]
% Licensed under the Apache License, Version 2.0 (the "License"); you may not use this file except in compliance with the License. You may obtain a copy of the License at
% http://www.apache.org/licenses/LICENSE-2.0
% Unless required by applicable law or agreed to in writing, software
% distributed under the License is distributed on an "AS IS" BASIS,
% WITHOUT WARRANTIES OR CONDITIONS OF ANY KIND, either express or implied.
% See the License for the specific language governing permissions and limitations under the License.


%%% additional multilang settings
\usepackage{cmap}					% search in PDF
\usepackage{mathtext} 				% other letters in equations
\usepackage{fontspec}
\defaultfontfeatures{Renderer=Basic,Ligatures={TeX}}
\setmainfont{CMU Serif}
\setsansfont{CMU Sans Serif}
\setmonofont{CMU Typewriter Text}
\usepackage[english]{babel}
%\usepackage[T1,T2A]{fontenc}			% encoding
%\usepackage[lutf8]{luainputenc}			% encoding
%\usepackage[english,russian]{babel}	% localization and hyphenation
\usepackage{indentfirst}            % first indentfirst
\usepackage{misccorr}               % aadons for babel
\frenchspacing                      % french style of spaces

%\usepackage{beton} % font
%\usepackage{concrete} % font
%%% Additional work with math
\usepackage{amsmath,amsfonts,amssymb,amsthm,mathtools} % AMS
\usepackage{icomma} % "smart" comma: $0,2$ --- number, $0, 2$ --- list

%% equation numbering
%\mathtoolsset{showonlyrefs=true} % Показывать номера только у тех формул, на которые есть \eqref{} в тексте.
%\usepackage{leqno} % Нумерация формул слева

%% Перенос знаков в формулах (по Львовскому)
\newcommand*{\hm}[1]{#1\nobreak\discretionary{}
	{\hbox{$\mathsurround=0pt #1$}}{}}

%%% Работа с картинками
\usepackage{graphicx}  % Для вставки рисунков
\graphicspath{{Images/}}  % папки с картинками
\setlength\fboxsep{3pt} % Отступ рамки \fbox{} от рисунка
\setlength\fboxrule{1pt} % Толщина линий рамки \fbox{}
\usepackage{wrapfig} % Обтекание рисунков текстом

%%% Работа с таблицами
\usepackage{array, tabularx, tabulary, booktabs, xtab} % Дополнительная работа с таблицами
\usepackage{longtable}  % Длинные таблицы
\usepackage{multirow} % Слияние строк в таблице

%%% Теоремы
\theoremstyle{plain} % Это стиль по умолчанию, его можно не переопределять.
\newtheorem{theorem}{Теорема}[section]
\newtheorem{proposition}[theorem]{Утверждение}
\newtheorem{lemma}[theorem]{Лемма}

\theoremstyle{definition} % "Определение"
\newtheorem{corollary}{Следствие}[theorem]
\newtheorem{problem}{Задача}[section]

\theoremstyle{remark} % "Примечание"
\newtheorem*{nonum}{Решение}

%%% Программирование
\usepackage{etoolbox} % логические операторы

\usepackage{lastpage} % Узнать, сколько всего страниц в документе.

\usepackage{keyval}

\usepackage{totcount} % Узнать, сколько всего объектов в документе.

%\usepackage{xcolor-solarized}

%%% Страница
%\usepackage{extsizes} % Возможность сделать 14-й шрифт
%\usepackage{geometry} % Простой способ задавать поля
%	\geometry{top=25mm}
%	\geometry{bottom=35mm}
%	\geometry{left=35mm}
%	\geometry{right=20mm}
%

%\usepackage{fancyhdr} % Колонтитулы

%	\pagestyle{fancy}
%\renewcommand{\headrulewidth}{0pt}  % Толщина линейки, отчеркивающей верхний колонтитул
%\fancyhf{}
%\lhead{Часть \thepart}
%\chead{Глава \thechapter}
%\rhead{Раздел \thesection}
%\lfoot{version 0.251}
%\cfoot{\today} % По умолчанию здесь номер страницы
%\rfoot{\thepage/\ref{LastPage}}
%\pagestyle{fancy}

%\usepackage{setspace} % Интерлиньяж
%\onehalfspacing % Интерлиньяж 1.5
%\doublespacing % Интерлиньяж 2
%\singlespacing % Интерлиньяж 1

\usepackage{soul} % Модификаторы начертания

\usepackage[usenames,dvipsnames,svgnames,table,rgb]{xcolor} % Подключение пакета для задания цвета

%\definecolor{Backcolor}{HTML}{042029} % Задание цвета для фона
%\definecolor{Textcolor}{HTML}{819090} % Задание цвета для текста
%\pagecolor{Backcolor}                 % Подключение тёмной
%\color{Textcolor}                     % темы

\usepackage{csquotes} % Ещё инструменты для ссылок

\usepackage[backend=biber,bibencoding=utf8,sorting=ynt,maxcitenames=5,sortupper=true,date=iso]{biblatex} % подключение пакета для работы с автоматизированной библиографией

%\usepackage[style=authoryear,maxcitenames=2,backend=biber,sorting=nty]{biblatex}

%\renewcommand\bibname{Источники информации} % Переопределение названия для библиографии

\usepackage{multicol} % Несколько колонок

\usepackage{microtype}              %<-- added for better inter word spacing

\usepackage{tabularx}

\usepackage{tikz} % Работа с графикой
\usepackage{pgfplots}
\usepackage{pgfplotstable}

\usepackage{eqlist}

\usepackage{desclist} % Дополнительное окружение для списка Глоссария

\setcounter{tocdepth}{8} % Глубина оглавления

% подавление висячих строк
\clubpenalty=400 % Разрешение = 300, абсолютный запрет = 10000
\widowpenalty=400 % Увеличиваем эти числа до тех пор, пока не начнёт увеличиваться количество страниц.

% Выбор между разрежением и переполнением
\tolerance=500 % max=10000, default=200

\looseness=-1 % иногда можно удлинять страницу на одну строку.

\hfuzz=2.5pt % иногда можно вылезти за край строки на 2.5 pt.

\usepackage{calc} % Вычисления

\usepackage{scrlayer-scrpage} % Стиль страницы

\usepackage{lineno} % нумерация строк

%\pagestyle{scrpage}

%\usepackage{concrete}

\usepackage{booktabs}

\usepackage[owncaptions]{vhistory} % Log of versions

\usepackage{progressbar} % Формирование линейки, показывающей прогресс в работе

\usepackage{epigraph} % работа с эпиграфами

\usepackage {listings}
\lstloadlanguages{[Latex]Tex, bash, R, Python, SQL}
\lstset{extendedchars=true , % включаем не латиницу
frame=tb, % рамка сверху и снизу
commentstyle=\itshape , % шрифт для комментариев
stringstyle =\ttfamily % шрифт для строк
%keywordstyle=\color{blue}
}

%\usepackage{titling} %дополнительная настройка титульного листа

\setcounter{secnumdepth}{8} % Установка глубины нумерации заголовков

% Работа с гиперрсылками, подключается последним
\usepackage{hyperref}       % Подключение пакета для работы с гиперссылками
\hypersetup{				% Гиперссылки
	unicode=true,           % русские буквы в раздела PDF
	pdftitle={Искусственный интеллект в~оценке стоимости},   % Заголовок
	pdfauthor={К.\,А.~Мурашев},      % Автор
	pdfsubject={Системы поддержки принятия решений, основанные на искусственном интеллекте},      % Тема
	pdfcreator={К.\,А.~Мурашев}, % Создатель
	pdfproducer={К.\,А.~Мурашев}, % Производитель
	pdfkeywords={Искусственный интеллект, машинное обучение, математические методы, оценочная деятельность, цифровая экономика, Data Science, анализ данных} % Ключевые слова
	colorlinks=true,       	% false: ссылки в рамках; true: цветные ссылки
	linkcolor=red,          % внутренние ссылки
	citecolor=green,        % на библиографию
	filecolor=magenta,      % на файлы
	urlcolor=blue           % на URL
}

\usepackage{pgfplots} 
\pgfplotsset{compat=1.15}
\usepackage{mathrsfs}
\usetikzlibrary{arrows}
%\usepackage{url}

%\usepackage{totpages}

%\usepackage[strings]{underscore}

%\author{К.\,А.~Мурашев\thanks {\href{kirill.murashev@tutanota.de}{kirill.murashev@tutanota.de}, \href{https://t.me/Maas\_88}{https://t.me/Maas\_88}, \href{https://www.facebook.com/murashev.kirill}{https://www.facebook.com/murashev.kirill}}}
%\title{\Large Современные системы поддержки принятия решений оценщиками, основанные на~применении методов машинного обучения: практическое руководство по~применению языка программирования R в~повседневной практике оценщика}
%\date{\today}

%\normalsize

% Макрос для рисунков, обтекаемых текстом
\newcommand*{\EpsWrapD}[7]{%
	\begin{wrapfigure}[#5]{#3}{#2 \textwidth} % #3=l,r,L,R
		\begin{center} \sffamily
			\includegraphics*[width= #2 \textwidth ]{#1} % 1-имя файла и метка заодно,
			% 2-ширина рисунка (доля от ширины страницы)
			\vspace{-#7mm} % #7: сократить расстояние между подписью снизу и рисунком
			\caption{\label{fig:#1}#4} % #4 - подпись под рисунком
			\vspace{-#6pt}
		\end{center}% #6: сократить расстояние между подписью снизу и текстом после таблицы 
	\end{wrapfigure}}
%
% макрос для создания таблицы, обтекаемой текстом
\newcommand*{\TableBE}[5]{
	\begin{table}[#1] %\captionabove
		\vspace*{-#5mm}
		\centering \sffamily \caption{\label{tab:#2}#3} \begin{tabular}{#4} \toprule }
		
		\newcommand*{\TableEN}[3]{
			\bottomrule \end{tabular}
		\vspace{-#2mm} \small \begin{flushleft} #1 \end{flushleft}
		\vspace{-#3mm}
\end{table}}


\addbibresource{/home/kaarlahti/TresoritDrive/Methodics/My/AI_for_valuers/Book/AI_for_valuers_book/Basic_principles.bib}
\addbibresource{/home/kaarlahti/TresoritDrive/Methodics/My/AI_for_valuers/Book/AI_for_valuers_book/LaTeX.bib}
\addbibresource{/home/kaarlahti/TresoritDrive/Methodics/My/AI_for_valuers/Book/AI_for_valuers_book/Mathstat.bib}
\addbibresource{/home/kaarlahti/TresoritDrive/Methodics/My/AI_for_valuers/Book/AI_for_valuers_book/Murashev.bib}
\addbibresource{/home/kaarlahti/TresoritDrive/Methodics/My/AI_for_valuers/Book/AI_for_valuers_book/Python.bib}
\addbibresource{/home/kaarlahti/TresoritDrive/Methodics/My/AI_for_valuers/Book/AI_for_valuers_book/R.bib}
\addbibresource{/home/kaarlahti/TresoritDrive/Methodics/My/AI_for_valuers/Book/AI_for_valuers_book/RussianLaws.bib}
\addbibresource{/home/kaarlahti/TresoritDrive/Methodics/My/AI_for_valuers/Book/AI_for_valuers_book/Sci&Tech.bib}
\addbibresource{/home/kaarlahti/TresoritDrive/Methodics/My/AI_for_valuers/Book/AI_for_valuers_book/Valuation.bib}
\addbibresource{/home/kaarlahti/TresoritDrive/Methodics/My/AI_for_valuers/Book/AI_for_valuers_book/ValuationStandards.bib}
\addbibresource{/home/kaarlahti/TresoritDrive/Methodics/My/AI_for_valuers/Book/AI_for_valuers_book/ZHZL.bib}

\pagestyle{headings} 
\markright{Искусственный интеллект в~оценке стоимости}
\usepackage{pgfplots}
\pgfplotsset{compat=1.15}
\usepackage{mathrsfs}
\usetikzlibrary{arrows}

%\usepackage{polyglossia}

%\usepackage{minted}

\newtheorem{Thexmpl}[theorem]{Пример}

\usepackage[inkscapearea=page]{svg}
\usepackage{adjustbox}

\DeclareMathOperator{\rank}{rank}
\makeatletter
\newenvironment{sqcases}{%
	\matrix@check\sqcases\env@sqcases
}{%
	\endarray\right.%
}
\def\env@sqcases{%
	\let\@ifnextchar\new@ifnextchar
	\left\lbrack
	\def\arraystretch{1.2}%
	\array{@{}l@{\quad}l@{}}%
}
\makeatother

\DeclareMathOperator{\arcsec}{arcsec}
\DeclareMathOperator{\arccot}{arccot}
\DeclareMathOperator{\arccsc}{arccsc}
\DeclareMathOperator{\sgn}{sgn}


\title{Short Introduction to~the~differences between Frequentist~and~Bayesian approaches to~probability in~valuation}
\author{K.\,A.\,Murashev}

\begin{document}


\maketitle

\begin{abstract}
	When you say "mathematical statistics" and~"statistical methods," most people who~are~not~mathematicians or~machine learning specialists think of~concepts like "null hypothesis," "significance level," "criterion statistics," and~many other things that are~taught in~college-level statistics courses. It~may~be a~revelation to~some people: all~this is~just part of~what the~modern science of~probability is~all~about. Traditionally, the~studied methods belong to~a~single variant: statistics, based on~the~\href{https://en.wikipedia.org/wiki/Frequentist_probability}{Frequentist approach to~probability}~\cite{Wiki:Freq-probability-eng}. This is~primarily due to~ the~fact that this area of~statistics has~been around for~more than three centuries and~is~historically the~first known to~mankind. At~the same time, this approach makes it possible to create descriptive and predictive models that provide good interpretation of the results, understandable even to non-specialists. Действительно, когда британские учёные совершают очередной научный прорыв, доказав, что~количество пластинок с~записями Вагнера находится в~тесной связи с~объёмом грудной клетки их~владельца, любой читатель может сформировать общее представление о~вопросе.
	
	Однако наука не~стоит на~месте. Примерно с~1950-х годов XX~века распространение начал получать и~другой подход к~понятию вероятности "--- т.\,н.~\href{https://ru.wikipedia.org/wiki/Байесовская_вероятность}{байесовский подход к~вероятности}~\cite{Wiki:Bayes-prob}. Данный подход стал особенно популярным в~последние 20--30 лет, и~на~сегодняшний день можно говорить о~том, что~именно он~определяет развитие такой области как~машинное обучение, являющейся предметом интереса всего исследования применения методов искусственного интеллекта в~оценочной деятельности. Забегая вперёд, можно сказать, что~эти~два подхода не~являются взаимоисключающими, используют один математический аппарат и~при~определённых условиях один из~них~перерождается в~другой.
	
	При~этом, следует признать, что~даже частотный подход к~вероятности, равно как~и~математические методы в~целом, пока что~не~получили широкого распространения в~практике российской оценки. Данный материал призван сформировать представление о~сути каждого из~подходов к~вероятности, помочь оценщикам получить некоторое представление о~возможных путях развития методов оценки, а~также о~дальнейших перспективах развития отрасли и~своём месте в~ней. 
\end{abstract}

\section{Историческая справка}
Невозможно сказать, когда именно появились первые представления о~вероятности и~статистических свойствах явлений, объектов и~процессов. Известно лишь то, что~появились они~вследствие любви человека к~двум вещам:
\begin{itemize}
	\item деньгам;
	\item играм.
\end{itemize}
Первые известные работы, в~которых были рассмотрены вопросы вероятности, описывают различные ситуации, связанные с~азартными играми. Первым серьёзным общетеоретическим научным выводом о~вероятности можно считать высказанное \href{https://ru.wikipedia.org/wiki/Кардано,_Джероламо}{Д.\,Кардано} утверждение о~том, что~\textit{<<при~небольшом числе игр реальное количество исследуемых событий может  сильно отличаться от~теоретического, но~чем~больше игр в~серии, тем~меньше доля этого различия>>}~\cite{Wiki:Kardano-Dzherolamo}. Известный междисциплинарный исследователь \href{https://ru.wikipedia.org/wiki/Галилей,_Галилео}{Галилео Галилей} в~1718~г. опубликовал свой трактат <<О~выходе очков при~игре в~кости», в~котором детально описал многие вопросы вероятности. В~своей главной работе <<Диалог о~двух главнейших системах мира, птолемеевой и~коперниковой>> Галилей также указал на~возможность оценки погрешности астрономических и~иных измерений, причём заявил, что~малые ошибки измерения вероятнее, чем~большие, отклонения в~обе стороны равновероятны, а~средний результат должен быть близок к~истинному значению измеряемой величины. Эти~качественные рассуждения стали первым в~истории предсказанием \href{http://www.machinelearning.ru/wiki/index.php?title=Нормальное_распределение}{нормального распределения}~\cite{Distrib:Normal} ошибок. 

Дальнейшее развитие науки о~вероятности произошло благодаря переписке Блеза Паскаля и~Пьера Ферма, вызванной поставленной шевалье де~Мере т.\,н. <<задачей об~очках>>, также посвящённой вопросам азартных игр. в~ходе данной переписки зародился ещё~один раздел математики "--- комбинаторика. Обсуждение подобных вопросов вдохновило Христиана Гюйгенса, предложившего собственное решение задачи, а~также опубликовавшего работу <<О~расчётах в~азартных играх>>. При~этом Гюйгенса стал первым исследователем, обратившим внимание на~серьёзную научную основу вероятностных задач. Предисловие к~вышеуказанной работе гласило <<Я~полагаю, что~при~внимательном изучении предмета читатель заметит, что~имеет дело не~только с~игрой, но~что здесь закладываются основы очень интересной и~глубокой теории>>. Гюйгенс впервые ввёл понятие математического ожидания, т.\,е. теоретического среднего арифметического случайной величины, дал окончательное решение задачи о~разделе ставок при~досрочном прекращении игры.

История изучения вопросов вероятности в~XVIII веке связана прежде всего с~именами Бернулли, Муавра, Эйлера и~Лапласа, разработавших первые законы распределения. Примерно в~это~же время фокус внимания переключился с~игр на~более практические области: демографию, экономику, страхование. В~1763 году Томас Байес опубликовал свою знаменитую формулу, ставшую спустя примерно 200~лет основной нового подхода к~вероятности. Тем~не~менее, несмотря на~его~открытие, практическое распространение в~тот период получал лишь частотный подход к~вероятности.

XIX век дал существенное число работ по~статистике, перечислить даже малую часть которых не~представляется возможным в~рамках данного фрагмента. Следует отметить, что~итоговое формирование частотной математической статистики произошло в~1930-е годы вследствие работ К.\,Пирсона, Р.\,Фишера и А.\,Н.\,Колмогорова. 

Байесовский подход к~вероятности начал получать распространение лишь после публикации работ Ф.\,Рэмси <<Истина и~вероятность>>~(1926), Б.\,де\,Финетти <<Предвидение: его логические законы, его субъективные источники>>~(1937). Однако его~широкое признание произошло только после выхода в~свет работы Л.\,Сэвиджа <<Основания статистики>>~(1954). Суть субъективной интерпретации вероятности можно выразить словами Рэмси: <<Степень уверенности (belief) "--- это~её~каузальное свойство (causal property of~it), которое мы~можем приблизительно сформулировать как~степень, в~какой мы готовы действовать в~соответствии с~нашей уверенностью>>.


\section{Современное состояние}
Как~уже было сказано ранее, на~сегодняшний день существуют два~практических подхода к~понятию вероятности: классический частотный и~байесовский. Можно сказать, что~ключевое различие между ними заключается в~интерпретации понятия случайности. Частотный подход гласит о~том, что~случайность является следствием объективной неопределённости, которая может быть уменьшена только путём проведения серии экспериментов. В~байесовском подходе утверждается, что~неопределённость является лишь следствием субъективного незнания. Таким образом можно утверждать о~том, что~разные субъекты могут обладать разным незнанием. Современная наука придерживается позиции о~том, что~именно байесовская трактовка даёт более корректное понимание окружающего мира, соответствующее текущим представлениям о~фундаментальных процессах. Дело в~том, что~по-настоящему неопределёнными являются всего два известных человечеству процесса:
	\begin{itemize}
		\item квантово-механические процессы;
		\item радиоактивный распад ядер атомов.\footnote{По~крайней мере по~современным представлениям в~области физики.}
	\end{itemize}

Все~остальные явления и~процессы на~самом деле являются детерминированными, т.\,е.~происходящими в~строгом соответствии с~некоторыми законами. И~лишь субъективное незнание законов, в~соответствии с~которыми они~протекают, либо отдельных параметров, являющихся необходимыми для~вычисления состояния объекта либо процесса вызывает неопределённость. Классическим примером применения вероятности является эксперимент с~подбрасыванием монеты. Несложно догадаться, что~незнание исхода подбрасывания не~является следствием склонности самой монетки либо неопределённости тех~законов механики, в~соответствии с~которыми происходит её~движение. На~самом деле, нет~никакой сложности в~том, чтобы точно предсказать исход броска. Достаточно знать массу монеты, её~геометрические свойства, упругость, расположение центра массы, параметры силы, приложенной к~монете в~момент броска, свойства среды (например кинематическая вязкость), в~которой будет происходить её~движение, высоту между точкой броска и~поверхностью, свойства поверхности (например та~же упругость) и~ряд иных параметров. Знание этих начальных совершенно конкретных значений, а~также умение применять полностью детерминированные законы классической механики позволят абсолютно точно предсказать исход каждого броска. И~лишь отсутствие значений этих параметров, а~также определённая вычислительная сложность позволяют говорить о~непредсказуемости поведения монеты. Т.\,е.~налицо лишь субъективная неопределённость, вызванная личным незнанием наблюдателя, но~отнюдь не~объективная, вызванная склонностью монеты либо недетерминированностью законов её~движения.

Применительно к~оценке стоимости можно сказать, что~её~неопределённость также является лишь субъективной. С~точки зрения оценщика, определяющего стоимость, например запасов строительного материала "--- песка, действительно имеет место неопределённость, которую он~стремится уменьшить путём применения ряда методов. Однако с~точки зрения многих других субъектов, например собственника предприятия-правообладателя запасов, части его~сотрудников, осуществляющих закупочную деятельность, поставщиков, производителей и~перевозчиков строительных материалов в~данном вопросе нет~никакой неопределённости: им~хорошо известна  стоимость строительного песка на~любой стадии его~существования от~залежей недр до~строительной площадки. Кроме того, стоимость строительного песка на~каждой стадии его~перемещения между этими состояниями прирастает вполне определённым образом "--- путём привнесения в~неё~добавленной стоимости, образуемой субъектами того или~иного этапа. Таким образом, можно говорить о~том, что~стоимость конкретного песка на~конкретной строительной площадке вполне детерминирована вкладами всех участников производственно-логистической цепочки. И~лишь субъективное незнание стороннего наблюдателя, вызванное в~том~числе множественностью возможных цепочек вызывает у~него наличие субъективной неопределённости относительно стоимости песка. При~этом вполне вероятна ситуация, когда один оценщик, может обладать знаниями, касающимися, например стоимости песка на~всех карьерах территории, к~которой относится исследуемая им~строительная площадка, но~при~этом он~также обладает незнанием всех~остальных слагаемых стоимости. В~этом случае можно говорить о~том, что~он является калиброванным экспертом в~части стоимости <<франко завод>>. Второй оценщик может ничего не~знать о~стоимости песка на~карьерах, но~при~этом отлично знать стоимость логистики сыпучих грузов, являясь в~таком случае калиброванным экспертом по~вопросам затрат на~перевозку. Третий оценщик не~обладает никакими знаниями о~стоимости самого песка и~его~перевозки, однако при~этом в~отличие от~двух предыдущих обладает знаниями в~вопросах налогообложения, вследствие чего он~может делать правильные выводы о~факте включения либо невключения НДС в~стоимость. Таким образом, каждый из~них обладает субъективным незнанием части слагаемых стоимости. Таким образхом в~условиях того, что~стоимость оцениваемого песка вполне детерминирована сама по~себе, для~оценщиков она~носит случайный характер исключительно в~силу их~незнания части её~слагаемых. При~этом каждый из~оценщиков обладает разным незнанием, что~подчёркивает субъективность неопределённости.
Вследствие сказанного выше, можно сделать вывод о~том, что, с~точки зрения байесовского подхода, плотность распределения случайной величины "--- это~способ закодировать субъективное незнание. В~случае отсутствия незнания функция плотности выродилась~бы в~\href{https://ru.wikipedia.org/wiki/Дельта-функцию}{дельта-функцию}~\cite{Wiki:delta-function},\footnote{$\delta$-функция, $\delta$-функция Дирака, дираковская дельта, единичная импульсная функция.} являющуюся в~таком случае пределом \href{https://ru.wikipedia.org/wiki/Гауссова_функция}{функции Гаусса}~\cite{Wiki:Gauss-function} с~дисперсией, стремящейся к~нулю. Поскольку чаще всего субъективное незнание  всё~же имеет место, вместо дельта-функции  мы~можем наблюдать именно функцию плотности распределения.

Следующее различие между подходами заключается в~их~отношении к~величинам параметров. В~частотном подходе существует чёткое разделение на~случайные и~неслучайные параметры. Типичной задачей является оценка тех~или~иных параметров генеральной совокупности, представляющей собой набор случайных величин на~основе детерминированных параметров выборки, например: среднее, мода, дисперсия и~т.\,д. Последние представляют собой конкретные значения, в~которых уже~нет~никакой случайности. В~байесовском~же подходе все~величины являются случайными. Поскольку в~этом случае предпосылкой является отсутствие объективной неопределённости, сами параметры выборки также могут быть описаны не~конкретными значениями, а~плотностями распределения, в~т.\,ч.~априорными. В~случае известности значения какого-либо параметра вместо плотности распределения будет использоваться $\delta$-функция. Таким образом, вероятностный аппарат распространяется на~все~без~исключения параметры. В~случае точного измерения значения одного из~которых происходит коллапс функции его~распределения.

Ещё~одно отличие между подходами заключается в~используемом каждом из~них методе вывода (методе статистического оценивания). В~частотном подходе используется \href{http://www.machinelearning.ru/wiki/index.php?title=Метод_наибольшего_правдоподобия}{Метод максимального правдоподобия}.~\cite{MLRU:max-likehood-method} Теоретические обоснования говорят о~том, что~оценки максимального правдоподобия являются оптимальными т.\,е.:
	\begin{itemize}
		\item несмещёнными;
		\item состоятельными;
		\item асимптотически эффективными;
		\item асимптотически нормальными.
	\end{itemize}
Детальное рассмотрение данного метода не~входит в~периметр настоящего фрагмента. Скажем только, что~оптимальность оценок максимального правдоподобия основана на~предпосылке о~том, что~объём исследуемой выборки стремится к~бесконечности. Однако при~решении практических задач это~почти всегда не~так. Например в~случае проведения оценки оценщик мог~бы найти всего два предложения о~продаже карьерного песка интересующей его~марки. В~силу каких-либо обстоятельств указанные в~объявлениях цены могли составить 10 и~15 рублей за~1~куб.~м (например оценщик мог~бы найти объявления о~самовывозе ненужного песка с~личных участков). С~точки зрения \emph{метода наибольшего правдоподобия} оценкой параметра являлось~бы среднее значение 12.5~руб. за~1~куб.~м. Однако наличие априорных знаний о~характерном значении цены (200-300 руб. за~1~куб.~м) не~позволило~бы сделать вывод о~корректности и~применимости полученной оценки. Однако в~случае анализа нескольких сотен или~тысяч объявлений уже~любое конкретное значение полученной стоимости могло~бы считаться корректным. В~байесовском подходе методом вывода служит сама Теорема Байеса:
\begin{equation}\label{Bayes-Theorem}
P(A|B) = \frac{P(B|A)P(A)}{P(B)}.
\end{equation}

Следующее различие заключается в~получаемых оценках. В~частотном подходе получаемая оценка является конкретным числом либо интервалом. В~байесовском "--- апостериорным распределением. Общая схема при~этом выглядит следующим образом. К~изначальном существовавшему субъективному незнанию, закодированному в~функцию плотности априорного распределения добавляются знания о~различных параметрах, прямо или~косвенно связанных с~интересующим нас~параметром. Вследствие этого посредством применения Теоремы Байеса априорное распределение трансформируется в~апостериорное без~потери информации. В~случае необходимости в~дальнейшем возможно получение точечной оценки, например моды либо матожидания. При~этом неизбежно возникает потеря части информации. С~точки зрения оценки стоимости можно предложить следующую схему: оценщик имеет некоторое изначальное представление о~стоимости объекта. Далее путём внесения новой информации (например о~площади объекта оценки) это~априорное знание трансформируется в~апостериорное, которое является априорным для~следующего этапа анализа (на~котором, например вносится информация о~техническом состоянии объекта). Таким образом, применение байесовского подхода позволяет объединять множество простых моделей в~более сложные.

Последнее различие заключается в~возможностях применения в~зависимости от~числа наблюдений. Частотный подход в~общем случае требует значительное их~число. Байесовский работает при~любом. Даже тогда, когда число наблюдений равно нулю. В~этом случае апостериорное распределение будет совпадать с~априорным. Поступление данных о~хотя~бы одном наблюдении позволяет скорректировать распределение в~правильную сторону. При~этом, в~случае, когда $n\longrightarrow \infty$ байесовский подход переходит в~частотный, что~является доказательством высказанного ранее утверждения о~непротиворечивости этих подходов относительно друг друга. С~математической точки зрения это~означает, что~при~очень большом числе наблюдений апостериорное распределение начинает переходить в~дельта-функцию в~точке максимального правдоподобия. Общая формула связи между подходами на~примере подбрасывания монеты выглядит следующим образом:
\begin{equation}\label{Freq-Baye-link}
\frac{m+1}{n+2},
\end{equation}
где~m "--- число выпадений <<орла>>,
n "--- общее число бросков.
Очевидно, что~при $n = 0$ число выпадений <<орла>> также будет равно нулю, что~приводит к~тому, что~апостериорная вероятность равна априорной и~составляет $\frac{1}{2}$. При~этом, в~случае если $n \longrightarrow \infty$, то~и~$m \longrightarrow \infty$, что~означает, что: 
\begin{equation}\label{Freq-Baye-link-inf}
\frac{m}{n}.
\end{equation}
В выражении \ref{Freq-Baye-link-inf} несложно узнать функцию правдоподобия. В~промежуточных случаях, когда число наблюдений больше нуля, но~существенно меньше бесконечности байесовский подход позволяет находить компромисс между априорными знаниями и~новой вносимой информацией.	

\begin{table}[ht]
\caption{Сводные сведения об~основных различиях между подходами к~вероятности} \label{tab:Freq-Baye-diff}
\centering% центрируем таблицу
\begin{tabularx}{\textwidth}{X|XX} 
	\hline
	\multicolumn{1}{c|}{} & \multicolumn{1}{c}{Частотный~(\foreignlanguage{english}{Frequentist})} & \multicolumn{1}{c}{Байесовский~(\foreignlanguage{english}{Bayesian})}  \\ 
	\hline\hline
Интерпретация случайности
  & Объективная неопределённость
  & Субъективное незнание
\\ \hline
Величины
  & Чёткое деление на~\emph{случайные} и~\emph{неслучайные}   
  & Все~величины случайны
  \\ \hline
Метод вывода
  & Метод максимального правдоподобия    
  & Теорема Байеса
  \\ \hline
Оценки
	& Точечные либо интервальные  
	& Апостериорное распределение     
	\\ \hline
Применимость
	& $n\longrightarrow \infty $ либо, как~минимум, $n \gg 1$
	&  \forall~n
	\\ \hline
\end{tabularx}
\end{table}


\nocite{CSC:intro-in-prob-lang-ML}
\printbibliography[title=Источники информации]
\end{document}
