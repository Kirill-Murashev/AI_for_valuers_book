\documentclass[]{scrartcl}

\input{standard_preamble.tex}
%opening
\title{Практическое применение критерия Манна-Уитни-Уилкоксона в~оценочной деятельности}

\author{K.\,A.~Мурашев}

\begin{document}

\maketitle
	
\begin{abstract}
	В~своей практике оценщики часто сталкиваются с~необходимостью учёта различий количественных характеристик объектов. В~частности, одной из~стандартных задач является установление признаков, влияющих на~стоимость~(т.\,н.~ценообразующих факторов) и~их~отделение от~признаков, влияние которых на~стоимость отсутствует либо не~может быть установлено. В~практике оценки широкое распространение получил субъективный отбор признаков, учитываемых при~определении стоимости. При~этом конкретные количественные показатели влияния этих признаков на~стоимость зачастую берутся из~т.\,н.~<<справочников>>. Не~отказывая такому подходу в~быстроте и~невысокой стоимости его~реализации, нельзя не~признать, что~только данные, непосредственно наблюдаемые на~открытом рынке, являются надёжной основой суждения о~стоимости. Приоритет таких данных над~прочими, в~частности, полученными путём опроса экспертов, закреплён, в~том~числе в~\href{https://www.rics.org/uk/upholding-professional-standards/sector-standards/valuation/red-book/red-book-global/}{Стандартах оценки~RICS}~\cite{RVGS-2022}, \href{https://www.rics.org/uk/upholding-professional-standards/sector-standards/valuation/red-book/international-valuation-standards/}{Международных стандартах оценки~2022}~\cite{IVS-2022}, а~также \href{https://normativ.kontur.ru/document?moduleId=1&documentId=326168#l0}{МСФО~13~<<Оценка справедливой стоимости}~\cite{MSFO-13}. Поэтому можно говорить о~том, что~математические методы анализа данных, полученных на~открытом рынке, являются наиболее надёжным средством интерпретации рыночной информации, применяемой при~исследованиях рынка и~предсказании стоимости конкретных объектов. В~данном материале будут рассмотрены основные теоретические вопросы, касающиеся теста Манна-Уитни-Уилкоксона~(далее U-тест), а~также проведён пошаговый разбор применения данного теста к~конкретным данным. Материал содержит строки кода, необходимые для~проведения U-теста с~использованием языков программирования Python и~R, а~также приложение в~виде электронной таблицы, содержащей формулы для~проведения данного теста и~полностью готовой для~её~применения на~любых иных данных.
	Данный материал и~все~приложения к~нему распространяются на~условиях лицензии \href{https://creativecommons.org/licenses/by-sa/4.0/}{cc-by-sa-4.0}~\cite{cc-by-sa-4.0}.
\end{abstract}
%
\section{Технические данные}
Данный материал, а~также приложения к~нему доступны по~\href{https://github.com/Kirill-Murashev/AI_for_valuers_book/tree/main/Parts-Chapters/Mann-Whitney-Wilcoxon}{постоянной ссылке}~\cite{Murashev:u-test}. Исходный код данной работы был~создан с~использованием языка~\href{https://www.ctan.org/}{\TeX}~\cite{TeX:site} c~набором макрорасширений~\href{https://www.latex-project.org/}{\LaTeX}~\cite{LaTeX:site}, дистрибутива~\href{https://www.tug.org/texlive/}{TeXLive}~\cite{TeXLive:site} и~редактора~\href{https://www.texstudio.org/}{TeXstudio}~\cite{TeXstudio:site}. Расчёт в~форме электронной таблицы был выполнен с~помощью \href{https://www.libreoffice.org/discover/calc/}{LibreOffice Calc}~\cite{LO:Calc} (Version: 7.3.3.2, Ubuntu package version: 1:7.3.3~rc2-0ubuntu0.20.04.1~lo1 Calc: threaded). Расчёт на~языке~\href{https://www.r-project.org/}{R}~\cite{R_language} (version 4.2.0 (2022-04-22) -- "Vigorous Calisthenics") был выполнен c~использованием IDE~\href{URL}{RStudio} (RStudio 2022.02.2+485 "Prairie Trillium" Release (8acbd38b0d4ca3c86c570cf4112a8180c48cc6fb, 2022-04-19) for Ubuntu Bionic Mozilla/5.0 (X11; Linux x86\_64) AppleWebKit/537.36 (KHTML, like Gecko) QtWebEngine/5.12.8 Chrome/69.0.3497.128 Safari/537.36)~\cite{RStudio:official_site}. Расчёт на~языке \href{https://www.python.org/}{Python}~\cite{Python:site} был выполнен с~использованием среды разработки \href{https://jupyter.org}{Jupyter Lab}~\cite{Jupyter:site}.
%
\section{Предмет исследования}
В~случае работы с~рыночными данными перед оценщиком часто встаёт задача проверки гипотезы о~существенности влияния того или~иного признака, измеренного в~количественной или~порядковой шкале, на~стоимость. Аналогичная задача возникает у~аналитиков рынка недвижимости, специалистов компаний-застройщиков, риелторов. При~этом зачастую отсутствует возможность сбора больших массивов данных, позволяющих применить широкий спектр методов машинного обучения. В~ряде случаев оценщики осознанно сужают область сбора данных до~узкого сегмента рынка, в~результате чего в~их~распоряжении оказываются лишь сверхмалые выборки объёмом менее тридцати наблюдений. При~этом, ценовые данные чаще всего имеют распределение отличное от~нормального. В~данном случае рациональным решением является применение U-теста. Сформулируем задачу:
\begin{itemize}
	\item предположим, что~у~нас~существуют две~выборки удельных цен коммерческих помещений, часть из~которых обладает некоторым признаком (например, имеет отдельный вход), часть "--- нет;
	\item необходимо установить: оказывает~ли наличие этого признака существенное влияние на~удельную стоимость недвижимости данного типа или~нет.
\end{itemize}
На~первый взгляд, согласно сложившейся практике, оценщик может просто субъективно признать те~или~иные признаки значимыми, а~прочие нет, после чего принять значения корректировок на~различия в~этих признаках из~справочников. Однако, как~было сказано выше, такой подход вряд~ли может считаться лучшей практикой, поскольку в~этом случае отсутствует какой-либо серьёзный анализ рынка. Кроме того, в~таком случае вряд~ли можно говорить о~какой-либо ценности такой работы в~принципе.

Вместо этого возможно использовать случайные выборки рыночных данных и~применять к~ним математические методы анализа, позволяющие делать доказательные с~научной точки зрения выводы о~значимости влияния того или~иного признака на~стоимость. Данные, используемые в~настоящей работе, являются вымышленными и~служат для~учебных целей. Прилагаемая электронная таблица настроена таким образом, что~исходные данные могут быть сгенерированы случайным образом, что~позволяет лучше понять поведение теста. 

\section{Основные сведения о~тесте}
Критерий Манна-Уитни-Вилкоксона~(U-Я) 

\section{Практическая реализация}

\section{Выводы}

\printbibliography[title=Источники информации]

\end{document}
