% Лицензия
% Apache License Version 2.0, January 2004
% http://www.apache.org/licenses/
% Copyright [2020] [Kirill A. Murashev]
% Licensed under the Apache License, Version 2.0 (the "License"); you may not use this file except in compliance with the License. You may obtain a copy of the License at
% http://www.apache.org/licenses/LICENSE-2.0
% Unless required by applicable law or agreed to in writing, software
% distributed under the License is distributed on an "AS IS" BASIS,
% WITHOUT WARRANTIES OR CONDITIONS OF ANY KIND, either express or implied.
% See the License for the specific language governing permissions and limitations under the License.


%%% Работа с русским языком
\usepackage{cmap}					% поиск в PDF
\usepackage{mathtext} 				% русские буквы в формулах
\usepackage{fontspec}
\defaultfontfeatures{Renderer=Basic,Ligatures={TeX}}
\setmainfont{CMU Serif}
\setsansfont{CMU Sans Serif}
\setmonofont{CMU Typewriter Text}
\usepackage[english,russian]{babel}
%\usepackage[T1,T2A]{fontenc}			% кодировка
%\usepackage[lutf8]{luainputenc}			% кодировка исходного текста
%\usepackage[english,russian]{babel}	% локализация и переносы
\usepackage{indentfirst}            % красная строка
\usepackage{misccorr}               % доработки для babel
\frenchspacing                      % французский стиль пробелов

%\usepackage{beton} %изменение шрифта для тёмной цветовой схемы
%\usepackage{concrete}
%%% Дополнительная работа с математикой
\usepackage{amsmath,amsfonts,amssymb,amsthm,mathtools} % AMS
\usepackage{icomma} % "Умная" запятая: $0,2$ --- число, $0, 2$ --- перечисление

%% Номера формул
%\mathtoolsset{showonlyrefs=true} % Показывать номера только у тех формул, на которые есть \eqref{} в тексте.
%\usepackage{leqno} % Нумерация формул слева

%% Перенос знаков в формулах (по Львовскому)
\newcommand*{\hm}[1]{#1\nobreak\discretionary{}
	{\hbox{$\mathsurround=0pt #1$}}{}}

%%% Работа с картинками
\usepackage{graphicx}  % Для вставки рисунков
\graphicspath{{Images/}}  % папки с картинками
\setlength\fboxsep{3pt} % Отступ рамки \fbox{} от рисунка
\setlength\fboxrule{1pt} % Толщина линий рамки \fbox{}
\usepackage{wrapfig} % Обтекание рисунков текстом

%%% Работа с таблицами
\usepackage{array, tabularx, tabulary, booktabs, xtab} % Дополнительная работа с таблицами
\usepackage{longtable}  % Длинные таблицы
\usepackage{multirow} % Слияние строк в таблице

%%% Теоремы
\theoremstyle{plain} % Это стиль по умолчанию, его можно не переопределять.
\newtheorem{theorem}{Теорема}[section]
\newtheorem{proposition}[theorem]{Утверждение}
\newtheorem{lemma}[theorem]{Лемма}

\theoremstyle{definition} % "Определение"
\newtheorem{corollary}{Следствие}[theorem]
\newtheorem{problem}{Задача}[section]

\theoremstyle{remark} % "Примечание"
\newtheorem*{nonum}{Решение}

%%% Программирование
\usepackage{etoolbox} % логические операторы

\usepackage{lastpage} % Узнать, сколько всего страниц в документе.

\usepackage{keyval}

\usepackage{totcount} % Узнать, сколько всего объектов в документе.

%\usepackage{xcolor-solarized}

%%% Страница
%\usepackage{extsizes} % Возможность сделать 14-й шрифт
%\usepackage{geometry} % Простой способ задавать поля
%	\geometry{top=25mm}
%	\geometry{bottom=35mm}
%	\geometry{left=35mm}
%	\geometry{right=20mm}
%

%\usepackage{fancyhdr} % Колонтитулы

%	\pagestyle{fancy}
%\renewcommand{\headrulewidth}{0pt}  % Толщина линейки, отчеркивающей верхний колонтитул
%\fancyhf{}
%\lhead{Часть \thepart}
%\chead{Глава \thechapter}
%\rhead{Раздел \thesection}
%\lfoot{version 0.251}
%\cfoot{\today} % По умолчанию здесь номер страницы
%\rfoot{\thepage/\ref{LastPage}}
%\pagestyle{fancy}

%\usepackage{setspace} % Интерлиньяж
%\onehalfspacing % Интерлиньяж 1.5
%\doublespacing % Интерлиньяж 2
%\singlespacing % Интерлиньяж 1

\usepackage{soul} % Модификаторы начертания

\usepackage[usenames,dvipsnames,svgnames,table,rgb]{xcolor} % Подключение пакета для задания цвета

%\definecolor{Backcolor}{HTML}{042029} % Задание цвета для фона
%\definecolor{Textcolor}{HTML}{819090} % Задание цвета для текста
%\pagecolor{Backcolor}                 % Подключение тёмной
%\color{Textcolor}                     % темы

\usepackage{csquotes} % Ещё инструменты для ссылок

\usepackage[backend=biber,bibencoding=utf8,sorting=ynt,maxcitenames=5,sortupper=true,date=iso]{biblatex} % подключение пакета для работы с автоматизированной библиографией

%\usepackage[style=authoryear,maxcitenames=2,backend=biber,sorting=nty]{biblatex}

%\renewcommand\bibname{Источники информации} % Переопределение названия для библиографии

\usepackage{multicol} % Несколько колонок

\usepackage{microtype}              %<-- added for better inter word spacing

\usepackage{tabularx}

\usepackage{tikz} % Работа с графикой
\usepackage{pgfplots}
\usepackage{pgfplotstable}

\usepackage{eqlist}

\usepackage{desclist} % Дополнительное окружение для списка Глоссария

\setcounter{tocdepth}{8} % Глубина оглавления

% подавление висячих строк
\clubpenalty=400 % Разрешение = 300, абсолютный запрет = 10000
\widowpenalty=400 % Увеличиваем эти числа до тех пор, пока не начнёт увеличиваться количество страниц.

% Выбор между разрежением и переполнением
\tolerance=500 % max=10000, default=200

\looseness=-1 % иногда можно удлинять страницу на одну строку.

\hfuzz=2.5pt % иногда можно вылезти за край строки на 2.5 pt.

\usepackage{calc} % Вычисления

\usepackage{scrlayer-scrpage} % Стиль страницы

\usepackage{lineno} % нумерация строк

%\pagestyle{scrpage}

%\usepackage{concrete}

\usepackage{booktabs}

\usepackage[owncaptions]{vhistory} % Log of versions

\usepackage{progressbar} % Формирование линейки, показывающей прогресс в работе

\usepackage{epigraph} % работа с эпиграфами

\usepackage {listings}
\lstloadlanguages{[Latex]Tex, bash, R, Python, SQL}
\lstset{extendedchars=true , % включаем не латиницу
frame=tb, % рамка сверху и снизу
commentstyle=\itshape , % шрифт для комментариев
stringstyle =\ttfamily % шрифт для строк
%keywordstyle=\color{blue}
}

%\usepackage{titling} %дополнительная настройка титульного листа

\setcounter{secnumdepth}{8} % Установка глубины нумерации заголовков

% Работа с гиперрсылками, подключается последним
\usepackage{hyperref}       % Подключение пакета для работы с гиперссылками
\hypersetup{				% Гиперссылки
	unicode=true,           % русские буквы в раздела PDF
	pdftitle={Искусственный интеллект в~оценке стоимости},   % Заголовок
	pdfauthor={К.\,А.~Мурашев},      % Автор
	pdfsubject={Системы поддержки принятия решений, основанные на искусственном интеллекте},      % Тема
	pdfcreator={К.\,А.~Мурашев}, % Создатель
	pdfproducer={К.\,А.~Мурашев}, % Производитель
	pdfkeywords={Искусственный интеллект, машинное обучение, математические методы, оценочная деятельность, цифровая экономика, Data Science, анализ данных} % Ключевые слова
	colorlinks=true,       	% false: ссылки в рамках; true: цветные ссылки
	linkcolor=red,          % внутренние ссылки
	citecolor=green,        % на библиографию
	filecolor=magenta,      % на файлы
	urlcolor=blue           % на URL
}

\usepackage{pgfplots} 
\pgfplotsset{compat=1.15}
\usepackage{mathrsfs}
\usetikzlibrary{arrows}
%\usepackage{url}

%\usepackage{totpages}

%\usepackage[strings]{underscore}

%\author{К.\,А.~Мурашев\thanks {\href{kirill.murashev@tutanota.de}{kirill.murashev@tutanota.de}, \href{https://t.me/Maas\_88}{https://t.me/Maas\_88}, \href{https://www.facebook.com/murashev.kirill}{https://www.facebook.com/murashev.kirill}}}
%\title{\Large Современные системы поддержки принятия решений оценщиками, основанные на~применении методов машинного обучения: практическое руководство по~применению языка программирования R в~повседневной практике оценщика}
%\date{\today}

%\normalsize

% Макрос для рисунков, обтекаемых текстом
\newcommand*{\EpsWrapD}[7]{%
	\begin{wrapfigure}[#5]{#3}{#2 \textwidth} % #3=l,r,L,R
		\begin{center} \sffamily
			\includegraphics*[width= #2 \textwidth ]{#1} % 1-имя файла и метка заодно,
			% 2-ширина рисунка (доля от ширины страницы)
			\vspace{-#7mm} % #7: сократить расстояние между подписью снизу и рисунком
			\caption{\label{fig:#1}#4} % #4 - подпись под рисунком
			\vspace{-#6pt}
		\end{center}% #6: сократить расстояние между подписью снизу и текстом после таблицы 
	\end{wrapfigure}}
%
% макрос для создания таблицы, обтекаемой текстом
\newcommand*{\TableBE}[5]{
	\begin{table}[#1] %\captionabove
		\vspace*{-#5mm}
		\centering \sffamily \caption{\label{tab:#2}#3} \begin{tabular}{#4} \toprule }
		
		\newcommand*{\TableEN}[3]{
			\bottomrule \end{tabular}
		\vspace{-#2mm} \small \begin{flushleft} #1 \end{flushleft}
		\vspace{-#3mm}
\end{table}}


\addbibresource{/home/kaarlahti/TresoritDrive/Methodics/My/AI_for_valuers/Book/AI_for_valuers_book/Basic_principles.bib}
\addbibresource{/home/kaarlahti/TresoritDrive/Methodics/My/AI_for_valuers/Book/AI_for_valuers_book/LaTeX.bib}
\addbibresource{/home/kaarlahti/TresoritDrive/Methodics/My/AI_for_valuers/Book/AI_for_valuers_book/Mathstat.bib}
\addbibresource{/home/kaarlahti/TresoritDrive/Methodics/My/AI_for_valuers/Book/AI_for_valuers_book/Murashev.bib}
\addbibresource{/home/kaarlahti/TresoritDrive/Methodics/My/AI_for_valuers/Book/AI_for_valuers_book/Python.bib}
\addbibresource{/home/kaarlahti/TresoritDrive/Methodics/My/AI_for_valuers/Book/AI_for_valuers_book/R.bib}
\addbibresource{/home/kaarlahti/TresoritDrive/Methodics/My/AI_for_valuers/Book/AI_for_valuers_book/RussianLaws.bib}
\addbibresource{/home/kaarlahti/TresoritDrive/Methodics/My/AI_for_valuers/Book/AI_for_valuers_book/Sci&Tech.bib}
\addbibresource{/home/kaarlahti/TresoritDrive/Methodics/My/AI_for_valuers/Book/AI_for_valuers_book/Valuation.bib}
\addbibresource{/home/kaarlahti/TresoritDrive/Methodics/My/AI_for_valuers/Book/AI_for_valuers_book/ValuationStandards.bib}
\addbibresource{/home/kaarlahti/TresoritDrive/Methodics/My/AI_for_valuers/Book/AI_for_valuers_book/ZHZL.bib}

\pagestyle{headings} 
\markright{Искусственный интеллект в~оценке стоимости}
\usepackage{pgfplots}
\pgfplotsset{compat=1.15}
\usepackage{mathrsfs}
\usetikzlibrary{arrows}

%\usepackage{polyglossia}

%\usepackage{minted}

\newtheorem{Thexmpl}[theorem]{Пример}

\usepackage[inkscapearea=page]{svg}
\usepackage{adjustbox}

\DeclareMathOperator{\rank}{rank}
\makeatletter
\newenvironment{sqcases}{%
	\matrix@check\sqcases\env@sqcases
}{%
	\endarray\right.%
}
\def\env@sqcases{%
	\let\@ifnextchar\new@ifnextchar
	\left\lbrack
	\def\arraystretch{1.2}%
	\array{@{}l@{\quad}l@{}}%
}
\makeatother

\DeclareMathOperator{\arcsec}{arcsec}
\DeclareMathOperator{\arccot}{arccot}
\DeclareMathOperator{\arccsc}{arccsc}
\DeclareMathOperator{\sgn}{sgn}
