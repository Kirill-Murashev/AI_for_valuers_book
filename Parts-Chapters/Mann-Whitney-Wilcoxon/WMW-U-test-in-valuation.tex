\documentclass[]{scrreprt}

\usepackage{amsmath,amsfonts,amssymb,amsthm,mathtools} % AMS

\usepackage{hyperref}       % hyperref
\hypersetup{				% settings
	unicode=true,           % non-latin letters
	pdftitle={Practical application of the Wilcoxon-Mann-Whitney test in valuation
},   % heading
	pdfauthor={K. A. Murashev},      % Author
	pdfsubject={Wilcoxon-Mann-Whitney test},      % Scope
	pdfcreator={K. A. Murashev}, % Creator
	pdfproducer={K. A. Murashev}, % Producer
	pdfkeywords={Wilcoxon-Mann-Whitney test, U-test} % Keywords
	colorlinks=true,       	% false: links in frames; true: coloured links
	linkcolor=red,          % internal links
	citecolor=green,        % bibliography links
	filecolor=magenta,      % file links
	urlcolor=blue           % URL links
}

\usepackage{url}

\usepackage[russian, english]{babel}
%\usepackage{csquotes}
% work with images
\usepackage{graphicx}
\graphicspath{{Images/}}

% work with tables
% additional forms of tables
\usepackage{array}
\usepackage{tabularx}
\usepackage{tabulary}
\usepackage{booktabs}
\usepackage{xtab}
\usepackage{longtable}  % long tables
\usepackage{multirow} % merge rows

% work with bibliography
\usepackage[backend=biber,bibencoding=utf8,sorting=ynt,maxcitenames=5,sortupper=true,date=iso]{biblatex}

% set the depth of table of contents
\setcounter{tocdepth}{8}

% set the depth of headings numbering
\setcounter{secnumdepth}{8}

\usepackage{float}

\usepackage[usenames,dvipsnames,svgnames,table,rgb]{xcolor}

% work with scripts
\usepackage{listings}
\lstloadlanguages{[Latex]Tex, bash, R, Python, SQL}
\lstset{extendedchars=true, % additional symbols
frame=tb, % top & bottom frames
%commentstyle=\itshape , % font for comments
%stringstyle =\ttfamily % font for 'strings'
%keywordstyle=\color{blue} % color for keywords
}

% connecting the automated bibliography package
\usepackage[backend=biber,bibencoding=utf8,sorting=ynt,maxcitenames=5,sortupper=true,date=iso]{biblatex} 

% add sources for bibliography
\addbibresource{/home/kaarlahti/TresoritDrive/Methodics/My/AI_for_valuers/Book/AI_for_valuers_book/Basic_principles.bib}
\addbibresource{/home/kaarlahti/TresoritDrive/Methodics/My/AI_for_valuers/Book/AI_for_valuers_book/LaTeX.bib}
\addbibresource{/home/kaarlahti/TresoritDrive/Methodics/My/AI_for_valuers/Book/AI_for_valuers_book/Mathstat.bib}
\addbibresource{/home/kaarlahti/TresoritDrive/Methodics/My/AI_for_valuers/Book/AI_for_valuers_book/Murashev.bib}
\addbibresource{/home/kaarlahti/TresoritDrive/Methodics/My/AI_for_valuers/Book/AI_for_valuers_book/Python.bib}
\addbibresource{/home/kaarlahti/TresoritDrive/Methodics/My/AI_for_valuers/Book/AI_for_valuers_book/R.bib}
\addbibresource{/home/kaarlahti/TresoritDrive/Methodics/My/AI_for_valuers/Book/AI_for_valuers_book/RussianLaws.bib}
\addbibresource{/home/kaarlahti/TresoritDrive/Methodics/My/AI_for_valuers/Book/AI_for_valuers_book/Sci&Tech.bib}
\addbibresource{/home/kaarlahti/TresoritDrive/Methodics/My/AI_for_valuers/Book/AI_for_valuers_book/Valuation.bib}
\addbibresource{/home/kaarlahti/TresoritDrive/Methodics/My/AI_for_valuers/Book/AI_for_valuers_book/ValuationStandards.bib}
\addbibresource{/home/kaarlahti/TresoritDrive/Methodics/My/AI_for_valuers/Book/AI_for_valuers_book/ZHZL.bib}

\usepackage{bbm} % indicator function

\newcommand{\github}{
	{%
		\includegraphics[width=3ex,height=3ex,keepaspectratio]{github-seeklogo.pdf}
}
}


% Title Page
\title{Practical application of~the~Wilcoxon-Mann-Whitney test in~valuation.}
\subtitle{Selection of~attributes as~pricing factors based on~the~principle of~unbiased estimates}
\author{\href{https://www.facebook.com/groups/1977067932456703}{K.~A.~Murashev}}

\begin{document}
\maketitle
%
\lstset{language=Python,
	basicstyle=\ttfamily,
	keywordstyle=\color{Blue}\ttfamily,
	stringstyle=\color{Red}\ttfamily,
	commentstyle=\color{Emerald}\ttfamily,
	morecomment=[l][\color{Magenta}]{\#},
	breaklines=true,
	breakindent=0pt,
	breakatwhitespace,
	columns=fullflexible,
	showstringspaces=false
}
%	
\begin{abstract}
	In~their practice appraisers often face the~need to~take into account differences in~quantitative and~qualitative characteristics of~objects. In~particular, one of~the~standard tasks is~to~determine the~attributes that influence the~cost (so-called "pricing factors") and~to~separate them from the~attributes that do~not or~cannot be~determined.
	
	Subjective selection of~attributes taken into account in~determining the~value is~widespread in~valuation practice. In~this case, specific quantitative indicators of~the~impact of~these attributes on~the~cost are~often taken from the~so-called "reference books". While not~denying the~speed and~low cost of~this approach, it~should~be recognized that only data directly observed in~the open markets is~a~reliable basis for~a~value judgment. The priority of~such data over other data, in~particular those obtained by~expert survey, is~enshrined, among others, in~\href{https://www.rics.org/uk/upholding-professional-standards/sector-standards/valuation/red-book/red-book-global/}{RICS Valuation --- Global Standards 2022}~\cite{RVGS-2022}, \href{https://www.rics.org/uk/upholding-professional-standards/sector-standards/valuation/red-book/international-valuation-standards/}{International Valuation Standards 2022}~\cite{IVS-2022}, as~well as~in~\href{http://eifrs.ifrs.org/eifrs/bnstandards/en/IFRS13.pdf}{IFRS~13 "Fair Value Measurement"}~\cite{IFRS-13}. Therefore, we~can say that mathematical methods for~analyzing data from the open market are the~most reliable means of~interpreting market information used in~market research and~predicting the~value of~individual objects.
	
	The aim of~this work is~to~justify the~necessity and~possibility of~using a~rigorous mathematical Wilcoxon--Mann--Whitney test, which allows us~to~answer the~question about the~necessity of~taking into account the~binary attribute as~a~price-generating factor. Instead of~the~judgmental approach, which is~most commonly used by~appraisers in~selecting the~attributes to~be considered in~appraisal, this paper proposes the~idea of~prioritizing the~measuring approach based on~the~results of~a~mathematical test that allows to~draw a~conclusion about the~importance or~otherwise of~the~binary attribute influence on~the~value. It~should~be noted that despite the~fact that the~statistical test under consideration belongs to~frequentist statistics, it, through its~connection to~ROC analysis and~AUC, is~related to~modern machine learning methods, which will~be discussed later in~the~text of~this material. The~presence of~this relationship and~elements of~Bayesian statistics seems particularly interesting and~promising from the~point of~view of~introducing machine learning and~data analysis methods into the~everyday practice of~appraisers.
	
	Users should have some general math background and~basic Python and~R programming skills to~understand and~practice all of~the material in~the~text, but~lack of~that knowledge and~skill is~not~a~barrier to~learning most of~the~material and~implementing the~test in~the~spreadsheet that comes with it.
	
	The material consists of~four blocks:
	\begin{itemize}
		\item a~description of~the~Wilcoxon--Mann--Whitney test (hereafter "U-test"), its probabilistic meaning, and~its relationship to~other mathematical methods;
		\item a~practical implementation of~the~U-test in~a~spreadsheet on~an~example of~test random data;
		\item practical implementation of~the~U-test on~the~real data of~the~residential real estate market of~St.~Petersburg agglomeration by~means of~Python programming language, the~purpose of~the~analysis was~to~check the~significance of~the~difference in~the~unit price between the~objects located in~the~urban and~suburban parts of~the~agglomeration;
		\item practical implementation of~the~U-test on~real data of~residential real estate market of~Almaty by~means of~R programming language, the~purpose of~the~analysis was~to~check the~significance of~difference in~unit price between the~objects sold without demountable improvements and~the~objects sold with them.
	\end{itemize}
	The~current version of~this material, its source code, Python and~R scripts, and~the~spreadsheet are~in~the~repository on~the GitHub portal and~are~available at~the~\href{https://github.com/Kirill-Murashev/AI_for_valuers_book/tree/main/Parts-Chapters/Mann-Whitney-Wilcoxon}{permanent link}~\cite{Murashev:u-test}.
	
	This material and~all of~its~appendices are~distributed under the~terms of~the~\href{https://creativecommons.org/licenses/by-sa/4.0/}{cc-by-sa-4.0} license~\cite{cc-by-sa-4.0}.
\end{abstract}	
%
\tableofcontents
\listoftables
\listoffigures
\lstlistoflistings
%	
\chapter{Technical details}

\clearpage

\nocite{Essential-Statistical-Inference}
\nocite{AUC-optimization}
\nocite{Mann-Whitney-1947}
\nocite{Optimizing-classifier-performance}
\nocite{ROC-R-1}
\nocite{ROC-AUC-1}
\nocite{ROC-AUC-meets-U-R-1}

\printbibliography

\end{document}          
