\documentclass[]{scrreprt}
% \KOMAoptions{fontsize=14pt}

\usepackage{amsmath,amsfonts,amssymb,amsthm,mathtools} % AMS

\usepackage{hyperref}       % hyperref
\hypersetup{				% settings
	unicode=true,           % non-latin letters
	pdftitle={Practical application of the Wilcoxon-Mann-Whitney test in valuation
},   % heading
	pdfauthor={K. A. Murashev},      % Author
	pdfsubject={Wilcoxon-Mann-Whitney test},      % Scope
	pdfcreator={K. A. Murashev}, % Creator
	pdfproducer={K. A. Murashev}, % Producer
	pdfkeywords={Wilcoxon-Mann-Whitney test, U-test} % Keywords
	colorlinks=true,       	% false: links in frames; true: coloured links
	linkcolor=red,          % internal links
	citecolor=green,        % bibliography links
	filecolor=magenta,      % file links
	urlcolor=blue           % URL links
}

\usepackage{url}

\usepackage[russian, english]{babel}
%\usepackage{csquotes}
% work with images
\usepackage{graphicx}
\graphicspath{{Images/}}

% work with tables
% additional forms of tables
\usepackage{array}
\usepackage{tabularx}
\usepackage{tabulary}
\usepackage{booktabs}
\usepackage{xtab}
\usepackage{longtable}  % long tables
\usepackage{multirow} % merge rows

% work with bibliography
\usepackage[backend=biber,bibencoding=utf8,sorting=ynt,maxcitenames=5,sortupper=true,date=iso]{biblatex}

% set the depth of table of contents
\setcounter{tocdepth}{8}

% set the depth of headings numbering
\setcounter{secnumdepth}{8}

\usepackage{float}

\usepackage[usenames,dvipsnames,svgnames,table,rgb]{xcolor}

% work with scripts
\usepackage{listings}
\lstloadlanguages{[Latex]Tex, bash, R, Python, SQL}
\lstset{extendedchars=true, % additional symbols
frame=tb, % top & bottom frames
%commentstyle=\itshape , % font for comments
%stringstyle =\ttfamily % font for 'strings'
%keywordstyle=\color{blue} % color for keywords
}

% connecting the automated bibliography package
\usepackage[backend=biber,bibencoding=utf8,sorting=ynt,maxcitenames=5,sortupper=true,date=iso]{biblatex} 

% add sources for bibliography
\addbibresource{/home/kaarlahti/TresoritDrive/Methodics/My/AI_for_valuers/Book/AI_for_valuers_book/Basic_principles.bib}
\addbibresource{/home/kaarlahti/TresoritDrive/Methodics/My/AI_for_valuers/Book/AI_for_valuers_book/LaTeX.bib}
\addbibresource{/home/kaarlahti/TresoritDrive/Methodics/My/AI_for_valuers/Book/AI_for_valuers_book/Mathstat.bib}
\addbibresource{/home/kaarlahti/TresoritDrive/Methodics/My/AI_for_valuers/Book/AI_for_valuers_book/Murashev.bib}
\addbibresource{/home/kaarlahti/TresoritDrive/Methodics/My/AI_for_valuers/Book/AI_for_valuers_book/Python.bib}
\addbibresource{/home/kaarlahti/TresoritDrive/Methodics/My/AI_for_valuers/Book/AI_for_valuers_book/R.bib}
\addbibresource{/home/kaarlahti/TresoritDrive/Methodics/My/AI_for_valuers/Book/AI_for_valuers_book/RussianLaws.bib}
\addbibresource{/home/kaarlahti/TresoritDrive/Methodics/My/AI_for_valuers/Book/AI_for_valuers_book/Sci&Tech.bib}
\addbibresource{/home/kaarlahti/TresoritDrive/Methodics/My/AI_for_valuers/Book/AI_for_valuers_book/Valuation.bib}
\addbibresource{/home/kaarlahti/TresoritDrive/Methodics/My/AI_for_valuers/Book/AI_for_valuers_book/ValuationStandards.bib}
\addbibresource{/home/kaarlahti/TresoritDrive/Methodics/My/AI_for_valuers/Book/AI_for_valuers_book/ZHZL.bib}

\usepackage{bbm} % indicator function

\newcommand{\github}{
	{%
		\includegraphics[width=3ex,height=3ex,keepaspectratio]{github-seeklogo.pdf}
}
}


% Title Page
\title{Practical application of~the~Wilcoxon-Mann-Whitney test in~valuation.}
\subtitle{Selection of~attributes as~pricing factors based on~the~principle of~unbiased estimates}
\author{\href{https://www.facebook.com/groups/1977067932456703}{K.~A.~Murashev}}

\begin{document}
\maketitle
%
\lstset{language=Python,
	basicstyle=\ttfamily,
	keywordstyle=\color{Blue}\ttfamily,
	stringstyle=\color{Red}\ttfamily,
	commentstyle=\color{Emerald}\ttfamily,
	morecomment=[l][\color{Magenta}]{\#},
	breaklines=true,
	breakindent=0pt,
	breakatwhitespace,
	columns=fullflexible,
	showstringspaces=false
}
%	
\begin{abstract}
	In~their practice appraisers often face the~need to~take into account differences in~quantitative and~qualitative characteristics of~objects. In~particular, one of~the~standard tasks is~to~determine the~attributes that influence the~cost (so-called "pricing factors") and~to~separate them from the~attributes that do~not or~cannot be~determined.
	
	Subjective selection of~attributes taken into account in~determining the~value is~widespread in~valuation practice. In~this case, specific quantitative indicators of~the~impact of~these attributes on~the~cost are~often taken from the~so-called "reference books". While not~denying the~speed and~low cost of~this approach, it~should~be recognized that only data directly observed in~the open markets is~a~reliable basis for~a~value judgment. The priority of~such data over other data, in~particular those obtained by~expert survey, is~enshrined, among others, in~\href{https://www.rics.org/uk/upholding-professional-standards/sector-standards/valuation/red-book/red-book-global/}{RICS Valuation --- Global Standards 2022}~\cite{RVGS-2022}, \href{https://www.rics.org/uk/upholding-professional-standards/sector-standards/valuation/red-book/international-valuation-standards/}{International Valuation Standards 2022}~\cite{IVS-2022}, as~well as~in~\href{http://eifrs.ifrs.org/eifrs/bnstandards/en/IFRS13.pdf}{IFRS~13 "Fair Value Measurement"}~\cite{IFRS-13}. Therefore, we~can say that mathematical methods for~analyzing data from the open market are the~most reliable means of~interpreting market information used in~market research and~predicting the~value of~individual objects.
	
	The aim of~this work is~to~justify the~necessity and~possibility of~using a~rigorous mathematical Wilcoxon--Mann--Whitney test, which allows us~to~answer the~question about the~necessity of~taking into account the~binary attribute as~a~price-generating factor. Instead of~the~judgmental approach, which is~most commonly used by~appraisers in~selecting the~attributes to~be considered in~appraisal, this paper proposes the~idea of~prioritizing the~measuring approach based on~the~results of~a~mathematical test that allows to~draw a~conclusion about the~importance or~otherwise of~the~binary attribute influence on~the~value. It~should~be noted that despite the~fact that the~statistical test under consideration belongs to~frequentist statistics, it, through its~connection to~ROC analysis and~AUC, is~related to~modern machine learning methods, which will~be discussed later in~the~text of~this material. The~presence of~this relationship and~elements of~Bayesian statistics seems particularly interesting and~promising from the~point of~view of~introducing machine learning and~data analysis methods into the~everyday practice of~appraisers.
	
	Users should have some general math background and~basic Python and~R programming skills to~understand and~practice all of~the material in~the~text, but~lack of~that knowledge and~skill is~not~a~barrier to~learning most of~the~material and~implementing the~test in~the~spreadsheet that comes with it.
	
	The material consists of~four blocks:
	\begin{itemize}
		\item a~description of~the~Wilcoxon--Mann--Whitney test (hereafter "U-test"), its probabilistic meaning, and~its relationship to~other mathematical methods;
		\item a~practical implementation of~the~U-test in~a~spreadsheet on~an~example of~test random data;
		\item practical implementation of~the~U-test on~the~real data of~the~residential real estate market of~St.~Petersburg agglomeration by~means of~Python programming language, the~purpose of~the~analysis was~to~check the~significance of~the~difference in~the~unit price between the~objects located in~the~urban and~suburban parts of~the~agglomeration;
		\item practical implementation of~the~U-test on~real data of~residential real estate market of~Almaty by~means of~R programming language, the~purpose of~the~analysis was~to~check the~significance of~difference in~unit price between the~objects sold without demountable improvements and~the~objects sold with them.
	\end{itemize}
	The~current version of~this material, its source code, Python and~R scripts, and~the~spreadsheet are~in~the~repository on~the GitHub portal and~are~available at~the~\href{https://github.com/Kirill-Murashev/AI_for_valuers_book/tree/main/Parts-Chapters/Mann-Whitney-Wilcoxon}{permanent link}~\cite{Murashev:u-test}.
	
	This material and~all of~its~appendices are~distributed under the~terms of~the~\href{https://creativecommons.org/licenses/by-sa/4.0/}{cc-by-sa-4.0} license~\cite{cc-by-sa-4.0}.
\end{abstract}	
%
\tableofcontents
\listoftables
\listoffigures
\lstlistoflistings
%	
\chapter{Technical details}
This material, as~well as~the~appendices to~it, are available at~\href{https://github.com/Kirill-Murashev/AI_for_valuers_book/tree/main/Parts-Chapters/Mann-Whitney-Wilcoxon}{permanent link}~\cite{Murashev:u-test}. The~source code for~this work was~created~using the~language~\href{https://www.ctan.org/}{\TeX}~\cite{TeX:site} with~a~set of~macro extensions~\href{https://www latex-project.org/}{\LaTeXe}~\cite{LaTeX:site}, distribution~\href{https://www.tug.org/texlive/}{TeXLive}~\cite{TeXLive:site} and~Editor~\href{https://www.texstudio.org/}{TeXstudio}~\cite{TeXstudio:site}. The~spreadsheet calculation was~done with \href{https://www.libreoffice.org/discover/calc/}{LibreOffice Calc}~\cite{LO:Calc} (Version: 7.3.4. 2 / LibreOffice Community Build ID: 30(Build:2); CPU threads: 4; OS: Linux 5.11; UI render: default; VCL: kf5 (cairo+xcb) Locale: en-US (en\_US.UTF-8); UI: en-US Ubuntu package version: 1:7.3.4~rc2-0ubuntu0.20.04.1~lo1; Calc: threaded). The~calculation in~\href{https://www.r-project.org/}{R}~\cite{R_language} (version 4.2.1 (2022-06-23) -- "Funny-Looking Kid") was~done~using an~IDE~\href{https://www.rstudio.com/}{RStudio} (RStudio 2022. 02.3+492 "Prairie Trillium" Release (1db809b8, 2022-05-20) for Ubuntu Bionic; Mozilla/5.0 (X11; Linux x86\_64); AppleWebKit/537.36 (KHTML, like Gecko); QtWebEngine/5.12.8; Chrome/69.0.3497.128; Safari/537.36)~\cite{RStudio:official_site}. The~calculation in~\href{https://www.python.org/}{Python}~(Version~3.9.12)~\cite{Python:site} was~performed using the~development environment~\href{https://jupyter.org}{Jupyter Lab} (Version 3.4.2)~\cite{Jupyter:site} and~IDE \href{https://www.spyder-ide.org/}{Spyder} (Spyder version: 5.1.5 None* Python version: 3.9.12 64-bit * Qt version: 5.9.7 * PyQt5 version: 5.9.2
* Operating System: Linux 5.11.0-37-generic)~\cite{Spyder:site}. The~graphics used in~the~subsection \ref{U-test-spreadsheet} were prepared using \href{Geogebra:official-site}{Geogebra}~(Version 6.0.666.0-202109211234)~\cite{Geogebra:official-site}. The~following values were used in~this material as~well as~in~most of~the~works in~the~series:
\begin{itemize}
	\item significance level: $\alpha = 0.05$;
	\item confidence interval: $Pr = 0.95$;
	\item initial position of the pseudo-random number generator: $seed=19190709$.
\end{itemize}
A~dot is~used as~a~decimal point. Most of~the~mathematical notations are~written as~they are~used in~English-speaking circles. For~example, a~tangent is~written as~$\tan$, not~$\tg$. The~results of~statistical tests are~considered significant when
\begin{equation}\label{eq:significance}
p \leq \alpha.
\end{equation}
This decision is~based, in~part, on~the~results of~the~discussion that took place on~\href{researchgate.net}{researchgate.net}~~\cite{RG:p-equals-alpha}.
%
\chapter{Subject of~research}
When working with market data, the~appraiser is~often faced with the~task of~testing the~hypothesis of~whether a~quantitative, ordinal or~nominal attribute has~a~significant effect on~the~price. Real estate market analysts, developers, realtors, employees of~collateral departments of~banks, leasing and~insurance companies, tax inspectors and~other specialists have a~similar task. At~the~same time, it~is~often impossible to~collect large amounts of~data that would allow a~wide range of~machine learning methods to~be~applied. In~some cases appraisers consciously narrow the~area of~data collection to~the~narrow market segment, resulting in~only very small samples of~less than thirty observations at~their disposal. In~this case, the~price data most often has~a~distribution that differs from the~normal one. In~this case, a~rational solution is~to~use U-test. Let~us formulate the~problem:
\begin{itemize}
	\item suppose that we~have two samples of~unit prices for~commercial premises, some of~which have some attribute (e.\,g., having a~separate entrance) and~some of~which do~not;
	\item it~is~necessary to~determine whether the~presence of~this feature has a~significant impact on~the~unit value of~this type of~real estate or~not.
\end{itemize}
At~first glance, according to~established practice, an~appraiser can simply subjectively recognize some attributes as~significant and~others as~not, and~then accept the~adjustment values for~differences in~these attributes from the~reference books. However, as~mentioned above, this approach is~hardly considered best practice because it~lacks any~market analysis. Also, in~that case, it~is~unlikely that such work is~of~any~serious value at~all.

Instead, it~is~possible to~use random samples of~market data and~apply mathematical analysis to~them, allowing scientific and evidence-based conclusions to~be~drawn about the~significance of~a~particular attribute's impact on~value. The~data used in~this paper to~perform the~U-test using Python and~R are~real market data, some of~which were collected by~the author through web scraping and some provided by~colleagues for~the~analysis. The~attached spreadsheet is~set~up so~that test raw data can~be generated in~a~pseudo-random fashion.

The~subject of~this paper is~the~nonparametric Wilcoxon-Mann-Whitney test, specifically designed for~samples that have a~distribution other than normal. This circumstance is~important because the~price data that appraisers deal with most often have this distribution, which excludes the~possibility of~applying the~parametric t-criterion and~z-criterion. In~addition, the~test under consideration is~of~great interest because it~has a~connection to~machine learning methods through AUC, the~calculation of~which through the~formula provided in~the~test framework gives a~value equal to~that calculated by~ROC analysis. Thus, the~study of~the~U-test paves the~way for~a~further dive into the~world of~machine learning, which is~entering many areas of~human activity and~will significantly change the~field of~value estimation in~the~foreseeable future.

The~material contains a~description of~the~test and~instructions for~performing it, sufficient in~the~author's opinion for~its~demonstrable use in~the~estimation process.
%
\chapter{Basic information about the~test}
\section{Assumptions and~formalization of~hypotheses}
First of~all, it~should~be said that, in~spite of~the~stated common name, it~is~more correct to~speak of~two tests:
\begin{itemize}
	\item \href{http://www.machinelearning.ru/wiki/index.php?title=Критерий_Уилкоксона_двухвыборочный}{Wilcoxon rank-sum test} developed by~Frank Wilcoxon in~1945~\cite{MLRU:Wilcoxon-test};
	\item \href{http://www.machinelearning.ru/wiki/index.php?title=Критерий_Уилкоксона-Манна"--~Уитни}{Mann--Whitney~U-test} which is~a~further development of~the~aforementioned criterion developed by~Henry Mann and~Donald Whitney in~1947~\cite{MLRU:Mann-Whitney}.
\end{itemize}
Looking ahead we~can say that the~statistics of~these criteria are~linearly related and~their p-values are~almost the~same which from a~practical point of~view allows us to~talk about variations of~one test rather than two separate tests. This paper uses the~common name throughout the~text, as~well as~a~shortened version of~"U-test" which historically refers to~the~Mann-Whitney test. Some authors\cite{Kobzarq-prikl-mathstat} recommend using the~Wilcoxon rank-sum test when there are~no~assumptions about variance, and the~Mann-Whitney U-test when variance of~the~two samples are~equal. However, the~experimental data indicate that the~Wilcoxon rank-sum test and~Mann-Whitney U-test values are~essentially the~same when the~variance of~the~samples is~significantly different. Adhering to~the~KISS principle~\cite{KISS-principle} underlying the~entire series of~publications, the~author concludes that a~unified approach is~possible. Also remember that the~Wilcoxon signed-rank test is~a~separate test designed to~analyze differences between two matched samples, whereas the~Mann-Whitney U-test discussed in~this paper is~designed to~work with two independent samples.

Suppose that there are~two samples:
\begin{equation*}
x^{m} = (x_{1},x_{2},\ldots,x_{m}), x_{i} \in \mathbb{R};\quad y^{n} = (y_{1},y_{2},\ldots,y_{n}), y_{i} \in \mathbb{R} \quad: m \leq n.
\end{equation*}
%
\begin{itemize}
	\item Both samples are~simple and~random (i.e., \href{https://en.wikipedia.org/wiki/Simple_random_sample}{SRS}~\cite{Wiki:SRS}), the~combined sample is~independent.
	\item The~samples are~taken from unknown continuous distributions \textit{F(x)} and~\textit{G(y)}, respectively.
\end{itemize}
%
\begin{description}
	\item[Simple random sample~(SRS) ---] is~a~subset of~individuals (\emph{a~sample}) chosen from a~larger set (\emph{a~population}) in~which a~subset of~individuals are~chosen randomly, all with the~same probability. It~is~a~process of~selecting a~sample in~a~random way. In~\textbf{SRS}, each subset of~\textit{k}~individuals has~the~same probability of~being chosen for~the~sample as~any~other subset of~\textit{k}~individuals.A~simple random sample is~an~unbiased sampling technique. Equivalent definition: a~sample ${\textstyle x^{m} = (x_{1},x_{2},\ldots,x_{m})}$ is~simple if~the~values~${\textstyle (x_{1},x_{2},\ldots,x_{m})}$ are~realizations of~\textit{m} independent equally distributed random variables. In~other words, the~selection of~observations is~not~only random but also does not~imply any~special selection rules (e.g., choosing every 10th observation).
\end{description}
%
\begin{description}
	\item[The~U-test ---] is~a~nonparametric criterion to~test the~null hypothesis that for~randomly chosen from~two samples of~observations~$x \in X$ and~$y \in Y$ the probability that~\textit{x} is~greater than \textit{y} is~equal to~the~probability that~\textit{y} is~greater than~\textit{x}. In~mathematical language, the~null hypothesis is~written as~follows
	\begin{equation}\label{eq:U-test-null-hypothesis}
	H_{0}:P\{x<y=\frac{1}{2}\}.
	\end{equation}
	For~the~test's own consistency, an~alternative hypothesis is~required, which is~that the~probability that the~value of~a~characteristic of~observation from~\textit{X} is~greater than that of~observation from~\textit{Y} differs upward or~downward from the~probability that the~value of~a~characteristic of~observation from~\textit{Y} is~greater than that of~observation from~\textit{X}. In~mathematical language, the~alternative hypothesis is~written as~follows
	\begin{equation}\label{eq:U-test-alt-hypothesis}
	H_{1}:P\{x<y\} \neq P\{y<x\} \vee P\{x<y\} + 0.5 \cdot P\{x=y\} \neq 0.5.
	\end{equation}
\end{description}
According to~the~basic concept of~the~U-test, if~the~null hypothesis is~true, the~distribution of~the~two samples is~continuous; if~the~alternative hypothesis is~true, the~distribution of~one sample is~stochastically greater than the~distribution of~the~other. In~this case, it~is~possible to~formulate a~number of~null and~alternative hypotheses for~which this test will give a~correct result. His~most extensive generalization lies in~the~following assumptions:
\begin{itemize}
	\item the~observations in~both samples are~independent;
	\item the~data type is~at~least ranked, i.\,e., with respect to~any two observations you can tell which one is~greater;
	\item the~null hypothesis assumes that the~distributions of~the~two samples are equal;
	\item the~alternative hypothesis assumes that the~distributions of~the~two samples are unequal.
\end{itemize}
With a~stricter set of~assumptions than those given above, for~example the~assumption that the~distribution of~the~two samples is~continuous if~the~null hypothesis is~valid and that the~distribution of~the~two samples has a~shift  in~the~distribution if~the~alternative one is~valid i.\,e.~$f_{1}(x)=f_{2}(x+\sigma)$,  we~can say that the~U-test is~a~test for~the~hypothesis of~equality of~medians. In~this case, the~U-test can~be interpreted as~a~test of~whether Hodges--Lehman's estimate of~the~difference in~central tendency measures differs from zero. In~this situation, the~Hodges--Lehman estimate is~the median of~all possible values of~differences between the~observations in~the~first and second samples. However, if~both the~variance and the~shape of~the~distribution of~the~two samples differ, the~U-test cannot correctly test the~medians. Examples can~be shown where the~medians are~numerically equal and the~test rejects the~null hypothesis because of~the~small p-value. Thus, a~more correct interpretation of~the~U-test is~to~use~it to~test the~\href{http://www.machinelearning.ru/wiki/index.php?title=Гипотеза_сдвига}{shift hypothesis}~\cite{MLRU:shift-hypothesis}.
\begin{description}
	\item[Shift hypothesis ---] is~a~statistical hypothesis often considered as~an~alternative to~the~hypothesis of~complete homogeneity of~samples. Let~us have two samples of~data. Let~us also give two random variables~\textit{X} and~\textit{Y}, which are~distributed as~elements of~these samples and have distribution functions~\textit{F(x)} and~\textit{G(y)}, respectively. In~these terms, the~shift hypothesis can~be written as~follows
	\begin{equation}\label{eq:shift-hypothesis}
	H:F(x)=G(x+\sigma)\ : \forall x,\ \sigma \neq 0.
	\end{equation}
\end{description}
In~this case, the~U-criterion is~valid regardless of~the~characteristics of~the~samples.

Simply put, the~essence of~the~U-test is~that it~allows~us to~answer the~question of~whether there~is a~significant difference in~the~value of~the~quantitative attribute of~the~two samples. With regard to~valuation, we~can say that the~use of~this test helps to~answer the~question of~whether it~is~necessary to~take into account one or~another attribute as~a~price-generating factor. It~follows from the~above that by~default we~are talking about a~two-sided test. In~practice, this means that the~test does~not give a~direct answer to~the~question, for~example: "Is~there a~significant excess of~the~unit value of~premises with a~separate entrance to~the~premises that do~not have it. At~the~same time, there~are also one-sided realizations that allow~us to~answer the~question about the~sign of~the~difference in~the~value of~the~attribute in~the~two samples.

In~addition to~the~above requirements for~the~samples themselves, the~conditions for~applying the~U-test are:
\begin{itemize}
	\item the~distribution of~quantitative attribute values of~samples is~different from normal (otherwise it~is advisable to~use parametric Student's t-test or~z-test for~independent samples).
	\item at~least three observations in~each sample, it~is~allowed to~have two observations in~one of~the~samples, provided that there~are at~least five in~the~other sample.
\end{itemize}
To~summarize the~above, there~are three variants of~the~null hypothesis, depending on~the~level of~rigor outlined in~the~table below~\ref{tab:nul-hypothesis-variants}.
\begin{table}[ht]
	\caption{Variants of~the~null hypothesis when using the~U-test in~valuation.}\label{tab:nul-hypothesis-variants}
	\centering
	\begin{tabularx}{\textwidth}{p{0.25\linewidth} p{0.7\linewidth}} 
		\hline
		Type of~hypothesis&Formulation\\
		\hline
		Scientific&The~two samples are~completely homogeneous, i.\,e.~they belong to~the~same distribution, there~is no~shift and the~estimate made for~the~first sample is~unbiased for~the~second one.\\
		\hline
		Practical&The~medians of~the~two samples are~equal to~each other.\\
		\hline
		Set forth in~terms of~valuation&The~difference in~the~attribute between the~two samples of~object-analogues is~not~significant, its accounting is~not required and this attribute is~not a~pricing factor.\\
		\hline
	\end{tabularx}
\end{table}
%
\section{Test implementation}
\subsection{Test statistic}
Suppose that~the elements $x_{1},\ldots,x_{n}$ represent a~simple independent sample from~$X \in \mathbb{R}$, and~the elements $y_{1},\ldots, y_{n}$ represent a~simple independent sample from $Y \in \mathbb{R}$ and the~samples are~independent of~each other. Then the~relevant U-statistic is~defined as~follows:
\begin{equation}\label{eq:U-statistic-base-formula}
\begin{aligned}
U&=\sum_{i=1}^{m} \sum_{j=1}^{n} S (x_{i},y_{j}),\\
&\text{при}\\
S(x,y)&=
\begin{cases}
1,\quad \text{если}\ x>y,\\
\frac{1}{2},\quad \text{если}\ x=y,\\
0,\quad \text{если}\ x<y.
\end{cases}
\end{aligned}
\end{equation}
%
\subsection{Calculation methods}
The~test involves calculating a~statistic usually called the~U-statistic whose distribution is~known if~the~null hypothesis is~true. When working with very small samples, the~distribution is~specified tabularly; when the~sample size is~more than twenty observations, it~is~approximated quite well by~the~normal distribution. There are~two methods of~calculating U-statistics: manual calculation using the~formula \ref{eq:U-statistic-base-formula} or~using a~special algorithm. The~first method, due~to its~labor-intensive nature, is~only suitable for~very small samples. The~second method can~be formalized as~a~step-by-step set of~instructions and will~be described below.
\begin{enumerate}
	\item You must construct a~common variation series for~the~two samples and then assign a~rank to~each observation, starting with one for~the~smallest of~them. If~there are~ties, i.\,e. groups of~repeating values (such a~group can~be, e.\,g., only two equal values), each observation from such a~group is~assigned a~value equal to~the~median of~the~group ranks before adjustment (for example, in~the~case of~a~variation series (\textit{3, 5, 5, 5, 5, 8}) the~ranks before adjustment are~(\textit{1, 2, 3, 4, 5, 6}) after --- (\textit{1, 3.5, 3.5, 3.5, 3.5, 6}).
	\item It~is necessary to~calculate the~sums of~the~ranks of~the~observations of~each sample, denoted as~${R_{1},\ R_{2}}$ respectively. In~this case, the~total sum of~ranks can~be calculated by~the~formula
	\begin{equation}\label{eq:common-R}
	R = \frac{N(N+1)}{2},
	\end{equation}
	where~\textit{N} ---the~total number of~observations in~both samples.
	%
	\item Next, we~calculate the~U-value for~the~first sample:
	\begin{equation}\label{eq:U1}
	U_{1}=R_{1}-\frac{n_{1}(n_{1}+1)}{2},
	\end{equation}
	where $R_{1}$ ---the~sum of~ranks of~the~first sample, $n_{1}$ --- the~number of~observations in~the~first sample.
	%
	\item The~U-value for~the~second sample is~calculated in~the~same way:
	\begin{equation}\label{eq:U2}
	U_{2}=R_{2}-\frac{n_{2}(n_{2}+1)}{2},
	\end{equation}
	where $R_{2}$ ---the~sum of~ranks of~the~second sample, $n_{2}$ --- the~number of~observations in~the~second sample.
	
	From the~above formulas it~follows that
	\begin{equation}\label{eq:U1-U2-relation}
	U_{1}+U_{2} = R_{1}-\frac{n_{1}(n_{1}+1)}{2} + R_{2}-\frac{n_{2}(n_{2}+1)}{2}.
	\end{equation}
	It~is~also known that
	\begin{equation}\label{eq:R-N-relation}
	\begin{cases}
	R_{1}+R_{2}=\dfrac{N(N+1)}{2}\\
	N=n_{1}+n_{2}.
	\end{cases}
	\end{equation}
	Then
	\begin{equation}\label{eq:check-U-value}
	U_{1}+U_{2}=n_{1}n{2}.
	\end{equation}
	Using this formula as~a~control ratio can~be useful for~checking the~correctness of~calculations in~a~spreadsheet processor.
	%
	\item From the~two values of~$U_{1},\ U_{2}$ in~all cases we~choose the~smaller which will~be the~U-statistic and~used in~further calculations. Let~us denote it~as~\emph{U}.
\end{enumerate}
%
\subsection{Interpretation of~the~result}
For~a~correct interpretation of~the~test result it~is necessary to~specify:
\begin{itemize}
	\item size of each sample;
	\item values of~the~measure of~central tendency for~each sample (given the~nonparametric nature of~the~test, the~median appears to~be the~appropriate measure of~central tendency);
	\item the~value of~the~U-statistic itself;
	\item the~\href{https://en.wikipedia.org/wiki/Effect_size#Common_language_effect_size}{CLES} index~\cite{Wiki:CLES} the~value of~which is~equivalent to~the~AUC and~$\rho$-statistic;
	\item \href{https://en.wikipedia.org/wiki/Effect_size#Rank-biserial_correlation}{rank-biserial correlation coefficient~(RBC)}~\cite{Wiki:rank-biserial-correlation};
	\item the~accepted level of~significance (usually 0.05);
	\item the~calculated p-value.
\end{itemize}
The~concept of~U-statistic was discussed earlier and most of~the~other indicators are widely known and do~not require any particular consideration. 
\subsubsection{CLES = $\rho$-statistic = AUC}
First of~all, it~must~be said that all of~these indicators are~equivalent to~each other. Thus
\begin{equation}\label{eq:AUC=CLES}
CLES = f = AUC_{1} = \rho.
\end{equation}
\paragraph{Common language effect size~(CLES)}
\begin{description}
	\item[Common language effect size~(CLES) ---] is the~probability that the~value of~a~randomly chosen observation from the first sample is greater than the~value of~a~randomly chosen observation from the second sample. This indicator is~calculated by~the~formula
	\begin{equation}\label{eq:CLES}
	CLES = \frac{U_{1}}{n_{1}n_{2}}.
	\end{equation}
	The designation \emph{f~(favorable)} is~often used instead of~\emph{CLES}. This sample value is~an~unbiased estimate of~the~value for~the~entire population of~objects belonging to~the~set.	
\end{description}
It~should~be noted that the~value and meaning of~this indicator is~equivalent to~the~value and meaning of~the ~\href{https://en.wikipedia.org/wiki/Receiver_operating_characteristic}{AUC}\cite{Wiki:ROC}. Thus, we~can say that this indicator characterizes the~quality of~ the~U-test as~a~binary classifier.
\begin{equation}\label{eq:AUC}
CLES = f = AUC_{1} = \frac{U_{1}}{n_{1}n_{2}}.
\end{equation}
The~relationship between the~U-statistic and~AUC is~discussed in~\ref{U-AUC}.
%
\paragraph{$\rho$-statistic}
A~statistic called~$\rho$ that is~linearly related to~U and widely used in~studies of~categorization (discrimination learning involving concepts), and elsewhere, is~calculated by~dividing~U by~its maximum value for~the~given sample sizes, which is~simply $n1 \times n2$. Thus, $\rho$ is~a~non-parametric measure of~the overlap between two~distributions; it~can take values between 0 and~1, and it~is~an~estimate of~$P(Y > X) + 0.5 P(Y = X)$, where \textit{X} and~\textit{Y} are~randomly chosen observations from the~two distributions. Both extreme values represent complete separation of~the~distributions, while a~$ρ = 0.5$ represents complete overlap. This statistic is~useful in~particular when despite a~large p-value the~medians of~the~two samples are~essentially equal to~each other. 
%
\subsubsection{Rank-biserial correlation~(RBC)}
The method of~representing the~measure of~impact for~the~U-test is~to~use a~measure of~rank correlation known as~rank-biserial correlation~(hereafter RBC). As~in~the~case of~other measures of~correlation, the~value of~the~RBC coefficient has a~range of~values [-1;1], with a~zero value indicating the~absence of~any relationship. The~RBC coefficient is~usually denoted as~\textit{r}. A~simple formula based on~the~CLES~(AUC, t, $\rho$) value is~used to~calculate it. Let~us state the~hypothesis that in~a~pair of~random observations, one of~which is~taken from the~first sample and the~other from the~second, the~value of~the~first is~greater.Let's write it~down in~mathematical language:
\begin{equation}\label{eq:RBC-hypothesis}
H: x_{i} > y_{j}, \quad x \ \in X,\ y \in Y.
\end{equation}
Then the~value of~the~RBC coefficient is~the~difference between the~proportion of~random pairs of~observations that are~favorable~(f) to~the~hypothesis and the~complementary proportion of~random pairs that are unfavorable to~the~hypothesis. Thus, this formula is~a~formula for~the~difference between the~CLES scores for~each of~the~groups.
\begin{equation}\label{eq:RBC-formula-1}
r = f - u = CLES_{1} - CLES_{2} = f - (1 - f)
\end{equation}
There are also a~number of~alternative formulas that give identical results:
\begin{equation}\label{eq:RBC-formula-2}
r = 2f -1 = \frac{2U_{1}}{n_{1}n_{2}}-1 = 1 - \frac{2U_{2}}{n_{1}n_{2}}.
\end{equation}
%
\subsection{Calculation of~the~p-value and the~final test of~the~null hypothesis}
If~the~number of~observations in~both samples is~large enough, the~U-statistic has~an~approximately normal distribution. Then its~\href{https://en.wikipedia.org/wiki/Standard_score}{standardized value}~(z-score)~\cite{Wiki:z-score} can~be calculated by~the formula
\begin{equation}\label{eq:z-score}
z = \frac{U-m_{U}}{\sigma_{U}},
\end{equation}
where~$m_{U}$ is~mean for~\textit{U} and $\sigma_{U}$ is~its~standard deviation. A~visualization of~the~concept of~\emph{standardized value for~a~normal distribution} is~shown in~Figure~\ref{fig:z-score}.
%
\begin{figure}[ht]
	\centering
	\includegraphics[width=0.8\textwidth]{The_Normal_Distribution.pdf}
	\caption{A~visualization of~the~concept of~standardized value for~a~normal distribution \cite{Wiki:z-score}}\label{fig:z-score}
\end{figure}
%
The mean for~the~U is~calculated by~the~formula
\begin{equation}\label{eq:U-mean}
m_{U} = \frac{n_{1}n_{2}}{2}.
\end{equation}
The~formula for~the standard deviation in~the case of~no~ties is~as~follows:
\begin{equation}\label{eq:standard-deviation-no-ties}
\sigma_{U} =  \sqrt{\frac{n_{1}n_{2}(n_{1}+n_{2}+1)}{12}}.
\end{equation}
In~case of~the~presence of~tied ranks, a~different formula is~used:
\begin{equation}\label{eq:standard-deviation-ties}
\sigma_{U_{ties}} = \sqrt{\frac{n_{1}n_{2}(n_{1}+n_{2}+1)}{12} - \frac{n_{1}n_{2}\sum_{k=1}^{K}({t_{k}}^{3} - t_{k})}{12n(n-1)}} = \sqrt{\frac{n_{1}n_{2}}{12} \left((n+1)-\frac{\sum_{k=1}^{K}({t_{k}}^{3} - t_{k})}{n(n-1)}\right)},
\end{equation}
where~$t_{k}$ is~the~number of~observations with rank~\textit{k} and \textit{K} is~the~total number of~tied ranks. Then, by~obtaining a~standardized value~(z-score) and~using an~approximation of~the~standard normal distribution, the~p-value for~a~given level of~significance (usually~0.05) is~calculated. The~interpretation of~the~result is~as~follows:
\begin{equation}\label{eq:p-interpretation}
\begin{aligned}
p &\leq 0.05 \Rightarrow \text{the~null hypothesis is~rejected}\\
p &> 0.05 \Rightarrow \text{the~null hypothesis can~not~be rejected}.
\end{aligned}
\end{equation}
However, there is~also an~alternative interpretation:
\begin{equation}\label{eq:p-interpretation-2}
\begin{aligned}
p &< 0.05 \Rightarrow \text{the~null hypothesis is~rejected}\\
p &\geq 0.05 \Rightarrow \text{the~null hypothesis can~not~be rejected}.
\end{aligned}
\end{equation}
To~date, there is~no~unambiguous position on~how the~situation when $p = \alpha$ should~be interpreted. This paper uses the~version described in~\ref{eq:p-interpretation}.
%
\section{Relationship to~other statistical tests}
\subsection{Comparison of~Wilcoxon-Mann-Whitney U-test with Student's t-test}
You often hear that the~U-test is~the~nonparametric counterpart of~the~Student's t-test, designed for~data whose distribution differs from the~normal one. From a~purely practical point of~view, we~can indeed say that in~the~case of~a~normal distribution it~is~advisable to~determine whether there is~a~significant difference between the~two samples by~means of~the~t-test, and in~the~case of~a~distribution that differs from the~normal by~means of~the~U-test. Thus, it~can~be said that these tests are~used for~the~same ultimate purpose.

However, the~mathematical meaning of~the~U-test and~the~t-test are~significantly different. As~stated earlier, the~U-test is~designed to~test the~null hypothesis, which is~that for~randomly chosen from two samples of~observations $x \in X$ and~$y \in Y$ the~probability that~\textit{x} is~greater than~\textit{y} is~equal to~the~probability that~\textit{y} is~greater than~\textit{x}, the~alternative hypothesis carries the~claim that these probabilities are~not equal. At~the~same time, the~t-test is~designed to~test the~null hypothesis that the~means of~the~two samples are~equal, while the~alternative hypothesis is~that the~means of~the~two samples are~not~equal. In~this regard, when comparing these tests, we~should keep in~mind that, in~general, the~U-test and~the~t-test check different null hypotheses, although they have partly similar practical meaning. The~result of~the~U-test is~most often very close to~the~result of~the~two-sample t-test for~ranked data. Table~\ref{tab:U-test-t-test-comparison} then provides a~general comparison of~the~U-test with the~t-test.
%
\begin{table}[ht]
	\caption{Properties of~the~U-test relative to~the~t-test.}  \label{tab:U-test-t-test-comparison}
	\centering
	\begin{tabularx}{\textwidth}{p{0.15\linewidth} p{0.8\linewidth}} 
		\hline
		Property&Description\\
		\hline
		Applicability to~ordinal data&When working with ordinal~(rank) data, rather than quantitative data, the~U-test is~preferable to~the~t-test, remembering that the~distance between neighboring values of~the~variation series cannot~be considered constant.\\
		\hline
		Robustness&Since the~U-test handles the~sum of~ranks rather than trait values, it~is less likely than the~t-test to~erroneously indicate significance due~to outliers. However, in~general, the~U-test is~more prone to~type~I error in~the~case when the~data simultaneously have the~property of~heteroscedasticity and~have a~distribution other than normal.\\
		\hline
		Efficiency&In~the~case of~a~normal distribution, the~asymptotic efficiency of~the~U-test is~$\frac{3}{4}\pi \approx 0.95$ of~the~t-test~\cite{U-test-efficiency}. If~the~distribution differs significantly from the~normal one and~the~number of~observations is~large enough, the~efficiency of~the~U-test is~significantly higher than the~efficiency of~the~t-test~\cite{Practical-Nonparametric-Statistics}. However, this efficiency comparison should~be interpreted with caution, because the~U-test and the~t-test examine different hypotheses and~estimate different values. In~the~case, for~example, of~the~need to~compare means, the~use of~the~U-test is~not justified in~principle.\\
		\hline
	\end{tabularx}
\end{table}
%
\subsection{Alternative tests in~the~case of~inequality of~distributions}
If~it~is necessary to~test the~stochastic ordering of~two samples (i.e.~the~alternative hypothesis: $H1:\ P(Y>X)+0.5P(Y=X)\neq0.5$) without assuming equality of~their distributions (i.e.~when the null hypothesis is~$H0:\ P(Y>X)+0.5P(Y=X)=0.5$ but not $F(X)=G(Y)$), more appropriate tests should~be used. These include the~Brunner-Munzel test~\cite{Bruner-Munzel-test-1}, which is~a~heteroskedasticity-resistant analog of~the~U-test, and the~Fligner-Policello test~\cite{Fligner-Policello-test}, which is~a~test for~equality of~medians. In~particular, in~the~case of~a~more general null hypothesis $H0:\ P(Y>X)+0.5P(Y=X)=0.5$, the~U-test can often lead to~a~type~I error even in~the case of~large samples (especially in~the~case of~disparity of~variance and significantly different sample sizes), so~that in~such cases the use of~alternative tests is~preferable~\cite{U-test-vs-Bruner-Munzel-test}. Thus, in~the absence of~the assumption of~equality of~distributions in~case the null hypothesis is~valid, the use of~alternative tests will~be preferable.

In~the case of~testing the hypothesis of~a~shift with significantly different distributions, the U-test may give an~erroneous interpretation of~significance~\cite{U-test-unequal-variance}, so~in~such circumstances it~is preferable to~use a~variant of~the \href{https://en.wikipedia.org/wiki/Welch's_t-test}{t-test}~\cite{Welch-t-test} designed for cases of~unequal variance~\cite{U-test-unequal-variance}. In~some cases, it~may~be justified to~convert quantitative data into ranks and then perform the t-test in~some variant depending on~the assumption of~equality of~variance. When converting quantitative data to~ordinal data, the original variances will not~be preserved; they must~be recalculated for the ranks themselves. In~the case of~equal variance, a~suitable nonparametric substitute for the \href{https://en.wikipedia.org/wiki/F-test}{F-test}~\cite{F-test} can~be the \href{https://en.wikipedia.org/wiki/Brown-Forsythe_test}{Brown-Forsythe test}~\cite{Brown-Forsythe-test}.
%
\subsection{The relationship between the U-test and the classification tasks}\label{U-test&classification}
The U-test is~a particular case of~the \href{https://en.wikipedia.org/wiki/Ordered_logit}{ordered logit model}~\cite{Ordered-logit}.
%
\section{The relationship between the U-test and the concepts of~Receiver operating characteristics~(ROC) and Area under the curve~(AUC)}\label{U-AUC}
Based on~what was said in~\ref{U-test&classification}, we~can conclude that the U-test is~not only a~test for testing the shift hypothesis (or~another one similar in~meaning), but also represents a~kind of~classifier. Looking ahead, the meaning of~the U-test as~a~classifier is~as~follows:
\begin{itemize}
	\item there is~a~"positive" outcome of~comparing two random observations, which is~that the observation from~\textit{X} is~greater than the observation from~\textit{Y};
	\item the proportion of~the sum of~the ranks of~the "positive" elements is~calculated.
	\item as~in~general with ROC, if~the value of~the share of~"positive" elements exceeds~0.5, this means that the classifier generally performs its function; if~it~is equal to~0.5, its efficiency is~equal to~guessing with a~coin flip; if~it~is less than~0.5, using such classifier yields the opposite result.	 
\end{itemize}
At~first glance, the relationship between the U-test and ROC does~not seem obvious. This section will attempt to~understand why these concepts are related and what~is the essence of~the U-test as~a~classifier.

ROC analysis itself is~outside the scope of~this paper. Therefore, let~us consider only its main points.
\begin{description}
	\item[ROC curve ---] is~a~graphical plot that allows us to~evaluate the quality of~binary classification. It~displays the ratio between the proportion of~objects from the total number of~feature carriers correctly classified as~carrying the feature (True Positive Rate~(TPR), called the \emph{sensitivity of the classification algorithm}) and the proportion of~objects from the total number of~objects not carrying the feature, incorrectly classified as~carrying the feature (False Positive Rate~(FPR), the \textbf{1-FPR} value is~called the \emph{specificity of~the classification algorithm}), when varying the threshold of~the deciding rule.	It~is also known as~\textbf{error curve}. Analysis of~classifications using ROC curves is~called \textbf{ROC analysis}.
\end{description}
Quantitative interpretation of~the ROC curve gives the Area under the curve~(AUC).
\begin{description}
	\item[Area under the curve~(AUC) ---] is~the area bounded by~the ROC curve and the axis of~the proportion of~false positive classifications (abscissa axis).
\end{description}
The higher the AUC, the better the quality of~the classifier, while a~value of~0.5 demonstrates the unsuitability of~the chosen classification method (corresponding to~a~random coin guessing). A value of less than 0.5 indicates that the classifier works exactly the other way around: if you call positive results negative and vice versa, the classifier will perform better~\cite{Wiki:ROC}.

Let's introduce some terms.
\begin{description}
	\item[Condition positive~(P) --- ] the number of real positive cases in~the data.
	\item[Condition negative~(N)---] the number of real negative cases in the data.
	\item[True positive~(TP) ---] a~test result that correctly indicates the presence of~a~condition or~characteristic.
	\item[True negative~(TN) ---] a~test result that correctly indicates the absence of~a~condition or~characteristic.
	\item[False positive~(FP) ---] a~test result which wrongly indicates that a~particular condition or~attribute is~present.
	\item[False negative~(FN) ---] a~test result which wrongly indicates that a~particular condition or~attribute is~absent.
\end{description}
Based on~the above, we~can create a~contingency table of~the results of~applying the binary classifier. The rows contain data on~the actual presence or~absence of~the feature, the columns on~the predicted with the classifier.
%
\begin{table}[ht]
	\caption{Binary classifier contingency table.}  \label{tab:ROC-contingency-table}
	\centering
	\begin{tabularx}{\textwidth}{p{0.2\linewidth} p{0.375\linewidth} p{0.375\linewidth}} 
		\hline
	Total $P+N$&Predicted Positive~(PP)&Predicted negative~(PN)\\
		\hline
		Positive~(P)&TP&FN, type~II error~\cite{Wiki:type-1-2-errors}\\
		\hline
		Negative~(N)&FP, type~I error~\cite{Wiki:type-1-2-errors}&TN\\
		\hline
	\end{tabularx}
\end{table}
%
As~can~be seen from Table~\ref{tab:ROC-contingency-table}, the binary classifier can lead to~errors of~two types. Let's introduce some more definitions and define the formulas for calculating the probabilities of~its outcomes~(see tables~\ref{tab:ROC-rates-1}--\ref{tab:ROC-rates-3}).
%
\begin{table}[ht]
	\caption{Additional definitions and formulas for calculating the probabilities of binary classifier outcomes (part~1 of~3).}\label{tab:ROC-rates-1}
	\tiny
	\begin{tabularx}{\textwidth}{p{0.15\linewidth} p{0.4\linewidth} p{0.4\linewidth}} 
		\hline
		Notation&Formula&Deciphering the notation and alternative terms.\\
		\hline
		TPR~(SEN)&\begin{equation}\label{TPR}
		TPR=\frac{TP}{P}=1-FNR=\frac{TP}{TP+FN}
		\end{equation}&\href{https://en.wikipedia.org/wiki/Sensitivity_(test)}{\textbf{true positive rate}}, \href{https://en.wikipedia.org/wiki/Sensitivity_(test)}{\textbf{sensitivity}}~\cite{Wiki:sensitivity-and-specificity}, \href{https://en.wikipedia.org/wiki/Precision_and_recall}{recall}~\cite{Wiki:precision-and-recall}, probability of~detection, \href{https://en.wikipedia.org/wiki/Hit_rate}{hit rate}~\cite{Wiki:hit-rate}, power\\
		\hline
		FPR&\begin{equation}\label{eq:FPR}
		FPR = \frac{FP}{N} = 1 - TNR = \frac{FP}{FP+TN}
		\end{equation}&\href{https://en.wikipedia.org/wiki/False_positive_rate}{\textbf{false positive rate}}, probability of~false alarm, \href{https://en.wikipedia.org/wiki/False_positive_rate}{fall-out}~\cite{Wiki:FPR}\\
		\hline
		FNR&\begin{equation}\label{eq:FNR}
		FNR = \frac{FN}{P} = 1 - TPR = \frac{FN}{FN+TP}
		\end{equation}&\href{https://en.wikipedia.org/wiki/Type_I_and_type_II_errors\#False_positive_and_false_negative_rates}{\textbf{false negative rate}}~\cite{Wiki:TypeI-TypeII-errors}, miss rate\\
		\hline
		TNR~(SPC)&\begin{equation}\label{eq:TNR}
		TNR = \frac{TN}{N} = 1 - FPR = \frac{TN}{TN+FP}
		\end{equation}&\href{https://en.wikipedia.org/wiki/Sensitivity_(test)}{\textbf{true negative rate}}, \href{https://en.wikipedia.org/wiki/Sensitivity_(test)}{\textbf{specificity}}, \href{https://en.wikipedia.org/wiki/Sensitivity_(test)}{selectivity}~\cite{Wiki:sensitivity-and-specificity}\\
		\hline
		PPV&\begin{equation}\label{eq:PPV}
		PPV = \frac{TP}{TP+FP} = 1 - FDR
		\end{equation}&\href{https://en.wikipedia.org/wiki/Positive_and_negative_predictive_values}{\textbf{positive predictive value}}~\cite{Wiki:PPV}, \href{https://en.wikipedia.org/wiki/Information_retrieval\#Precision}{precision}~\cite{Wiki:precision}\\
		\hline
		NPV&\begin{equation}\label{eq:NPV}
		NPV = \frac{TN}{TN+FN} = 1 -FOR
		\end{equation}&\href{https://en.wikipedia.org/wiki/Positive_and_negative_predictive_values}{\textbf{negative predictive value}}~\cite{Wiki:PPV}\\
		\hline
		FDR&\begin{equation}\label{eq:FDR}
		FDR = \frac{FP}{FP + TP} = 1 - PPV
		\end{equation}&\href{https://en.wikipedia.org/wiki/False_discovery_rate}{\textbf{false discovery rate}}~\cite{Wiki:FDR}\\
		\hline
	\end{tabularx}
	\normalsize
\end{table}
%
\begin{table}[ht]
	\caption{Additional definitions and formulas for calculating the probabilities of binary classifier outcomes (part~2 of~3).}\label{tab:ROC-rates-2}
	\tiny
	\begin{tabularx}{\textwidth}{p{0.15\linewidth} p{0.4\linewidth} p{0.4\linewidth}} 
		\hline
		Notation&Formula&Deciphering the notation and alternative terms.\\
		\hline
		FOR&\begin{equation}\label{eq:FOR}
		FOR = \frac{FN}{FN+TN}=1-NPV
		\end{equation}&\href{https://en.wikipedia.org/wiki/Positive_and_negative_predictive_values}{\textbf{false omission rate}}~\cite{Wiki:PPV}\\
		\hline
		LR+&\begin{equation}\label{eq:LR+}
		LR+=\frac{TPR}{FPR}
		\end{equation}&\href{https://en.wikipedia.org/wiki/Likelihood_ratios_in_diagnostic_testing\#positive_likelihood_ratio}{\textbf{\textbf{positive likelihood ratio}}}~\cite{Wiki:likehoods-ratios}\\
		\hline
		LR-&\begin{equation}\label{eq:LR-}
		LR-=\frac{FNR}{TNR}
		\end{equation}&\href{https://en.wikipedia.org/wiki/Likelihood_ratios_in_diagnostic_testing\#negative_likelihood_ratio}{\textbf{negative likelihood ratio}}~\cite{Wiki:likehoods-ratios}\\
		\hline
		PT&\begin{equation}\label{eq:PT}
		PT=\frac{\sqrt{TPR(-TNR+1)}+TNR-1}{TPR+TNR-1}=\frac{\sqrt{FPR}}{\sqrt{TPR}+\sqrt{FPR}}
		\end{equation}&\href{https://en.wikipedia.org/wiki/Sensitivity_(test)}{\textbf{prevalence threshold}}~\cite{Wiki:sensitivity-and-specificity}\\
		TS~(CSI)&\begin{equation}\label{eq:TS|CSI}
		TS = \frac{TP}{TP+TN+FP}
		\end{equation}&\href{https://en.wikipedia.org/wiki/Jaccard_index\#Jaccard_index_in_binary_classification_confusion_matrices}{Jaccard index} \textbf{threat score}, \textbf{critical success index}~\cite{Wiki:jaccard-index}\\
		\hline
		PRV&\begin{equation}\label{eq:PRV}
		PRV = \frac{P}{P+N}
		\end{equation}&\href{https://en.wikipedia.org/wiki/Prevalence}{\textbf{prevalence}}~\cite{Wiki:prevalence}\\
		\hline
		ACC&\begin{equation}\label{eq:ACC}
		ACC = \frac{TP+TN}{P+N} = \frac{TP+TN}{TP+TN+FP+FN}
		\end{equation}&\href{https://en.wikipedia.org/wiki/Accuracy_and_precision}{\textbf{accuracy}}~\cite{Wiki:accuracy-precision}\\
		\hline
	\end{tabularx}
	\normalsize
\end{table}
%
\begin{table}[ht]
	\caption{Additional definitions and formulas for calculating the probabilities of binary classifier outcomes (part~3 of~3).}\label{tab:ROC-rates-3}
	\tiny
	\begin{tabularx}{\textwidth}{p{0.15\linewidth} p{0.4\linewidth} p{0.4\linewidth}} 
		\hline
		Notation&Formula&Deciphering the notation and alternative terms.\\
		\hline
		BA&\begin{equation}\label{eq:BA}
		BA = \frac{TPR+TNR}{2}
		\end{equation}&\textbf{balanced accuracy}\\
		\hline
		F1 score&\begin{equation}\label{eq:F1-score}
		F_{1} = 2 \times \frac{PPV \times TPR}{PPV +TPR} = \frac{2TP}{2TP + FP + FN}
		\end{equation}&\href{https://en.wikipedia.org/wiki/F-score}{F1 score} is~the harmonic mean of~\href{https://en.wikipedia.org/wiki/Information_retrieval\#Precision}{precision} and \href{https://en.wikipedia.org/wiki/Sensitivity_(test)}{sensitivity}~\cite{Wiki:F-score}\\
		\hline
		MCC~($\phi$ or~$r_{\phi}$)&\begin{equation}\label{eq:MCC}
		MCC = \frac{TP \times TN - FP \times FN}{\sqrt{(TP+FP)(TP+FN)(TN+FP)(TN+FN)}}
		\end{equation}&\href{https://en.wikipedia.org/wiki/Phi_coefficient}{\textbf{Matthews correlation coefficient}},\href{https://en.wikipedia.org/wiki/Phi_coefficient}{\textbf{phi coefficient}}~\cite{Wiki:phi-coefficient}\\
		\hline
		FM&\begin{equation}\label{eq:FM}
		FM = \sqrt{\dfrac{TP}{TP+FP} \times \dfrac{TP}{TP+FN}} = \sqrt{PPV \times TPR}
		\end{equation}&\href{https://en.wikipedia.org/wiki/Fowlkes–Mallows_index}{Fowlkes–Mallows index}~\cite{Wiki:Fowlkes–Mallows-index}\\
		\hline
		BM&\begin{equation}\label{eq:BM}
		BM = TPR + TNR -1
		\end{equation}&\textbf{bookmaker informedness}, \href{https://en.wikipedia.org/wiki/Youden's_J_statistic}{informedness}~\cite{Wiki:j-statistic}\\
		\hline
		MK~($\delta P$)&\begin{equation}\label{eq:MK}
		MK = PPV + NPV - 1
		\end{equation}&\href{https://en.wikipedia.org/wiki/Markedness}{\textbf{markedness}}, deltaP~\cite{Wiki:markedness}\\
		\hline
		DOR&\begin{equation}\label{eq:DOR}
		\frac{LR+}{LR-}
		\end{equation}&\href{https://en.wikipedia.org/wiki/Diagnostic_odds_ratio}{\textbf{diagnostic odds ration}}~\cite{Wiki:DOR}\\
		\hline
	\end{tabularx}
	\normalsize
\end{table}
%
The TPR probability can~be written as
\begin{equation}\label{eq:TPR-probability}
P_{TPR} = \mathbb{P}(1,\ x\in C_{1}),
\end{equation}
which means that if~object~\textit{x} belongs to~class~$C_{1}$, this indicator estimates the probability that the binary classifier assigns object~\textit{x} to~this class. The probability of~FPR is~written as
\begin{equation}\label{eq:FPR-probability}
P_{FPR} = \mathbb{P}(1,\ x\in C_{0}),
\end{equation}
which means the probability that an~object belonging to class~$C_0$ will~be mistakenly assigned to~class~$C_1$.

Typically, the working principle of~a~binary classifier is~based on~comparing the measurement of~\textit{x} with some fixed threshold~\textit{c}. It~follows that the previous two expressions can~be rewritten and combined into a~system.
\begin{equation}\label{eq:TRP+FPR-probability}
\begin{cases}
P_{TPR} = \mathbb{P}(x>c,\ x \in C_{1})\\
P_{FPR} = \mathbb{P}(x>c,\ x \in C_{0})
\end{cases}
\end{equation}
It~follows that the ROC curve is~a~diagram
\begin{equation}\label{eq:ROC-contour}
P_{FPR}(c),\ P_{TPR}(c),
\end{equation}
thus, drawing the curve means changing the value of~threshold~\textit{c}.

Let's consider the example~\cite{AUC-Derivation}. Let's take~$f(x\in C_{0}) = \mathcal{N}(0,1)$ and~$f(x\in C_{1}) = \mathcal{N}(2,1)$ as~probability density functions~$C_{0}$ and~$C_{1}$, respectively. Next we~build the ROC curve step by~step using the Python language. At~the first step, consider diagram~\ref{fig:plot-TPR-FPR-prob-density-1}, built using the code given in~script~\ref{lst:plot-TPR-FPR-prob-density}. The area shaded blue shows the probability of~FPR, i.\,e., false-positive significance detection, while the area shaded green shows the probability density of TPR, i.\,e., correct significance detection. The ROC curve shows the values of~these very indicators. The vertical dashed line is~the sensitivity threshold~\textit{c}. In~this situation it~is at~0 on~the abscissa axis. If~it~is moved to~1, the area under the FPR curve (blue) will significantly decrease, i.\,e. the probability of~false-positive detection will decrease, but the TPR area (green) will decrease as~well, which means an~increase in~the probability of~false-negative results. This situation is~illustrated in~Diagram~\ref{fig:plot-TPR-FPR-prob-density-2}.
%
\begin{figure}[ht]
	\centering
	\includegraphics[width=0.95\textwidth]{Plot-ROC-step-1.pdf}
	\caption{Diagram of~TPR and FPR probability distribution densities at~threshold~0.}
	\label{fig:plot-TPR-FPR-prob-density-1}
\end{figure}
%
\begin{lstlisting}[float, caption = Plotting TPR and FPR probability density functions, firstnumber=1, label= lst:plot-TPR-FPR-prob-density]
# Import Libraries
import numpy as np
import matplotlib.pyplot as plt
from scipy import stats

# Plot
f0 = stats.norm(0, 1)
f1 = stats.norm(2, 1)
fig, ax = plt.subplots()
xi = np.linspace(-2, 5, 100)
ax.plot(xi, f0.pdf(xi), label=r'$f(x|C_0)$')
ax.plot(xi, f1.pdf(xi), label=r'$f(x|C_1)$')
ax.legend(fontsize=16, loc=(1, 0))
ax.set_xlabel(r'$x$', fontsize=18)
ax.vlines(0, 0, ax.axis()[-1] * 1.1, linestyles='--', lw=3.)
ax.fill_between(xi, f1.pdf(xi), where=xi > 0, alpha=.3, color='g')
ax.fill_between(xi, f0.pdf(xi), where=xi > 0, alpha=.3, color='b')

# Save to .pdf
plt.savefig('Plot-ROC-step-1.pdf', bbox_inches='tight')

\end{lstlisting}
%
\begin{figure}[ht]
	\centering
	\includegraphics[width=0.95\textwidth]{Plot-ROC-step-2.pdf}
	\caption{Diagram of~TPR and FPR probability distribution densities at~threshold~1.}
	\label{fig:plot-TPR-FPR-prob-density-2}
\end{figure}

As~you can see from the diagrams above, increasing the threshold leads to~the loss of~a~part of~both true-positive and false-positive results, while decreasing~it leads to~an~increase in~the number of~fixations of~the feature presence (both true and false). In~extreme cases, too low a~threshold value will lead to~the fact that all results will~be interpreted as~positive, too high --- to~a~zero number of~observations in~which the feature was detected. The task of~ROC analysis is~to~choose a~rational threshold value.

Let's add the ROC curves corresponding to~thresholds~0 and~1 to~the already existing diagrams. And also create an~interactive diagram using the code from script~\ref{lst:plot-TPR-FPR-prob-density+ROC-interactive}. The PDF format does~not allow you to~add such interactive elements, so~let's consider cases with fixed values of~0 and~1, shown in~Diagrams~\ref{fig:plot-TPR-FPR-prob-density-3} and \ref{fig:plot-TPR-FPR-prob-density-4}, respectively. The left side of~each of~them shows the already familiar probability density function graphs for the TPR and FPR distributions. The right part shows the ROC curve and the point corresponding to~the set threshold value. It~is easy to~guess that the x-coordinate of~the point matches the area under the FPR curve, and the y-coordinate matches the area under the TPR curve. Increasing the threshold value entails shifting the point to~the left, decreasing~it to~the right.

The better the binary classifier itself, the closer to~the upper left corner will~be the ROC curve corresponding to~it, because in~this case a~high TPR value will~be combined with a~low FPR value. The binary classifier, which works as~well (actually badly) as~the coin flip guessing algorithm (in~case the coin is~"fair"), gives a~ROC curve, which is~a~straight line between~(0,0) and~(1,1). In~this case, the left part of~the diagram will show a~complete overlap of~TPR and FPR probability density function curves. Such a~case is~shown in~Diagram~\ref{fig:plot-TPR-FPR-prob-density-3}. For self-practice, you can use Script~\ref{lst:plot-TPR-FPR-prob-density+ROC-interactive} by~running~it in~the Jupyter Lab environment, which allows you to~use the interactive features of~the browser.
%
\begin{lstlisting}[float, caption = Build an~interactive graph of~TPR and FPR distribution density and its corresponding ROC curve for a~given threshold value, firstnumber=1, label= lst:plot-TPR-FPR-prob-density+ROC-interactive]
# Import Libraries
%matplotlib inline
from ipywidgets import interact
import numpy as np
import matplotlib.pyplot as plt
from scipy import stats

# Plot
f0 = stats.norm(0, 1)
f1 = stats.norm(2, 1)
fig, ax = plt.subplots()
xi = np.linspace(-2, 5, 100)
ax.plot(xi, f0.pdf(xi), label=r'$f(x|C_0)$')
ax.plot(xi, f1.pdf(xi), label=r'$f(x|C_1)$')
ax.legend(fontsize=16, loc=(1, 0))
ax.set_xlabel(r'$x$', fontsize=18)
ax.vlines(0, 0, ax.axis()[-1] * 1.1, linestyles='--', lw=3.)
ax.fill_between(xi, f1.pdf(xi), where=xi > 0, alpha=.3, color='g')
ax.fill_between(xi, f0.pdf(xi), where=xi > 0, alpha=.3, color='b')

# Plot ROC-curve and make all interactive
def plot_roc_interact(c=0):
xi = np.linspace(-3,5,100)
fig,axs = plt.subplots(1,2)
fig.set_size_inches((10,3))
ax = axs[0]
ax.plot(xi,f0.pdf(xi),label=r'$f(x|C_0)$')
ax.plot(xi,f1.pdf(xi),label=r'$f(x|C_1)$')
ax.set_xlabel(r'$x$',fontsize=18)
ax.vlines(c,0,ax.axis()[-1]*1.1,linestyles='--',lw=3.)
ax.fill_between(xi,f1.pdf(xi),where=xi>c,alpha=.3,color='g')
ax.fill_between(xi,f0.pdf(xi),where=xi>c,alpha=.3,color='b')
ax.axis(xmin=-3,xmax=5)
crange = np.linspace(-3,5,50)
ax=axs[1]
ax.plot(1-f0.cdf(crange),1-f1.cdf(crange))
ax.plot(1-f0.cdf(c),1-f1.cdf(c),'o',ms=15.)
ax.set_xlabel('False-alarm probability')
ax.set_ylabel('Detection probability')

interact(plot_roc_interact,c=(-3,5,.05))

\end{lstlisting}
%
\begin{figure}[ht]
	\centering
	\includegraphics[width=0.95\textwidth]{Plot-ROC-step-30.pdf}
	\caption{Diagram of~TPR and FPR probability distribution densities at~threshold~0.}
	\label{fig:plot-TPR-FPR-prob-density-3}
\end{figure}
%
\begin{figure}[ht]
	\centering
	\includegraphics[width=0.95\textwidth]{Plot-ROC-step-4.pdf}
	\caption{Diagram of TPR and FPR probability distribution densities at threshold 1.}
	\label{fig:plot-TPR-FPR-prob-density-4}
\end{figure}
%
\begin{figure}[ht]
	\centering
	\includegraphics[width=0.95\textwidth]{Plot-ROC-step-5.pdf}
	\caption{Diagram of~probability densities of~TPR and FPR probability distributions at~equal mean.}
	\label{fig:plot-TPR-FPR-prob-density-5}
\end{figure}
%
\subsection{The concept of~AUC and its calculation}
As~the name implies, the AUC is~the area under the ROC curve bounded by~the point corresponding to~a~given threshold value. In~the normalized space in~which the ROC curve is~usually plotted, the AUC value is~equivalent to~the probability that the classifier assigns a~higher weight to~a~randomly chosen positive entity than to~a~randomly chosen negative entity. The AUC does not depend on~a~specific threshold value, because the ROC curve is~constructed by~fitting~it. This means that the AUC is~calculated by~integrating over the thresholds. The AUC is~given by~the expression:
\begin{equation}\label{eq:AUC-computation-0}
AUC = \int P_{TPR}(P_{FPR}) d P_{FPR}.
\end{equation}
The step-by-step calculation of~the AUC is~as~follows.
\begin{equation}\label{eq:AUC-computation-1}
P_{TPR}(c) = 1 - F_{1}(c),
\end{equation}
where~$F_{1}$ is~the cumulative density function for~$C_{1}$. Similarly calculate
\begin{equation}\label{eq:AUC-computation-2}
P_{FPR}(c) = 1 - F_{0}(c),
\end{equation}
where~$F_{0}$ is~the cumulative density function for~$C_{0}$.


Let~us take some particular value of~$c^{*}$ to~which a~certain~$P_{FPR}(c^{*})$ corresponds. In~other words, it~corresponds to~the probability that a~random element~$x_{0}$ belonging to~class~$C_{0}$ is~greater than the threshold value~$c^{*}$, i.e.
\begin{equation}\label{eq:AUC-computation-3}
P_{FPR}(c^{*}) = \mathbb{P}(x_{0}>c^{*}|x_{0} \in C_{0}).
\end{equation}
Then, reasoning similarly with respect to~TPR, we~get
\begin{equation}\label{eq:AUC-computation-4}
P_{TPR}(c^{*}) = \mathbb{P}(x_{1}>c^{*}|x_{1} \in C_{1}).
\end{equation}
Next, based on~the fact that the AUC is~realized through an~integral, we~select its value so~that the distribution of~$c^{*}$ matches the distribution of~$F_{0}$. In~this case,~$P_{TPR}$ is~an~independent random variable with a~corresponding expectation in~the form of
\begin{equation}\label{eq:AUC-computation-integral}
\mathbb{E}(P_{TPR}) = \int P_{TPR} d P_{FPR} = AUC.
\end{equation}
It~is~now possible to~formulate a~definition for the AUC.
\begin{description}
	\item[AUC ---] is~the expected probability that element~$x_{1} \in C_{1}$ will~be assigned to~$C_{1}$ with higher probability than element~$x_{0} \in C_{0}$. Thus,
	\begin{equation}\label{eq:AUC-definition}
	1-F_{1}(t)>1-F_{0}(t) \forall t.
	\end{equation}
	The wording "for any t" means that~$1-F_{1}(t)$ is~\emph{stochastically} greater than~$1-F_{0}(t)$. The latter circumstance is~key in~terms of~the relationship of~the AUC to~the U-test, which will~be shown later.
\end{description}
%
\subsection{Relation between U-test and AUC}\label{U-test&AUC-relation}
A~fairly detailed description of~the U-test was given earlier. This subsection contains only brief information about~it, which is~directly relevant to~the question of~its relationship to~the AUC.

The U-test is~a~non-parametric test that allows you to~test whether two samples belong to~the same distribution. His basic idea is~that if~there is~no~difference between two classes, then combining them into one larger class (set) and then calculating any statistic for the new larger class will give an~unbiased estimate for any of~the initial classes. In~other words, if~there is~no~difference in~the distribution of~the two samples, combining them and assuming that the actually observed data from the two samples represent only one of~the equal-valued variants of~the moving observations means that there is~no~difference in~any statistical estimate for any of~the moving variants relative to~the other, and relative to~the combined set.

Let's suppose that we~need to~compare two samples using the median, the mean, or~some other measure of~central tendency. In~terms of~cumulative distribution functions for the two populations, in~the case of~$H0$ we~have the following:
\begin{equation}\label{eq:U-AUC-H0}
H_0: F_{X}(t) = F_{Y}(t), \quad \forall t,
\end{equation}
which indicates that all observations belong to~the same distribution. Then an~alternative hypothesis is~that
\begin{equation}\label{eq:U-AUC-H1}
H_1: F_{X}(t) < F_{Y}(t), \quad \forall t,
\end{equation}
which is~possible, in~particular, in~the case of~the existence of~a~shift of~one distribution relative to~the other. In~this case, the samples ${X_{i}}_{i=1}^{n},\ {X_{j}}_{j=1}^{m}$ represent independent groups of~observations. In~this case, the size of~the samples may vary.

The test technique consists of~combining two samples into one set and assigning ranks to~each item within~it. The U-statistic is~the sum of~the ranks for the set~\textit{X}. If~the value of~the statistic is~small enough, it~means that the distribution of~set~\textit{X} is~stochastically shifted to~the left relative to~the distribution of~set~\textit{Y}, i.\,e.~$F_{X}{t} < F_{Y}{t}$.

Since with a~sufficiently large number of~observations (20 or~more) the distribution of~U-statistics is~well approximated by~the normal distribution, the p-value is~suitable for assessing significance. Let's calculate it~using the Python language according to~the script~\ref{lst:AUC-p-value}.
%
\begin{lstlisting}[float, caption = Calculation of~the p-value for the test data, firstnumber=1, label= lst:AUC-p-value]
print('p-value:',stats.wilcoxon(f1.rvs(30), f0.rvs(30))[1])

\end{lstlisting}
The p-value is~1.9729484515803686e-05, which is~less than the significance level~(0.05), so~we~can reject the null hypothesis~\ref{eq:U-AUC-H0}. Since the data were randomly generated, if~the experiment is~repeated, the particular p-value will differ from that obtained when writing this paper. However, it~will always be below the threshold because of~the parameters set in~the algorithm.

The U-statistic can be~written as~follows:
\begin{equation}\label{eq:U-statistics}
U = \frac{1}{mn}\sum_{i=1}^{m}\sum_{j=1}^{n}\mathbbm{1}{(Y_{j}>X_{i})},
\end{equation}
where~$\mathbbm{1}{(Y_{j}>X_{i})}$ is~the indicator (characteristic) function showing that the statistic (for the discrete case) estimates the probability that~\textit{Y} is~stochastically greater than~\textit{X}. Thus, this correspondence means that its value is~equal to~the AUC. The relationship between the AUC and the U-test is~in a~similar sense: checking the stochastic excess value of~observations belonging to~one sample relative to~observations belonging to~another sample.
%
\subsection{Practice of~ROC analysis and AUC calculation.}\label{ROC-AUC-theory}
This subsection is~not required reading if~the goal is~only the practical implementation of~the U-test itself. However, it~gives an~insight into machine learning methods that are not related to~the so-called \emph{frequentist statistics} to~which the U-test itself belongs, and shows the relationship between these areas of~data analysis. In~addition, it~will provide sufficient knowledge to~perform a~ROC analysis as~such, which may~be useful in~other situations that an~appraiser may encounter in~his or~her practice.
%
\subsubsection{Plotting the ROC curve}\label{plot-ROC-theory}
%
\lstset{language=R,
	basicstyle=\ttfamily,
	keywordstyle=\color{Blue}\ttfamily,
	stringstyle=\color{Red}\ttfamily,
	commentstyle=\color{Emerald}\ttfamily,
	morecomment=[l][\color{Magenta}]{\#},
	breaklines=true,
	breakindent=0pt,
	breakatwhitespace,
	columns=fullflexible,
	showstringspaces=false
}
%
ROC analysis and in~particular the construction of~ROC curves are widely used to~find a~compromise between the \emph{sensitivity} and \emph{specificity} of~a~binary classifier. Most of~the classifiers used in~machine learning produce a~result in~the form of~a~quantification that a~given object has a~"positive" feature value. Some threshold value is~needed to~convert such a~quantitative assessment into a~concrete "yes" or~"no" prediction. In~his case, observations with a~score above this threshold will~be classified as~"positive", below as~"negative". Different thresholds provide different levels of~sensitivity and specificity. Setting a~relatively high threshold value provides a~conservative approach to~the issue of~classifying a~particular case as "positive", which reduces the likelihood of~false positives. At~the same time, this increases the risk of~missing the observed positive values, i.\,e., it~reduces the level of~true positive classification results. A~relatively low threshold value provides a~more liberal approach to~classifying observations as~"positive".  This reduces specificity (increases the number of~false negatives) and increases sensitivity (increases the number of~true positives). The ROC curve shows the ratio of~true positives to~false positives, giving an~overview of~the entire spectrum of~such trade-offs. There are many R~language libraries that plot ROC curves and calculate metrics for ROC analysis. In~this case, to~better understand the essence of~ROC analysis, some actions will~be performed by~writing our own functions. The following will show an~algorithm for constructing a~ROC curve based on~a~set of~real outcomes and their corresponding estimates. The calculation involves two steps:
\begin{itemize}
	\item sort the observed outcomes in~descending order by~their predicted scores;
	\item calculation of~total true positive (TPR) and true negative (TNR) scores for ordered observed outcomes.
\end{itemize}
Let's create an appropriate function (script~\ref{lst:create-ROC-function-R}).
%
\begin{lstlisting}[float, caption = Creating a~function to~calculate TPR and FPR, firstnumber=1, label= lst:create-ROC-function-R]
# create own function for ROC
appraiserRoc <- function(labels, scores){
labels <- labels[order(scores, decreasing=TRUE)]
data.frame(TPR=cumsum(labels)/sum(labels),
FPR=cumsum(!labels)/sum(!labels), labels)
}
 
\end{lstlisting}
%
This function has two inputs:
\begin{itemize}
	\item \emph{labels} --- Boolean vector containing actual classification data;
	\item \emph{scores} --- a~vector of~real numbers containing data about the scores predicted by~some classifier.
\end{itemize}
%
Since only two classification outcomes are possible, the labels vector can only contain \emph{TRUE} or~\emph{FALSE} values (or~\emph{1} and~\emph{0} depending on~the analyst's preference). A~sequence of~such binary values can~be interpreted as~a~set of~instructions for a~\href{https://en.wikipedia.org/wiki/Turtle_graphics}{turtle graphics}~\cite{Wiki:turtle-graphics}. There~is one important feature: in~this case the turtle has a~compass and receives instructions for absolute directions of~movement: "to~the north" or~"to~the east" instead of~relative "to~the right" and "to~the left". The turtle starts its movement from the starting point with coordinates~(0,0) and makes its way on~the plane according to~the sequence of~instructions. When a~\emph{TRUE} command is~received, it~takes one step north, i.\,e., in~the positive direction of~the y-axis, and when a~\emph{FALSE} command is~received, it~takes one step east, i.\,e., in~the positive direction of~the x-axis. The length of~the steps is~chosen in~such a~way that if~all \emph{TRUE~(1)} commands are received consecutively, the turtle will~be at~a~point with coordinates~(0,1), all \emph{FALSE~(0)} commands at~a~point with coordinates~(1,0). Thus, the length of~the step "to~the north" may~be different from the length of~the step "to~the east". The path in~the plane is~determined by~the order of~the \emph{TRUE~(1)} and \emph{FALSE~(0)} commands and always ends at~(1,1).

Advancing the turtle through the bits of~the instruction string is~an~adjustment of~the classification threshold to~less and less stringent. Once the turtle has passed the bit, it~means that it~has decided to~classify that bit as~"positive". If~this bit was actually "positive", it~is a~true positive, if~it~was actually "negative" it~is a~false positive. The y-axis shows the TPR, calculated as~the ratio of~the number of~positive results detected to~this time to~the total number of~actual positive results. The x-axis shows the (FPR), calculated as~the ratio of~the number of~currently detected positive results to~the total number of~actual negative results. The vectorized implementation of~this logic uses cumulative sums (the \textbf{cumsum} function) instead of~going through the values one by~one, although that is~what the computer does at~a~lower level.

The ROC curve calculated in~this way is~actually a~step function. With a~very large number of~positive and negative cases, these steps are very small, and the curve looks smooth. In~this case, with a~really large number of~observations, the construction of~each point is~difficult. As~a~consequence, in~practice, most ROC curve functions used for practical purposes contain additional steps and often use some form of~approximation.

As~an~example, consider a~situation in~which an~appraiser evaluates parts manufactured by~an~enterprise. Some of~the parts are known to~be of~good quality and some are defective. The valuation of~quality parts is~carried out on~the basis of~cost market approaches in~the usual manner. And defective parts are valued at~a~scrap value. In~this case, it~is necessary to~assign each part to~one or~another category. There is~some feature~\emph{x}, which can~be measured by~the appraiser. And there is~also some feature~\emph{y}, which cannot~be measured by~the appraiser. The value of~the feature~\emph{y} allows you to~classify parts as~quality or~defective. It~is also known that there is~some finitary relation function between features \emph{x} and~\emph{y}. Thus, knowing the value of~\emph{x}, we~can infer the value of~y with some probability.
In~this case, it~is advisable to~take a~certain sample of~parts. Then, together with the specialists of~the customer company, measure the values of~features \emph{0} and~\emph{y} for each element of~this sample.

We~will use simulated data to~consider the example. There is~some input feature~\emph{x} that is~linearly related to~the implicit result~\emph{y}. This relationship implies the presence of~some randomness. The y-value shows whether the part exceeds the tolerance requirements. If~so, it~should~be classified as~defective. The algorithm used in~this paper involves the following steps:
\begin{itemize}
	\item create the~\textbf{sim\_parts\_data} function that generates data according to~certain rules and sets the $"y>100"$ threshold value to~classify parts as~defective.
	\item create the~dataframe \textbf{parts\_data} with this function;
	\item create the~\textbf{test\_set\_idx} rule, whereby 80\,\% of~the data is~randomly assigned to~the training sample, and 20\,\% to the test sample;
	\item applying the rule \textbf{test\_set\_idx} to~data \textbf{parts\_data};
	\item create training (\textbf{<<training\_set>>}) and testing (\textbf{<<test\_set>>}) sub-samples;
	\item plot the diagram showing the distribution of~observations from the training sample.
\end{itemize}
To~implement the above algorithm, the code from script~\ref{lst:create-sample-data-plot-graph-R} was used.
%
\begin{lstlisting}[float, caption = Creation and primary visualization of~data on~quality and defective parts, firstnumber=1, label= lst:create-sample-data-plot-graph-R]
# Sample of ROC-analysis

# enable libraries
library(ggplot2)
library(dplyr)
library(pROC)

#set seed
set.seed(19190709)

# create own function for ROC
appraiserRoc <- function(labels, scores){
labels <- labels[order(scores, decreasing=TRUE)]
data.frame(TPR=cumsum(labels)/sum(labels),
FPR=cumsum(!labels)/sum(!labels), labels)
}

# create function 
sim_parts_data <- function(N, noise=100){
x <- runif(N, min=0, max=100)
y <- 122 - x/2 + rnorm(N, sd=noise)
bad_parts <- factor(y > 100)
data.frame(x, y, bad_parts)
}

# create dataset
parts_data <- sim_parts_data(2000, 10)

# create rule for test subset
test_set_idx <- sample(1:nrow(parts_data), size=floor(nrow(parts_data)/4))

# create training and test subsets
test_set <- parts_data[test_set_idx,]
training_set <- parts_data[-test_set_idx,]

# plot graph
test_set %>% 
ggplot(aes(x=x, y=y, col=bad_parts)) + 
scale_color_manual(values=c("green", "red")) + 
geom_point() + 
ggtitle("Bad parts related to x")

\end{lstlisting}
%

The result was diagram~\ref{fig:bad-parts-r}. As~you can see, if~the value of~the parameter~\emph{x} is~less than~15, all dots are red, which means that the parts are defective. Above 96 ерун are green, which means that the parts are of good quality. If~the value is~higher than 96, they are green, which means that the parts are of good quality. Between these values is~an area of~uncertainty, the right side of~which is~dominated by~green dots, and the left side by~red dots.
%
\begin{figure}[ht]
	\centering
	\includegraphics[width=0.95\textwidth]{bad-parts-r.pdf}
	\caption{Diagram of~the distribution of~parts with respect to~the parameter~\emph{x}.}
	\label{fig:bad-parts-r}
\end{figure}

The training sub-sample will~be used to~create a~logistic regression model based on~the values of~the attribute~\emph{x}, which allows you to~assign a~particular part to~quality or~defective. This model will~be used to~assign scores to~the observations in~the training sample. In~the future, these scores will~be used to~construct the ROC curve together with the true labels. Recall that the ROC curve is~plotted for observations with known values of~parameters \emph{x} and~\emph{y}. This ROC curve is~then applied to~the entire set of~objects for which x~values are known but y~values are unknown. The scores themselves as~well as~the~\emph{x} and \emph{y}~values are not displayed on~the graph and are only used for sorting labels. Two different classifiers sorting labels in~the same order will give identical ROC curves regardless of~the absolute values of~the scores. This can~be seen by~constructing an~ROC curve based on~"response" or~"link" predictions from a~logistic regression model. The "response" scores were mapped to~a~(0, 1) scale using a~\href{https://en.wikipedia.org/wiki/Sigmoid_function}{Sigmoid function}\cite{Wiki:sigmoid-function}, the "link" scores were left untransformed. In~this case, the points showing specific observations are ordered in~the same way. To test this hypothesis, we~use the code~\ref{lst:link-response-comparison}. As~you can see in~Figure~\ref{fig:link-response-comparison-r}, the order of~the dots is~the same for "link" and "response".
%
\begin{lstlisting}[float, caption = Comparing "link" and "response" predictions, firstnumber=1, label= lst:link-response-comparison]
fit_glm <- glm(bad_parts ~ x, training_set, family=binomial(link="logit"))

glm_link_scores <- predict(fit_glm, test_set, type="link")

glm_response_scores <- predict(fit_glm, test_set, type="response")

score_data <- data.frame(link=glm_link_scores, 
response=glm_response_scores,
bad_parts=test_set$bad_parts,
stringsAsFactors=FALSE)

score_data %>% 
ggplot(aes(x=link, y=response, col=bad_parts)) + 
scale_color_manual(values=c("green", "red")) + 
geom_point() + 
geom_rug() + 
ggtitle("Both link and response scores put cases in the same order")

\end{lstlisting}
%
\begin{figure}[ht]
	\centering
	\includegraphics[width=0.95\textwidth]{link-response-comparison-r.pdf}
	\caption{Comparison of~the order of~points for "link" and "response".}
	\label{fig:link-response-comparison-r}
\end{figure}
%

Let's go directly to~the construction of~the ROC curve. We use both the ready function from the "pROC" package and the previously created \textbf{"appraiserRoc"} function (see script~\ref{lst:plot-ROC-1-r}). The result of~the first is~represented as~an~orange curve, the second as~circles of~red for defective parts and black for quality parts (see Diagram~\ref{fig:test-ROC-r}). It~is not difficult to~guess that the red dot corresponded to~the "North" command and the black dot to~the "East" command. Since the library function and the own function perform the same actions, the two curves are identical.

Note that the "Specificity" scale is~plotted on~the abscissa axis, not the FPR, so~the values on~the axis are inverted. Since, according to~Table~\ref{tab:ROC-rates-1}, $"Specificity = 1 - FPR"$ we can talk about the mutual unambiguity of~these indicators. Consequently, when plotting the ROC curve any of~them can~be used. This version of~the scale display was self-selected by~the \textbf{roc} function from the "pRoc" library. If~the user does not set his settings, the function chooses to~display the scale so~that the AUC value is~always greater than 0.5. This calculation is~based on~which group (quality parts, defective parts) has a~higher median score. Since the \textbf{appraiserRoc} function is~of~course not that smart, a~simple subtraction was performed during its use, making it~possible to~build a~joint diagram.

This approach has one limitation: based on~the prognostic nature of~the ordering of~outcomes, it~does not allow correct processing of~information if~the sequence consists of~identical estimates. "Turtle" assumes that the order of~the labels matters, but there is~no~meaningful order in~the situation of~the same scores. These areas should~be displayed with a~diagonal line, but not the traditional steps.
%
\begin{lstlisting}[float, caption = Plotting the ROC curve using library and own functions, firstnumber=1, label= lst:plot-ROC-1-r]
# plot ROC
plot(roc(test_set$bad_parts, glm_response_scores, direction="<"),
col="orange", lwd=3, main="The turtle finds its way", xlim = c(1, 0))
glm_simple_roc <- appraiser_roc(test_set$bad_parts=="TRUE", glm_link_scores)
with(glm_simple_roc, points(1 - FPR, TPR, col=1 + labels))
\end{lstlisting}
%
\begin{figure}[ht]
	\centering
	\includegraphics[width=0.95\textwidth]{test-ROC-r.pdf}
	\caption{Identical ROC curves plotted with library and own functions.}
	\label{fig:test-ROC-r}
\end{figure}
%

Consider an~example where a~diagonal is~the only adequate way to~plot a~ROC curve. To~do~this, create an~extremely unbalanced data set in~which only~1\,\% of~the observations are "positive". In~this case, the result of~the prediction will always be~negative. Since all scores will~be the same, there is~no~need for any ordering. The \textbf{roc} function from the "pRoc" package correctly recognizes such situations and draws a~diagonal line~(1,0; 0,1). In~doing so, the turtle assumes that the order of~scores has some significance, and moves between these points along a~random trajectory, alternating between "north" and "east" directions. The code calling the construction of~such a~ROC curve is~given in~script~\ref{lst:plot-ROC-rare-success-r}. In~Diagram~\ref{fig:ROC-rare-success-r}, the black diagonal line was plotted by~the library function, while the blue dashed line was plotted by~our own previously written \textbf{appraiserRoc} function.  As~you can see, the library function correctly determined the case of~identical estimates, while applying our own function resulted in~random turtle wanderings.
%
\begin{lstlisting}[float, caption = Plotting the ROC curve in~case of~absence of~order of~scores, firstnumber=1, label= lst:plot-ROC-rare-success-r]
# plot ROC for 99% negative cases
N <- 2000
P <- 0.01
rare_success <- sample(c(TRUE, FALSE), N, replace=TRUE, prob=c(P, 1-P))
guess_not <- rep(0, N)
plot(roc(rare_success, guess_not), print.auc=TRUE)
appr_roc <- appraiserRoc(rare_success, guess_not)
with(appr_roc, lines(1 - FPR, TPR, col="blue", lty=2))
\end{lstlisting}
%
\begin{figure}[ht]
	\centering
	\includegraphics[width=0.95\textwidth]{ROC-rare-success-r.pdf}
	\caption{ROC curve arising when there is~no~value of~the order of~scores.}
	\label{fig:ROC-rare-success-r}
\end{figure}
%

The greater the value of~N, the closer to~the diagonal the turtle will wander. Greater unbalance requires more points in~order for the path to~run roughly close to~the diagonal. In~less extreme cases, the emergence of~diagonal sections is~possible, in~particular, in~the case of~rounding of~estimates, leading to~equality of~some of~them.

To further familiarize yourself with the topic of constructing ROC curves, we can recommend studying this \href{https://web.tresorit.com/l/APSpC#AfkTKO5_-ijMhPuXE-qEzg}{theoretical material}~\cite{ROC-analysis}, as~well as~practice on~the \href{https://kennis-research.shinyapps.io/ROC-Curves/}{online simulator}~\cite{ROC-curve-practice}.
%
\subsubsection{The concept of~AUC and its calculation}\label{calculate-AUC-theory}
The ROC curve is~a~popular means of~visualizing the trade-off between sensitivity and specificity of~a~binary classifier. Earlier in~\ref{plot-ROC-theory}, the issue of~constructing the ROC curve was considered in~terms of~the turtle steps, which takes a~vector with instructions as~steps "north", i.\,e., in~the positive direction on~the y-axis, and "east", i.\,e., in~the positive direction on~the x-axis. In~this case, the sequence of~scores, on~the basis of~which the vector of~bitwise instructions is~formed, is~ordered in~such a~way that the cases that are most likely to~be positive come first. In~doing so, the turtle assumes that all cases are positive. Next, for the training sample (in~which true values are known for certain), we~determine whether the case was true-~(TP) or~false-positive~(FP). The step size on~the y-axis is~inversely proportional to~the number of~positive observations, on~the x-axis to~the number of~negative observations. Thus, the curve path always ends at~(1,1). The result is~a~plot of~the frequency of~true positives (\textit{TPR} or~\textit{sensitivity}) against the frequency of~false positives (\textit{FPR} or~$1-specificity$), which is~actually the ROC curve. At~the same time, the graph itself does~not provide any quantitative estimates of~the quality of~the binary classifier.

Calculating the area under the ROC curve is~one way to~summarize the quantitative assessment of~classifier quality. This metric is~so~common that in~the context of~data analysis, the terms \emph{Area under the curve} or~\emph{AUC} refer specifically to~the area under the ROC curve, unless explicitly stated otherwise.

At~first glance, the simplest and most intuitive metric for classifier performance is~its accuracy. Unfortunately, in~some cases, such a~metric simply will not work. For example, in~the case of~a~disease that occurs in~one person per million, a~completely useless test that always shows a~negative result would~be 99.9999\% accurate. In~contrast to~the accuracy index, ROC curves are insensitive to~unbalanced classes. The aforementioned useless test will have an~$AUC=0.5$, which is~equivalent to~no~test at~all (see diagram~\ref{fig:ROC-rare-success-r}).

In~this subsubsection, we~will first consider the geometric approach to~the concept of~AUC and develop a~function that calculates its value. Next, we~turn to~another --- probabilistic --- interpretation of~the concept of~AUC.

\paragraph{Geometric approach to~the concept of~AUC}
First, let's create a~test dataset using the code shown in~\ref{lst:AUC-theory-create-dataset-r}.
%
\begin{lstlisting}[float, caption = Create a~test data set, firstnumber=1, label= lst:AUC-theory-create-dataset-r]
# Geometric interpretation of AUC

# activate libraries
library(pROC)

# create dataset
category <- c(1, 1, 1, 1, 0, 1, 1, 0, 1, 0, 1, 0, 1, 0, 0, 1, 0, 0, 0, 0)
prediction <- rev(seq_along(category))
prediction[9:10] <- mean(prediction[9:10])
\end{lstlisting}
%
The \textbf{prediction} vector contains pseudo-estimates, which in~practice are assigned by~the classifier. In~this study problem, they are a~decreasing sequence, which generally corresponds to~the "category" labels. The scores for observations with ordinal numbers 9 and 10, one representing the positive case and the other the negative case, are replaced by~their mean values to~create a~ties effect.

To~construct the ROC curve, TPR and FPR must~be calculated. It~was shown in~\ref{plot-ROC-theory} how this can~be done semi-automatically by~calculating cumulative sums for positive and negative labels. This section will use the library "pRoc", which performs all calculations automatically at~a~low level. Let's calculate the TPR, FPR, and AUC values with the code~\ref{lst:AUC-as-geom-calculate-AUC-r}. A~dataframe will also~be created, containing the data for each observation, and shown in~Table~\ref{tab:roc_df-r}. The \textbf{roc} function can return the values of~many indicators, but at~this point we only need TPR and FPR. Recall that TPR means sensitivity and FPR is~equivalent to~the expression $1 - specificity$. By~default, the \textbf{roc} function returns values in~ascending order. As~a~consequence, they have been inverted so~that the starting point has coordinates~(0,0). The AUC value returned by~the function was~0.825. In~the following, we will compare it with the one that will~be obtained in the course of~its independent semi-automatic calculation.
%
\begin{lstlisting}[float, caption = Calculation of~the AUC using the pRoc library, firstnumber=1, label= lst:AUC-as-geom-calculate-AUC-r]
# create ROC object&dataframe and calculate AUC
roc_obj <- roc(category, prediction)
auc(roc_obj)
roc_df <- data.frame(
TPR=rev(roc_obj$sensitivities), 
FPR=rev(1 - roc_obj$specificities), 
labels=roc_obj$response, 
scores=roc_obj$predictor)
\end{lstlisting}
%
\begin{table}[ht]
	\caption{TPR, FPR, labels, scores for the training dataset.}\label{tab:roc_df-r}
	\centering
	\begin{tabular}{lllll}
		\hline
		& TPR & FPR & labels & scores \\ 
		\hline
		1 & 0.00 & 0.00 & 1.00 & 20.00 \\ 
		2 & 0.10 & 0.00 & 1.00 & 19.00 \\ 
		3 & 0.20 & 0.00 & 1.00 & 18.00 \\ 
		4 & 0.30 & 0.00 & 1.00 & 17.00 \\ 
		5 & 0.40 & 0.00 & 0.00 & 16.00 \\ 
		6 & 0.40 & 0.10 & 1.00 & 15.00 \\ 
		7 & 0.50 & 0.10 & 1.00 & 14.00 \\ 
		8 & 0.60 & 0.10 & 0.00 & 13.00 \\ 
		9 & 0.60 & 0.20 & 1.00 & 11.50 \\ 
		10 & 0.70 & 0.30 & 0.00 & 11.50 \\ 
		11 & 0.80 & 0.30 & 1.00 & 10.00 \\ 
		12 & 0.80 & 0.40 & 0.00 & 9.00 \\ 
		13 & 0.90 & 0.40 & 1.00 & 8.00 \\ 
		14 & 0.90 & 0.50 & 0.00 & 7.00 \\ 
		15 & 0.90 & 0.60 & 0.00 & 6.00 \\ 
		16 & 1.00 & 0.60 & 1.00 & 5.00 \\ 
		17 & 1.00 & 0.70 & 0.00 & 4.00 \\ 
		18 & 1.00 & 0.80 & 0.00 & 3.00 \\ 
		19 & 1.00 & 0.90 & 0.00 & 2.00 \\ 
		20 & 1.00 & 1.00 & 0.00 & 1.00 \\ 
		\hline
	\end{tabular}
\end{table}
%
\subparagraph{Plotting the graph}
If~the ROC curve were a~perfect step function, the area under it could~be found by adding a~set of~vertical bars equal to~the spaces between the points on~the abscissa axis (FPR) and to~the height of~the step on~the ordinate axis (TPR). Real ROC curves may include sections corresponding to~repeated values. In~this case, there are segments other than steps. Such repeats require proper consideration. In~Figure~\ref{fig:ROC-bars-R}, the area of~ordinary steps is~shown in~green, the cases of~repeats~(ties) are indicated by~blue rectangles divided in~half by~sloping ROC curve segments. Thus, half of~the area of~these steps is~included in~the total area under the curve.
%
\begin{figure}[ht]
	\centering
	\includegraphics[width=0.95\textwidth]{ROC-bars-R.pdf}
	\caption{ROC curve and the area under it, taking into account the presence of~ties.}
	\label{fig:ROC-bars-R}
\end{figure}

The following steps were taken to~plot diagram~\ref{fig:ROC-bars-R}.
\begin{enumerate}
	\item Define a~\textbf{rectangle} function that takes as~arguments:
	\begin{itemize}
		\item the initial \emph{x} and~\emph{y} coordinates;
		\item width and height of~step;
		\item angle of~rotation after each step;
		\item hatching density.
	\end{itemize}
	The algorithm in~script~\ref{lst:create-rectangle-function-r} builds the rectangle as~follows:
	\begin{itemize}
		\item the function accepts the initial \emph{x} and~\emph{y} coordinates from the appraiser.
		\item then it gets the value of~the step length "to the east" (to~the right on the x-axis), calculates the new value of~the x-coordinate, keeping the value of~the y-coordinate.
		\item after performing the step is~a~turn of~45\textdegree counterclockwise.
		\item then it~gets the value of~the step length "to the North" (up~on~the y-axis), calculates a~new value of~the y-coordinate, keeping the value of~the x-coordinate.
		\item this~is followed by~a~new turn to~the left by~45\textdegree and new steps in~opposite directions.
		\item thus, performing one step each to~the “east”, “north”, “west” and “south”, as~well as~three turns of~45\textdegree counterclockwise function returns to~the starting point, completing the construction of~the rectangle.
	\end{itemize}
	\item To~calculate the step length "to~the East" and "to~the West" it~is necessary to~calculate the difference between neighboring values of~FPR, "to~the North" and "to~the South" between neighboring values of~TPR. To~do~this, we add the two columns \emph{dFPR} and \emph{dTPR}, respectively, using the code shown in~\ref{lst:add-dFPR&dTPR-columns-r}. Since the number of~pairs for which the difference is~calculated is~less than the number of~observations by~one, zero should~be added at~the end (since the dataframe data are sorted in~descending order).
	\item Next, the coordinate grid is~laid out from zero to~one on~each axis using the code~\ref{lst:plot-empty-graph-from-0-to-1-r}.
	\item For the case of~repeating values~(ties), there is~a~special kind of~step "to~the North-East" in~the form of~a~diagonal line.
	\item The final step is~to~build the ROC curve itself and the rectangles that form the area under~it, according to~script~~\ref{lst:plot-ROC-curve-and-rectangles-under-it-r}. The \textbf{mapply} function allows you to~apply the \textbf{rectangle} function sequentially to~each row of~the data frame.
\end{enumerate}
%
\begin{lstlisting}[float, caption = Create the \textbf{rectangle} function, firstnumber=1, label= lst:create-rectangle-function-r]
# create function for plotting rectangles
rectangle <- function(x, y, width, height, density=12, angle=45, ...) 
polygon(c(x,x+width,x+width,x), c(y,y,y+height,y+height), 
density=density, angle=angle, ...)
\end{lstlisting}
%
\begin{lstlisting}[float, caption = Adding \textit{dFPR} and \textit{dTPR} columns, firstnumber=1, label= lst:add-dFPR&dTPR-columns-r]
# add dFPR and dTPR columns
roc_df <- transform(roc_df, 
dFPR = c(diff(FPR), 0),
dTPR = c(diff(TPR), 0))
\end{lstlisting}
%
\begin{lstlisting}[float, caption = Drawing an~empty graph and marking axes from~0 to~1, firstnumber=1, label= lst:plot-empty-graph-from-0-to-1-r]
# plot empty graph from 0 to 1 for each axis
plot(0:10/10, 0:10/10, type='n', xlab="FPR", ylab="TPR",
main = 'ROC-curve and rectangles under it')
abline(h=0:10/10, col="lightblue")
abline(v=0:10/10, col="lightblue")
\end{lstlisting}
%
\begin{lstlisting}[float, caption = Construction of~ROC curve and rectangles under~it, firstnumber=1, label= lst:plot-ROC-curve-and-rectangles-under-it-r]
# plot ROC-curve and rectangles under it
with(roc_df, {
mapply(rectangle, x=FPR, y=0,   
width=dFPR, height=TPR, col="green", lwd=2)
mapply(rectangle, x=FPR, y=TPR, 
width=dFPR, height=dTPR, col="blue", lwd=2)
lines(FPR, TPR, type='b', lwd=3, col="red")
})
\end{lstlisting}

\subparagraph{The summation of~areas by~means of~an~own function}
The area under the curve~(AUC) (highlighted in~red) is~the sum of~the areas of~all the green rectangles and half the area of~the blue one. To~calculate the area of~each rectangle it~is not necessary to~know the absolute coordinates of~its vertices, its width and height are enough. Since one of~the sides of~each rectangle lies on~the x-axis, the height of~any of~them is~determined by~the value of~TPR, the width by~dFPR. Then the total area of~all green rectangles is~equal to~the \href{https://en.wikipedia.org/wiki/Dot_product}{dot~(scalar) product}~\cite{Wiki:dot-product} of~TPR and dFPR. This vector approach calculates the area for each data point, even if~its width or~height is~zero. However, in~this case their further inclusion in~the calculation is of no~importance. The area of~the blue rectangles (if~any) is~determined by~the values of~dFPR and dTPR and also represents their scalar product as~vectors. For areas of~the graph containing northward or~eastward steps, one of~these values (dTPR, dFPR) will~be zero. As~a~consequence, blue rectangles are possible only if~TPR and FPR are changed simultaneously. In~this case, only half of~such rectangle is~under the curve.

Recall that the previously calculated AUC was~0.825. Now let's calculate it in~semi-automatic mode according to~the algorithm described in the previous paragraph. To~do this, first create the function \textbf{appraiser\_auc}, and then apply it to~the test data (see script~\ref{lst:create&apply-appraiser-auc-function-r}). The returned value will~be~0.825, which indicates that the logic and algorithm are correct.
%
\begin{lstlisting}[float, caption = Creating a function to calculate AUC in semi-automatic mode and applying it to test data, firstnumber=1, label= lst:create&apply-appraiser-auc-function-r]
# create function for AUC calculation
appraiser_auc <- function(TPR, FPR){
# inputs already sorted, best scores first 
dFPR <- c(diff(FPR), 0)
dTPR <- c(diff(TPR), 0)
sum(TPR * dFPR) + sum(dTPR * dFPR)/2
}

# apply function to data
with(roc_df, appraiser_auc(TPR, FPR))
\end{lstlisting}

\paragraph{The rank comparison approach}
It~is also possible to~use a~fundamentally different approach to~calculate the AUC. To~implement~it, it~is necessary to~create a~matrix containing all possible combinations of~positive and negative cases. Each row represents a~positive case. They are ordered so that the top row contains the case with the lowest score, and the bottom row contains the case with the highest score. Similarly, the columns contain the negative cases, sorted so that the left column contains the highest scores. Then each cell represents a~comparison of~a~particular positive case with a~particular negative case. If~the score or rank of~a~positive case is higher than that of~a~negative case, that cell takes TRUE. In the case of a~good enough classifier, most positive cases will have higher scores (ranks) than negative cases. All exceptions will~be concentrated in the upper left corner, where positive cases with low scores and negative cases with high scores are located. The~\ref{lst:rank_comparison_auc-r} script contains code to~implement this algorithm and its visualization, shown in~Figure~\ref{fig:rank-comparison-matrix-vizualization-r}.
%
\begin{lstlisting}[float, caption = Create the function to~build a~comparison matrix and apply it to~the test data, firstnumber=1, label= lst:rank_comparison_auc-r]
# create  function for rank comparison
rank_comparison_auc <- function(labels, scores, plot_image=TRUE, ...){
score_order <- order(scores, decreasing=TRUE)
labels <- as.logical(labels[score_order])
scores <- scores[score_order]
pos_scores <- scores[labels]
neg_scores <- scores[!labels]
n_pos <- sum(labels)
n_neg <- sum(!labels)
M <- outer(sum(labels):1, 1:sum(!labels), 
function(i, j) (1 + sign(pos_scores[i] - neg_scores[j]))/2)

AUC <- mean (M)
if (plot_image){
image(t(M[nrow(M):1,]), ...)
library(pROC)
with( roc(labels, scores),
lines((1 + 1/n_neg)*((1 - specificities) - 0.5/n_neg), 
(1 + 1/n_pos)*sensitivities - 0.5/n_pos, 
col="blue", lwd=2, type='b'))
text(0.5, 0.5, sprintf("AUC = %0.4f", AUC))
}

return(AUC)
}

# apply function to data
rank_comparison_auc(labels=as.logical(category), scores=prediction)
\end{lstlisting}
%
\begin{figure}[ht]
	\centering
	\includegraphics[width=0.95\textwidth]{rank-comparison-matrix-vizualization-r.pdf}
	\caption{Visualization of~the matrix of~comparisons of~scores (ranks).}
	\label{fig:rank-comparison-matrix-vizualization-r}
\end{figure}
%

The plotting of~the ROC curve in~this case is~done almost in the usual way. The only difference is that it~is slightly shifted and stretched to~make the coordinates coincide with the corners of~the matrix cells. This way of~constructing the ROC curve makes the following fact obvious: the ROC curve represents the boundary of the area where positive cases have higher scores (ranks) than negative cases. Thus, the AUC can~be calculated by replacing the values in the matrix so that
\begin{itemize}
	\item cells in which the scores (ranks) of positive cases exceed the scores (ranks) of negative cases take the value~1;
	\item cells in which the scores (ranks) are equal take the value~0.5;
	\item cells in which the scores (ranks) of negative cases exceed the scores (ranks) of positive cases take the value~0.
\end{itemize}
Since applying the \textbf{sign} function results in one of three possible values: -1, 0, 1, we use the following trick to place the values in the desired range: we add one to them and divide by two. The final calculation of the AUC is done by calculating the mean value.

\paragraph{Probabilistic approach to calculating AUC}
The probabilistic interpretation is that if you randomly choose a~positive case and a~negative case, the probability that the score (rank) value of the positive case will~be greater than that of the negative case is determined by AUC and equal to it. This follows, in particular, from Diagram~\ref{fig:rank-comparison-matrix-vizualization-r}, in which the total area of the graph is normalized to one. The total area of the graph is normalized to one, the cells of the matrix contain information about all possible combinations of positive and negative cases. The area under the curve consists of cells in which the scores (ranks) of positive cases exceed those of negative cases. To approximate the AUC in the probabilistic approach, let's create and apply a~function according to the script~\ref{lst:create&apply-auc-as-probability-r}. The returned AUC value was 0.8248116, which approximately corresponds to its exact value calculated earlier.
%
\begin{lstlisting}[float, caption = Creating and applying a~function to calculate AUC as a~probability, firstnumber=1, label= lst:create&apply-auc-as-probability-r]
# create function for calculation AUC as probability
auc_probability <- function(labels, scores, N=1e7){
pos <- sample(scores[labels], N, replace=TRUE)
neg <- sample(scores[!labels], N, replace=TRUE)
# sum( (1 + sign(pos - neg))/2)/N # does the same thing
(sum(pos > neg) + sum(pos == neg)/2) / N # give partial credit for ties
}

# apply function to data
auc_probability(as.logical(category), prediction)
\end{lstlisting}
%


%
\nocite{Essential-Statistical-Inference}
\nocite{AUC-optimization}
\nocite{Mann-Whitney-1947}
\nocite{Optimizing-classifier-performance}
\nocite{ROC-R-1}
\nocite{ROC-AUC-1}
\nocite{ROC-AUC-meets-U-R-1}

\printbibliography
\end{document}          
